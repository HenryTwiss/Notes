\chapter{Integrality and Dedekind Domains}
  Throughout, all rings will be understood to be commutative and with unity.
  \section{Integrality}
    Let \(\mc{O}/\smc{O}\) be a ring extension. We say that \(\b \in \mc{O}\) is \textbf{integral}\index{integral} over \(\smc{O}\) if \(\b\) is the root of a monic polynomial \(f(x) \in \smc{O}[x]\). In other words, \(\b\) satisfies an equation of the form
    \[
      \b^{n}+\a_{n-1}\b^{n-1}+\cdots+\a_{0} = 0,
    \]
    for some positive integer \(n\) and \(\a_{i} \in \smc{O}\). We say that \(\mc{O}\) is \textbf{integral}\index{integral} over \(\smc{O}\) if every element of \(\mc{O}\) is integral over \(\smc{O}\). As we may expect, the set of integral elements form a ring.

    \begin{proposition}\label{prop:integral_if_finitely_generated}
      Let \(\mc{O}/\smc{O}\) be a ring extension. Then \(\b_{1},\ldots,\b_{n} \in \mc{O}\) are all integral over \(\smc{O}\) if and only if \(\smc{O}[\b_{1},\ldots,\b_{n}]\) is a finitely generated \(\smc{O}\)-module. In particular, the elements of \(\mc{O}\) that are integral over \(\smc{O}\) form a ring.
    \end{proposition}
    \begin{proof}
      We will prove the forward implication by induction. First suppose \(\b \in \mc{O}\) is integral over \(\smc{O}\). Then there exists a monic polynomial \(f(x) \in \smc{O}[x]\) of say degree \(n\) such that \(f(\b) = 0\). Let \(g(x) \in \smc{O}[x]\) and write
      \[
        g(x) = q(x)f(x)+r(x),
      \]
      with \(q(x),r(x) \in \smc{O}[x]\) where the degree of \(r(x)\) is strictly less than \(n\). Then \(g(\b) = r(\b)\). Letting
      \[
        r(x) = \a_{n-1}x^{n-1}+\cdots+\a_{1}x+a_{0},
      \]
      with \(\a_{i} \in \smc{O}\), it follows that
      \[
        g(\b) = \a_{n-1}\b^{n-1}+\cdots+\a_{1}\b+\a_{0}.
      \]
      As \(g(x)\) was arbitrary, we see that \(1,\b,\ldots,\b^{n-1}\) generate \(\smc{O}[\b]\) as an \(\smc{O}\)-module. Now assume by induction that \(\smc{O}[\b_{1},\ldots,\b_{n-1}]\) is a finitely generated \(\smc{O}\)-module. Then \(\smc{O}[\b_{1},\ldots,\b_{n}]\) is a finitely generated \(\smc{O}\)-module. This proves the forward implication of the first statement. For the reverse implication, suppose \(\smc{O}[\b_{1},\ldots,\b_{n}]\) is a finitely generated \(\smc{O}\)-module and let \(\w_{1},\ldots,\w_{r}\) be generators. Then for any \(\b \in \smc{O}[\b_{1},\ldots,\b_{n}]\), we have
      \[
        \b\w_{i} = \sum_{j}\a_{i,j}\w_{j},
      \]
      with \(\a_{i,j} \in \smc{O}\). These \(r\) equations are equivalent to the identity
      \[
        \begin{pmatrix} \b-\a_{1,1} & \a_{1,2} & \cdots & -\a_{1,r} \\ -\a_{2,1} & \b-\a_{2,2} & & \\ \vdots & & \ddots & \\ -\a_{r,1} & & & \b-\a_{r,r} \end{pmatrix}\begin{pmatrix} \w_{1} \\ \w_{2} \\ \vdots \\ \w_{r} \end{pmatrix} = \mathbf{0}.
      \]
      The determinant of the matrix on the left-hand side must be zero. This shows that \(\b\) is the root of the characteristic polynomial \(\det(xI-(\a_{i,j}))\) which is a monic polynomial with coefficients in \(\smc{O}\). Hence \(\b\) is integral over \(\smc{O}\). As \(\b\) was arbitrary, the elements \(\b_{1},\ldots,\b_{n}\) are all integral over \(\smc{O}\) and that the sum and product of elements that are integral over \(\smc{O}\) are also integral over \(\smc{O}\). This proves the reverse implication and the second statement.
    \end{proof}

    Integrality is also transitive.

    \begin{proposition}\label{prop:integrality_is_transitive}
      Let \(\wtilde{\mc{O}}/\mc{O}/\smc{O}\) be a tower of rings. If \(\wtilde{\mc{O}}\) is integral over \(\mc{O}\) and \(\mc{O}\) is integral over \(\smc{O}\) then \(\wtilde{\mc{O}}\) is integral over \(\smc{O}\).
    \end{proposition}
    \begin{proof}
      Let \(\g \in \wtilde{\mc{O}}\). Since \(\wtilde{\mc{O}}\) is integral over \(\mc{O}\), we have
      \[
        \g^{n}+\b_{n-1}\g^{n-1}+\cdots+\b_{0} = 0,
      \]
      for some positive integer \(n\) and \(\b_{i} \in \mc{O}\). Set \(R = \smc{O}[\b_{0},\ldots,\b_{n-1}]\). Then \(R[\g]\) is a finitely generated \(R\)-module. As \(\mc{O}\) is integral over \(\smc{O}\), \cref{prop:integral_if_finitely_generated} implies that \(R[\g]\) is also a finitely generated \(\smc{O}\)-module whence \(\g\) is integral over \(\smc{O}\). As \(\g\) was arbitrary, this proves \(\wtilde{\mc{O}}\) is integral over \(\smc{O}\).
    \end{proof}

    In light of this result, we define the \textbf{integral closure}\index{integral closure} \(\conj{\smc{O}}\) of \(\smc{O}\) in \(\mc{O}\) by
    \[
      \conj{\smc{O}} = \{\b \in \mc{O}:\b \text{ is integral over } \smc{O}\}.
    \]
    Then \(\conj{\smc{O}}\) is a subring of \(\mc{O}\) by \cref{prop:integral_if_finitely_generated}. Clearly \(\smc{O} \subseteq \conj{\smc{O}}\). Moreover, we say that \(\smc{O}\) is \textbf{integrally closed}\index{integrally closed} in \(\mc{O}\) if \(\smc{O} = \conj{\smc{O}}\). As \(\smc{O} \subseteq \conj{\smc{O}} \subseteq \conj{\conj{\smc{O}}}\), \cref{prop:integrality_is_transitive} implies that \(\conj{\smc{O}}\) is automatically integrally closed in \(\mc{O}\). It will often be more fruitful to work with the integral closure rather than a generic ring.
    
    There is a particular situation of integrally closed rings which is deserving of additional interest. Suppose \(\smc{O}\) is an integral domain with field of fractions \(K\). We call the integral closure \(\conj{\smc{O}}\) of \(\smc{O}\) in \(K\) the \textbf{normalization}\index{normalization} of \(\smc{O}\) and simply say that \(\smc{O}\) is an \textbf{integrally closed domain}\index{integrally closed domain} if \(\smc{O}\) is equal to its normalization. It turns out that every unique factorization domain is an integrally closed domain.

    \begin{lemma}\label{lemma:unique_factorization_domains_are_integrally_closed}
      Let \(\smc{O}\) be a unique factorization domain with field of fractions \(K\). Then \(\smc{O}\) is an integrally closed domain. In particular, every principal ideal domain is an integrally closed domain.
    \end{lemma}
    \begin{proof}
      Let \(\k \in K\) be such that
      \[
        \k^{n}+\a_{n-1}\k^{n-1}+\cdots+\a_{0} = 0,
      \]
      for some positive integer \(n\) and \(\a_{i} \in \smc{O}\). Write \(\k = \frac{\a}{\b}\) for \(\a,\b \in \smc{O}\) with \(\b\) nonzero and \((\a,\b) = 1\). Multiplying by \(\b^{n}\) and isolating the leading term shows
      \[
        \a^{n} = -(\a_{n-1}\b\a^{n-1}+\cdots+\a_{0}\b^{n}).
      \] 
      As \(\b\) divides the right-hand side it divides the left-hand side as well. But then \(\b\) is a unit in \(\smc{O}\) since \((\a,\b) = 1\). This means \(\k \in \smc{O}\) whence \(\smc{O}\) is an integrally closed domain. This proves the first statement. The second statement is immediate since every principal ideal domain is a unique factorization domain.
    \end{proof}
    
    We will often consider the more refined setting where \(\smc{O}\) is an integrally closed domain with field of fractions \(K\), \(L/K\) is a finite separable extension, and \(\mc{O}\) is the integral closure of \(\smc{O}\) in \(L\). In this setting, \(L\) is the field of fractions of \(\mc{O}\) and the elements of \(L\) which are integral over \(\smc{O}\) have a simple description.

    \begin{proposition}\label{prop:field_of_fractions_AKBL}
      Let \(\smc{O}\) be an integrally closed domain with field of fractions \(K\), \(L/K\) be a finite separable extension, and \(\mc{O}\) be the integral closure of \(\smc{O}\) in \(L\). Then every \(\l \in L\) is of the form
      \[
        \l = \frac{\b}{\a},
      \]
      for some \(\b \in \mc{O}\) and nonzero \(\a \in \smc{O}\). In particular, \(\mc{O}\) is an integrally closed domain with field of fractions \(L\). Moreover, \(\l \in L\) is integral over \(\smc{O}\) if and only if the minimal polynomial \(m_{\l}(x)\) of \(\l\) over \(K\) has coefficients in \(\smc{O}\).
    \end{proposition}
    \begin{proof}
      As \(L/K\) is finite, it is necessarily algebraic so that any \(\l \in L\) satisfies
      \[
        \a\l^{n}+\a_{n-1}\l^{n-1}+\cdots+\a_{0} = 0,
      \]
      for some positive integer \(n\) and \(\a,\a_{i} \in \smc{O}\) with \(\a\) nonzero. We claim that \(\a\l\) is integral over \(\smc{O}\). Indeed, multiplying the previous identity by \(\a^{n-1}\) yields
      \[
        (\a\l)^{n}+\a'_{n-1}(\a\l)^{n-1}+\cdots+\a'_{0} = 0,
      \]
      where \(\a'_{i} = \a_{i}\a^{n-1-i}\). Whence \(\a\l\) is integral over \(\smc{O}\) and thus \(\a\l \in \mc{O}\). Then \(\a\l = \b\) for some \(\b \in \mc{O}\) which is equivalent to \(\l = \frac{\b}{\a}\). Therefore \(L\) is the field of fractions of \(\mc{O}\). Now \(\mc{O}\) is an integral domain as it is a subring of a field. Being the integral closure of \(\smc{O}\) in \(L\), \(\mc{O}\) is an integrally closed domain with field of fractions \(L\). It remains to prove the last statement. For the forward implication, suppose \(\l\) is integral over \(\smc{O}\). Then \(\l\) is a root of a monic polynomial \(f(x) \in \smc{O}[x]\). As \(m_{\l}(x)\) divides \(f(x)\) in \(\smc{O}[x]\), all of the roots of \(m_{\l}(x)\) are integral over \(\smc{O}\) too. By Vieta's formulas, the coefficients of \(m_{\l}(x)\) integral over \(\smc{O}\) as well whence \(m_{\l}(x) \in \smc{O}[x]\). For the reverse implication, if the minimal polynomial \(m_{\l}(x)\) of \(\l\) over \(K\) has coefficients in \(\smc{O}\) then \(\l\) is automatically integral over \(\smc{O}\).
    \end{proof}

    The previous result can also be generalized to towers in light of the fact that integral closure is transitive.

    \begin{proposition}\label{prop:integral_closure_is_transitive}
      Let \(\smc{O}\) be an integrally closed domains with field of fractions \(K\), let \(M/L/K\) be a finite separable tower, and suppose \(\mc{O}\) and \(\wtilde{\mc{O}}\) are the integral closures of \(\smc{O}\) in \(L\) and \(M\) respectively. Then \(\mc{O}\) and \(\wtilde{\mc{O}}\) are integrally closed domains. In particular, \(\wtilde{\mc{O}}\) is the integral closure of \(\mc{O}\) in \(L\).
    \end{proposition}
    \begin{proof}
      The first statement follows from two applications of \cref{prop:field_of_fractions_AKBL}. To prove the second statement, we must show \(\wtilde{\mc{O}} = \conj{\mc{O}}\). The forward inclusion is obvious as \(\smc{O}\) is a subset of \(\mc{O}\). For the reverse inclusion, \cref{prop:integrality_is_transitive} implies that \(\conj{\mc{O}}\) is integral over \(\smc{O}\). As \(\wtilde{\mc{O}}\) is the integral closure of \(\smc{O}\) in \(M\), the reverse containment follows.
    \end{proof}

    Number fields are a particular instance of the aforementioned setting. A \textbf{number field}\index{number field} \(K\) is a field extension of \(\Q\) of finite degree. That is, \(K\) is a finite dimensional \(\Q\)-vector space. In particular, \(K/\Q\) is a finite separable extension since \(\Q\) is perfect whence the primitive element theorem applies. Moreover, \(K/\Q\) is Galois if and only if it is normal. We say that the \textbf{degree}\index{degree} of \(K\) is \([K:\Q]\) which is simply the degree of \(K\) as a \(\Q\)-vector space. In the cases of small degrees we will often use the latin derived names \textbf{quadratic}\index{quadratic}, \textbf{cubic}\index{cubic}, etc. Any \(\k \in K\) is called an \textbf{algebraic number}\index{algebraic number}. We define the \textbf{ring of integers}\index{ring of integers} \(\mc{O}_{K}\) of \(K\) to be the integral closure of \(\Z\) in \(K\). In other words,
    \[
      \mc{O}_{K} = \{\k \in K:\k \text{ is integral over } \Z\}.
    \]
    Any \(\a \in \mc{O}_{K}\) is called an \textbf{algebraic integer}\index{algebraic integer}. Note that \(\a\) is an algebraic integer if and only if it is the root of a monic polynomial \(f(x) \in \Z[x]\).

    \begin{remark}\label{rem:integers_are_integrally_closed}
      By \cref{lemma:unique_factorization_domains_are_integrally_closed}, \(\Z\) is an integrally closed domain and therefore the ring of integers of \(\Q\) is exactly \(\Z\).
    \end{remark}

    It follows from \cref{prop:field_of_fractions_AKBL} that every \(\k \in K\) is of the form
    \[
      \k = \frac{\a}{a},
    \]
    for some algebraic integer \(\a\) and nonzero integer \(\smc{O}\). In particular, \(K\) is the field of fractions of \(\mc{O}_{K}\) and \(\mc{O}_{K}\) is an integrally closed domain. Moreover, \(\k\) is an algebraic integer if and only if the minimal polynomial \(m_{\k}(x)\) of \(\k\) over \(\Q\) has coefficients in \(\Z\).
  \section{Traces and Norms}
    Let \(\mc{O}/\smc{O}\) be a ring extension such that \(\mc{O}\) is a free \(\smc{O}\)-module of rank \(n\). Recall that by fixing a basis for \(\mc{O}/\smc{O}\), there is an isomorphism
    \[
      \End_{\smc{O}}(\mc{O}) \cong \Mat_{n}(\smc{O})
    \]
    This permits us to view \(\smc{O}\)-linear operators from \(\mc{O}\) to itself as \(n \x n\) matrices with coefficients in \(\smc{O}\). To any \(\b \in \mc{O}\), we associate the \(\smc{O}\)-linear operator \(T_{\b}\) defined by
    \[
      T_{\b}:\mc{O} \to \mc{O} \qquad x \mapsto \b x.
    \]
    This is simply multiplication by \(\b\). The trace and determinant of \(T_{\b}\) are deserving of special interest. For instance, let \(f_{\b}(x)\) denote the characteristic polynomial of \(T_{\b}\) so that
    \[
      f_{\b}(x) = \det(xI-T_{\b}) = x^{n}-\a_{n-1}x^{n-1}+\cdots+(-1)^{n}\a_{0},
    \]
    with \(\a_{i} \in \smc{O}\). Then the trace and determinant have the alternative representations
    \[
      \tr(T_{\b}) = \a_{n-1} \quad \text{and} \quad \det(T_{\b}) = \a_{0},
    \]
    We define the \textbf{trace}\index{trace} and \textbf{norm}\index{norm} of \(\mc{O}\) over \(\smc{O}\), denoted \(\Trace_{\mc{O}/\smc{O}}\) and \(\Norm_{\mc{O}/\smc{O}}\), by
    \[
      \Trace_{\mc{O}/\smc{O}}:\mc{O} \to \smc{O} \qquad \b \mapsto \tr(T_{\b}) \quad \text{and} \quad \Norm_{\mc{O}/\smc{O}}:\mc{O} \to \smc{O} \qquad \b \mapsto \det(T_{\b})
    \]
    Our primary aim will be to develop a sufficient understanding of the trace and norm. The trace is additive while the norm is multiplicative because they correspond to the trace and determinant of matrices. Whence
    \[
      \Trace_{\mc{O}/\smc{O}}(\b+\b') = \Trace_{\mc{O}/\smc{O}}(\b)+\Trace_{\mc{O}/\smc{O}}(\b') \quad \text{and} \quad \Norm_{\mc{O}/\smc{O}}(\b\b') = \Norm_{\mc{O}/\smc{O}}(\b)\Norm_{\mc{O}/\smc{O}}(\b'),
    \]
    for all \(\b,\b' \in \mc{O}\). In fact, we have the additional relations
    \[
      \Trace_{\mc{O}/\smc{O}}(\a\b) = \a\Trace_{\mc{O}/\smc{O}}(\b) \quad \text{and} \quad \Norm_{\mc{O}/\smc{O}}(\a\b) = \a^{n}\Norm_{\mc{O}/\smc{O}}(\b),
    \]
    for all \(\a \in \smc{O}\) by scalar multiplication of matrices. The trace and norm also behave well with respect to direct sums. If \(\mc{O} = \mc{O}_{1} \op \mc{O}_{2}\) and we have a decomposition \(\b = \b_{1}+\b_{2}\) with \(\b_{1} \in \mc{O}_{1}\) and \(\b_{2} \in \mc{O}_{2}\) then
    \[
      \Trace_{\mc{O}/\smc{O}}(\b) = \Trace_{\mc{O}_{1}/\smc{O}}(\b_{1})+\Trace_{\mc{O}_{2}/\smc{O}}(\b_{2}) \quad \text{and} \quad \Norm_{\mc{O}/\smc{O}}(\b) = \Norm_{\mc{O}_{1}/\smc{O}}(\b_{1})\Norm_{\mc{O}_{2}/\smc{O}}(\b_{2}),
    \]
    because the corresponding matrix for \(\b\) is block diagonal.

    In the case of a degree \(n\) field extension \(L/K\), we call \(\Trace_{L/K}\) and \(\Norm_{L/K}\) the \textbf{trace}\index{trace} and \textbf{norm}\index{norm} of \(L/K\). For any \(\l \in L\), the \textbf{trace}\index{trace} and \textbf{norm}\index{norm} of \(\l\) are \(\Trace_{L/K}(\l)\) and \(\Norm_{L/K}(\l)\). In this case, \(\Norm_{L/K}(\l) = 0\) if and only if \(\l = 0\) because otherwise \(T_{\l}\) has inverse \(T_{\l^{-1}}\) and hence a nonzero determinant. Therefore we obtain homomorphisms
    \[
      \Trace_{L/K}:L \to K \quad \text{and} \quad \Norm_{L/K}:L^{\ast} \to K^{\ast}.
    \]
    When \(L/K\) is also separable, we can derive alternative descriptions of the trace and norm of \(L/K\). This additional assumption is weak because we are mostly interested in field extensions of \(\Q\) and \(\F_{p}\) of finite degree which are always separable as these fields are perfect. In any case, to do this we need to work in the algebraic closure \(\conj{K}\) of \(K\). As \(L/K\) is a degree \(n\) separable extension, there are exactly \(n\) distinct \(K\)-embeddings \(\s_{1},\ldots,\s_{n}\) of \(L\) into \(\conj{K}\). That is, there are \(n\) elements of \(\Hom_{K}(L,\conj{K})\). These \(K\)-embeddings are constructed by letting \(\t\) be a primitive element for \(L/K\) and sending \(\t\) to one of its conjugate roots in the minimal polynomial \(m_{\t}(x)\) of \(\t\) over \(K\). These \(K\)-embeddings are deeply connected to the trace and norm.

    \begin{proposition}\label{prop:formulas_for_trace_and_norm}
      Let \(L/K\) be a degree \(n\) separable extension and let \(\s\) run over the elements of \(\Hom_{K}(L,\conj{K})\). For any \(\l \in L\), the characteristic polynomial \(f_{\l}(x)\) of \(T_{\l}\) over \(K\) is a power of the minimal polynomial \(m_{\l}(x)\) of \(\l\) over \(K\) and satisfies
      \[
        f_{\l}(x) = \prod_{\s}(x-\s(\l)).
      \]
      We also have
      \[
        \Trace_{L/K}(\l) = \sum_{\s}\s(\l) \quad \text{and} \quad \Norm_{L/K}(\l) = \prod_{\s}\s(\l).
      \]
      Moreover, if \(L/K\) is Galois and \(\l_{1},\ldots,\l_{n}\) are the conjugates of \(\l\) then
      \[
        \Trace_{L/K}(\l) = \sum_{i}\l_{i} \quad \text{and} \quad \Norm_{L/K}(\l) = \prod_{i}\l_{i}.
      \]
    \end{proposition}
    \begin{proof}
      Let $m$ and $d$ be the degrees of \(K(\l)/K\) and \(L/K(\l)\) respectively. Write
      \[
        m_{\l}(x) = x^{m}+\k_{m-1}x^{m-1}+\cdots+\k_{0},
      \]
      with \(\k_{i} \in K\). We claim
      \[
        f_{\l}(x) = m_{\l}(x)^{d}.
      \]
      Indeed, recall that \(1,\l,\ldots,\l^{m-1}\) is a basis of \(K(\l)/K\). If \(\a_{1},\ldots,\a_{d}\) is a basis for \(L/K(\l)\) then
      \[
        \a_{1},\a_{1}\l,\ldots,\a_{1}\l^{m-1},\ldots,\a_{d},\a_{d}\l,\ldots,\a_{d}\l^{m-1},
      \]
      is a basis for \(L/K\). The matrix of \(T_{\l}\) with respect to this basis is block diagonal with \(d\) blocks each of the form
      \[
        \begin{pmatrix} & 1 & & \\ & & \ddots & \\ & & & 1 \\ -\k_{0} & -\k_{1} & \cdots & -\k_{m-1} \\ \end{pmatrix}.
      \]
      This matrix is the companion matrix to \(m_{\l}(x)\) and hence the characteristic polynomial is \(m_{\l}(x)\) as well. Our claim follows since the characteristic polynomial of a block diagonal matrix is the product of the characteristic polynomials of the blocks. Since \(\l\) is algebraic over \(K\) of degree \(m\), \(K(\l)\) is the splitting field of \(m_{\l}(x)\) and there are \(m\) elements of \(\Hom_{K}(K(\l),\conj{K})\). Then the elements of \(\Hom_{K}(L,\conj{K})\) are partitioned into \(m\) many equivalence classes each of size \(d\) where two \(K\)-embeddings are equivalent if and only if they take the same value at \(\l\). If \(\tau\) runs over the elements of \(\Hom_{K}(K(\l),\conj{K})\) then this a complete set of representatives. As
      \[
        m_{\l}(x) = \prod_{\tau}(x-\tau(\l))
      \]
      it follows that
      \[
        f_{\l}(x) = \prod_{\s}(x-\s(\l)).
      \]
      The formulas for the trace and norm follow from Vieta's formulas applied to this product for \(f_{\l}(x)\). Now suppose \(L/K\) is Galois. Then \(\Hom_{K}(L,\conj{K}) = \Gal(L/K)\). Therefore the conjugates of \(\l\) are exactly the images of \(\l\) under these \(K\)-embeddings and the last statement follows.
    \end{proof}

    As an application of this result, we can show how the trace and norm act when \(\smc{O}\) is an integrally closed domain with field of fractions \(K\), \(L/K\) is a finite separable extension, and \(\mc{O}\) is the integral closure of \(\smc{O}\) in \(L\).

    \begin{proposition}\label{prop:trace_and_norm_AKBL}
      Let \(\smc{O}\) be an integrally closed domain with field of fractions \(K\), \(L/K\) be a finite separable extension, and \(\mc{O}\) be the integral closure of \(\smc{O}\) in \(L\). If \(\b \in \mc{O}\) then the trace and norm of \(\b\) are in \(\smc{O}\).
    \end{proposition}
    \begin{proof}
      By \cref{prop:field_of_fractions_AKBL}, the minimal polynomial \(m_{\b}(x)\) of \(\b\) over \(K\) has coefficients in \(\smc{O}\). By \cref{prop:formulas_for_trace_and_norm}, the characteristic polynomial \(f_{\b}(x)\) is a power of \(m_{\b}(x)\) and hence \(f_{\b}(x)\) has coefficients in \(\smc{O}\) too. Whence trace and norm of \(\b\) are in \(\smc{O}\) as they are coefficients of \(f_{\b}(x)\) up to sign.
    \end{proof}

    In this setting it is also easy to classify the units of \(\mc{O}\) in terms of the units of \(\smc{O}\).

    \begin{proposition}\label{prop:unit_if_and_only_if_AKBL}
      Let \(\smc{O}\) be an integrally closed domain with field of fractions \(K\), \(L/K\) be a finite separable extension, and \(\mc{O}\) be the integral closure of \(\smc{O}\) in \(L\). Then \(\b \in \mc{O}\) is a unit if and only if \(\Norm_{L/K}(\b) \in \smc{O}\) is a unit.
    \end{proposition}
    \begin{proof}
      For the forward implication, if \(\b \in \mc{O}\) is a unit then \(\frac{1}{\b} \in \mc{O}\). Whence
      \[
        \Norm_{L/K}(\b)\Norm_{L/K}\left(\frac{1}{\b}\right) = 1.
      \]
      By \cref{prop:trace_and_norm_AKBL}, both of the norms on the left-hand side are in \(\smc{O}\). This proves \(\Norm_{L/K}(\b) \in \smc{O}\) is a unit. For the reverse implication, we know by assumption that \(\b\) is nonzero. Also, \cref{prop:field_of_fractions_AKBL} implies that the minimal polynomial \(m_{\b}(x)\) of \(\b\) over \(K\) has coefficients in \(\smc{O}\). The constant term is a unit since it is \(\Norm_{L/K}(\b)\) up to sign. Letting the degree of \(m_{\b}(x)\) be \(m\), we have shown that
      \[
        m_{\b}(x) = x^{m}+\a_{m-1}x^{m-1}+\cdots+\a,
      \]
      with \(\a,\a_{i} \in \smc{O}\) where \(\a\) is a unit. Dividing this relation by \(\a\b^{m}\), we find that \(\frac{1}{\b}\) is a root of the monic polynomial
      \[
        f(x) = x^{m}+\frac{\a_{1}}{\a}x^{m-1}+\cdots+\frac{1}{\a},
      \]
      whose coefficients are in \(\smc{O}\). Hence \(\frac{1}{\b} \in \mc{O}\) and thus \(\b\) is a unit.
    \end{proof}

    Having introduced traces and norms, we turn to discussing discriminants of free modules. Let \(\mc{O}/\smc{O}\) be a ring extension such that \(\mc{O}\) is a free \(\smc{O}\)-module of rank \(n\). If \(\b_{1},\ldots,\b_{n}\) is a basis for \(\mc{O}/\smc{O}\), we define its \textbf{trace matrix}\index{trace matrix} \(\Trace_{\mc{O}/\smc{O}}(\b_{1},\ldots,\b_{n})\) by
    \[
      \Trace_{\mc{O}/\smc{O}}(\b_{1},\ldots,\b_{n}) = \begin{pmatrix} \Trace_{\mc{O}/\smc{O}}(\b_{1}\b_{1}) & \cdots & \Trace_{\mc{O}/\smc{O}}(\b_{1}\b_{n}) \\ \vdots & & \vdots \\ \Trace_{\mc{O}/\smc{O}}(\b_{n}\b_{1}) & \cdots & \Trace_{\mc{O}/\smc{O}}(\b_{n}\b_{n}) \end{pmatrix}.
    \]
    The \textbf{discriminant}\index{discriminant} \(d_{\mc{O}/\smc{O}}(\b_{1},\ldots,\b_{n})\) of \(\b_{1},\ldots,\b_{n}\) is defined to be the determinant of the trace matrix. That is,
    \[
      d_{\mc{O}/\smc{O}}(\b_{1},\ldots,\b_{n}) = \det(\Trace_{\mc{O}/\smc{O}}(\b_{1},\ldots,\b_{n})).
    \]
    In particular, \(d_{\mc{O}/\smc{O}}(\b_{1},\ldots,\b_{n}) \in \smc{O}\) since the entries of the trace matrix belong to \(\smc{O}\). It is also independent of the choice of basis up to elements of \((\smc{O}^{\ast})^{2}\). For if \(\b_{1}',\ldots,\b_{n}'\) is another basis for \(\mc{O}/\smc{O}\), we have
    \[
      \b_{i}' = \sum_{j}\a_{i,j}\b_{j},
    \]
    with \(\a_{i,j} \in \smc{O}\). Then \((\a_{i,j})\) is the base change matrix from \(\b_{1},\ldots,\b_{n}\) to \(\b_{1}',\ldots,\b_{n}'\) and so has nonzero determinant. Thus \(\det((\a_{i,j})) \in \smc{O}^{\ast}\). Moreover,
    \[
      \Trace_{\mc{O}/\smc{O}}(\b_{1}',\ldots,\b_{n}') = (\a_{i,j})\Trace_{\mc{O}/\smc{O}}(\b_{1},\ldots,\b_{n})(\a_{i,j})^{t},
    \]
    which, upon taking the determinant, shows
    \[
      d_{\mc{O}/\smc{O}}(\b_{1}',\ldots,\b_{n}') = \det((\a_{i,j}))^{2}d_{\mc{O}/\smc{O}}(\b_{1},\ldots,\b_{n}).
    \]
    Accordingly, we define the \textbf{discriminant}\index{discriminant} \(d_{\smc{O}}(\mc{O})\) of \(\mc{O}/\smc{O}\) to be the coset in \(\smc{O}/(\smc{O}^{\ast})^{2}\) represented by any discriminant \(d_{\mc{O}/\smc{O}}(\b_{1},\ldots,\b_{n})\). In other words,
    \[
      d_{\smc{O}}(\mc{O}) = d_{\mc{O}/\smc{O}}(\b_{1},\ldots,\b_{n})(\smc{O}^{\ast})^{2}.
    \]
    In particular, \(d_{\smc{O}}(\mc{O}) = 0\) is independent of the choice of representative. The discriminant is also multiplicative with respect to direct sums.

    \begin{proposition}\label{prop:discriminant_and_direct_sums}
      Let \(\mc{O}/\smc{O}\) be a ring extension such that \(\mc{O}\) is a free \(\smc{O}\)-module of rank \(n\). Suppose we have a direct sum decomposition
      \[
        \mc{O} = \mc{O}_{1} \op \mc{O}_{2},
      \]
      for free \(\smc{O}\)-modules \(\mc{O}_{1}\) and \(\mc{O}_{2}\) of ranks \(n_{1}\) and \(n_{2}\) with bases \(\b_{1,1},\ldots,\b_{n_{1},1}\) and \(\b_{1,2},\ldots,\b_{n_{2},2}\) over \(\smc{O}\) respectively. Then \(\b_{1,1},\ldots,\b_{n_{1},1},\b_{1,2},\ldots,\b_{n_{2},2}\) is a basis of \(\mc{O}\) over \(\smc{O}\) and
      \[
        d_{\mc{O}/\smc{O}}(\b_{1,1},\ldots,\b_{n_{1},1},\b_{1,2},\ldots,\b_{n_{2},2}) = d_{\mc{O}_{1}/\smc{O}}(\b_{1,1},\ldots,\b_{n_{1},1})d_{\mc{O}_{2}/\smc{O}}(\b_{1,2},\ldots,\b_{n_{2},2}).
      \]
    \end{proposition}
    \begin{proof}
      Clearly \(\b_{1,1},\ldots,\b_{n_{1},1},\b_{1,2},\ldots,\b_{n_{2},2}\) is a basis for \(\mc{O}/\smc{O}\) and \(\b_{i,1}\b_{j,2} = 0\). It follows that \(d_{\mc{O}/\smc{O}}(\b_{1,1},\ldots,\b_{n_{1},1},\b_{1,2},\ldots,\b_{n_{2},2})\) is the determinant of the block diagonal matrix
      \[
        \begin{pmatrix} \Trace_{\mc{O}/\smc{O}}(\b_{1,1},\ldots,\b_{n_{1},1}) & \\ & \Trace_{\mc{O}/\smc{O}}(\b_{1,2},\ldots,\b_{n_{2},2}) \end{pmatrix}.
      \]
      Recall that
      \[
        \Trace_{\mc{O}/\smc{O}}(\b_{1}) = \Trace_{\mc{O}_{1}/\smc{O}}(\b_{1}) \quad \text{and} \quad \Trace_{\mc{O}/\smc{O}}(\b_{2}) = \Trace_{\mc{O}_{2}/\smc{O}}(\b_{2}),
      \]
      for any \(\b_{1} \in \mc{O}_{1}\) and \(\b_{2} \in \mc{O}_{2}\). Whence the block diagonal matrix above is
      \[
        \begin{pmatrix} \Trace_{\mc{O}_{1}/\smc{O}}(\b_{1,1},\ldots,\b_{n_{1},1}) & \\ & \Trace_{\mc{O}_{2}/\smc{O}}(\b_{1,2},\ldots,\b_{n_{2},2}) \end{pmatrix}.
      \]
      The determinant of this matrix is \(d_{\mc{O}_{1}/\smc{O}}(\b_{1,1},\ldots,\b_{n_{1},1})d_{\mc{O}_{2}/\smc{O}}(\b_{1,2},\ldots,\b_{n_{2},2})\) which completes the proof.
    \end{proof}
    
    We now specialize to the setting of a degree \(n\) separable extension \(L/K\). In this case, it turns out that the discriminant of a basis is nonzero. To see this will require some work. First, we define the \textbf{trace form}\index{trace form} of \(L/K\) to be the map
    \[
      \Trace_{L/K}:L \x L \to K \qquad (\l,\eta) \mapsto \Trace_{L/K}(\l\eta).
    \]
    In other words, this is just the trace of \(L/K\) considered as a pairing. Clearly this is a symmetric bilinear form. In fact, the trace form is also nondegenerate as \cref{prop:formulas_for_trace_and_norm} implies
    \[
      \Trace_{L/K}(1) = n,
    \]
    and we can pair any \(\l \in L\) with its inverse. Using the trace form, we can show that the discriminant of any basis for \(L/K\) is nonzero.

    \begin{proposition}\label{prop:discriminant_not_zero}
      Let \(L/K\) be a degree \(n\) separable extension and let \(\l_{1},\ldots,\l_{n}\) be a basis for \(L/K\). Then \(d_{L/K}(\l_{1},\ldots,\l_{n}) \neq 0\).
    \end{proposition}
    \begin{proof}
      Suppose by contradiction that \(d_{L/K}(\l_{1},\ldots,\l_{n}) = 0\). Then the trace matrix \(\Trace_{L/K}(\l_{1},\ldots,\l_{n})\) is not invertible. Therefore
      \[
        \Trace_{L/K}(\l_{1},\ldots,\l_{n})\begin{pmatrix} \k_{1} \\ \vdots \\ \k_{n} \end{pmatrix} = \mathbf{0},
      \]
      for some \(\k_{i} \in K\) not all of which are zero. Then the element
      \[
        \l = \sum_{j}\k_{j}\l_{j}.
      \]
      is nonzero. Moreover, the previous identity is equivalent to the fact that \(\Trace_{L/K}(\l\l_{i}) = 0\). As \(\l_{1},\ldots,\l_{n}\) is a basis for \(L/K\), it follows that the trace form is degenerate which is a contradiction. Hence \(d_{L/K}(\l_{1},\ldots,\l_{n}) \neq 0\).
    \end{proof}

    In addition to the discriminant \(d_{L/K}(\l_{1},\ldots,\l_{n})\) being nonzero, we can also write it in a more useful form. To do this, we define the \textbf{embedding matrix}\index{embedding matrix} \(M(\l_{1},\ldots,\l_{n})\) of the basis \(\l_{1},\ldots,\l_{n}\) by
    \[
      M(\l_{1},\ldots,\l_{n}) = \begin{pmatrix} \s_{1}(\l_{1}) & \cdots & \s_{1}(\l_{n}) \\ \vdots & & \vdots \\ \s_{n}(\l_{1}) & \cdots & \s_{n}(\l_{n}) \end{pmatrix},
    \]
    where \(\s_{1},\ldots,\s_{n}\) are the elements of \(\Hom_{K}(L,\conj{K})\). The discriminant turns out to be the square of the determinant of the embedding matrix.

    \begin{proposition}\label{disc_as_square_of_embedding_matrix}
      Let \(L/K\) be a degree \(n\) separable extension. Then for any basis \(\l_{1},\ldots,\l_{n}\) of \(L/K\), we have
      \[
        d_{L/K}(\l_{1},\ldots,\l_{n}) = \det(M(\l_{1},\ldots,\l_{n}))^{2}.
      \]
    \end{proposition}
    \begin{proof}
      Recall that the \(ij\)-entry of \(M(\l_{1},\ldots,\l_{n})^{t}M(\l_{1},\ldots,\l_{n})\) is the dot product of the \(i\)-th and \(j\)-th columns of \(M(\l_{1},\ldots,\l_{n})\). Then a straightforward computation shows
      \[
        \det(M(\l_{1},\ldots,\l_{n})^{t}M(\l_{1},\ldots,\l_{n})) = \det\left(\left(\sum_{\s \in \Hom_{K}(L,\conj{K})}\s(\l_{i}\l_{j})\right)_{i,j}\right).
      \]
      By \cref{prop:formulas_for_trace_and_norm}, this identity is equivalent to the claim.
    \end{proof}

    In general, it is difficult to compute the discriminant of a basis for \(L/K\). However, if the basis is of the form \(1,\l,\ldots,\l^{n-1}\) (take \(\l = \t\) for a primitive element \(\t\) of \(L/K\)) then the discriminant of this basis can be easily computed. Indeed, the embedding matrix becomes
    \[
      M(1,\l,\ldots,\l^{n-1}) = \begin{pmatrix} 1 & \s_{1}(\l) & \cdots & \s_{1}(\l)^{n-1} \\ \vdots & \vdots & & \vdots \\ 1 & \s_{n}(\l) & \cdots & \s_{n}(\l)^{n-1} \end{pmatrix},
    \]
    which is a Vandermonde matrix. By \cref{disc_as_square_of_embedding_matrix}, we have
    \begin{equation}\label{equ:Vandermonde_determinant_for_discriminant}
      d_{L/K}(1,\l,\ldots,\l^{n-1}) = \prod_{i \le j}(\s_{i}(\l)-\s_{j}(\l))^{2},
    \end{equation}
    which is the square of the Vandermonde determinant of \(M(1,\l,\ldots,\l^{n-1})\).

    Let us now consider the situation when \(\smc{O}\) is an integrally closed domain with field of fractions \(K\), \(L/K\) be a degree \(n\) separable extension, and \(\mc{O}\) is the integral closure of \(\smc{O}\) in \(L\). The discriminant relates inclusion of \(\mc{O}\) relative to \(\smc{O}\) when the underlying basis is contained in \(\mc{O}\).

    \begin{lemma}\label{lem:lemma_for_integral_basis_AKBL}
      Let \(\smc{O}\) be an integrally closed domain with field of fractions \(K\), \(L/K\) be a degree \(n\) separable extension, and \(\mc{O}\) be the integral closure of \(\smc{O}\) in \(L\). If \(\l_{1},\ldots,\l_{n}\) is a basis for \(L/K\) that is contained in \(\mc{O}\) then
      \[
        d_{L/K}(\l_{1},\ldots,\l_{n})\mc{O} \subseteq \smc{O}\l_{1}+\cdots+\smc{O}\l_{n}.
      \]
    \end{lemma}
    \begin{proof}
      Let \(\b \in \mc{O}\) and write
      \[
        \b =  \sum_{j}\k_{j}\l_{j}.
      \]
      for some \(\k_{j} \in K\). Linearity of the trace implies
      \[
        \Trace_{L/K}(\l_{i}\b) = \sum_{j}\k_{j}\Trace_{L/K}(\l_{i}\l_{j}),
      \]
      These \(n\) equations are equivalent to the identity
      \[
        \begin{pmatrix} \Trace_{L/K}(\b\l_{1}) \\ \vdots \\ \Trace_{L/K}(\b\l_{n}) \end{pmatrix} = \Trace_{L/K}(\l_{1},\ldots,\l_{n})\begin{pmatrix} \k_{1} \\ \vdots \\ \k_{n} \end{pmatrix}.
      \]
      By \cref{prop:trace_and_norm_AKBL}, the column vector on the left-hand side and the trace matrix have entries in \(\smc{O}\). Cramer's rule implies \(d_{L/K}(\l_{1},\ldots,\l_{n})\k_{i} \in \smc{O}\). As \(\b\) was arbitrary, this means
      \[
        d_{L/K}(\l_{1},\ldots,\l_{n})\mc{O} \subseteq \smc{O}\l_{1}+\cdots+\smc{O}\l_{n}.
      \]
    \end{proof}

    In this situation, we say that \(\b_{1},\ldots,\b_{n}\) is an \textbf{integral basis}\index{integral basis} for \(\mc{O}/\smc{O}\) if
    \[
      \mc{O} = \smc{O}\b_{1}+\cdots+\smc{O}\b_{n}.
    \]
    Equivalently, \(\mc{O}\) is a free \(\smc{O}\)-module of rank \(n\). An integral basis is necessarily a basis for \(L/K\) by \cref{prop:field_of_fractions_AKBL}. However, an integral basis need not always exist. For if \(\l_{1},\ldots,\l_{n}\) is a basis of \(L/K\), \cref{prop:field_of_fractions_AKBL} implies that we can multiply by a nonzero element of \(\smc{O}\) to ensure that this basis is contained in \(\mc{O}\). However, \(\l_{1},\ldots,\l_{n}\) need not also be a basis of \(\mc{O}\) over \(\smc{O}\). Nevertheless, if \(\smc{O}\) is a principal ideal domain then we can ensure the existence of an integral basis.
    
    \begin{theorem}\label{thm:integral_basis_AKBL}
      Let \(\smc{O}\) be a principal ideal domain with field of fractions \(K\), let \(L/K\) be a degree \(n\) separable extension, and let \(\mc{O}\) be the integral closure of \(\smc{O}\) in \(L\). Then \(\mc{O}/\smc{O}\) admits an integral basis. Moreover, every finitely generated nonzero \(\mc{O}\)-submodule of \(L\) is a free \(\smc{O}\)-module of rank \(n\).
    \end{theorem}
    \begin{proof}
      Let \(\l_{1},\ldots,\l_{n}\) be a basis for \(L/K\). From \cref{lemma:unique_factorization_domains_are_integrally_closed} we see that \(\smc{O}\) is an integrally closed domain. Using \cref{prop:field_of_fractions_AKBL}, we may multiply by a nonzero element of \(\smc{O}\), if necessary, to ensure that this basis belongs to \(\mc{O}\). Then \cref{lem:lemma_for_integral_basis_AKBL} implies
      \[
        d_{L/K}(\l_{1},\ldots,\l_{n})\mc{O} \subseteq \smc{O}\l_{1}+\cdots+\smc{O}\l_{n}.
      \]
      Since \(\smc{O}\l_{1}+\cdots+\smc{O}\l_{n}\) is a free \(\smc{O}\)-module of rank \(n\) and \(\smc{O}\) is a principal ideal domain, it follows from the structure theorem of finitely generated modules over principal ideal domains that \(\mc{O}\) is also a free \(\smc{O}\)-module of rank at most \(n\). But any basis for \(\mc{O}/\smc{O}\) must also be a basis for \(L/K\) by \cref{prop:field_of_fractions_AKBL}. Hence the rank is exactly \(n\) and \(\mc{O}\) admits an integral basis over \(\smc{O}\) which proves the first statement.
      
      Now suppose \(M\) is a nonzero \(\mc{O}\)-submodule of \(L\) and let \(\w_{1},\ldots,\w_{r}\) be generators. By \cref{prop:field_of_fractions_AKBL} again, we may multiply by a nonzero element of \(\smc{O}\), if necessary, to ensure that these generators belong to \(\mc{O}\). Then
      \[
        d_{L/K}(\w_{1},\ldots,\w_{n})M \subseteq d_{L/K}(\w_{1},\ldots,\w_{n})\mc{O}.
      \]
      By the structure theorem of finitely generated modules over principal ideal domains again, \(M\) is a free \(\smc{O}\)-module of rank at most \(n\). To see that the rank is at least \(n\), let \(\a \in M\) be nonzero and, as before, let \(\l_{1},\ldots,\l_{n}\) be a basis for \(L/K\) that is contained in \(\mc{O}\). Then \(\a\l_{1},\ldots,\a\l_{n}\) is a basis for \(L/K\) contained in \(M\). Thus the rank of \(M\) is at least \(n\), and in particular it must be \(n\). This proves the second statement.
    \end{proof}

    Recall that if \(L_{1}/K\) and \(L_{2}/K\) are finite separable extensions then the composite \(L\) of \(L_{1}\) and \(L_{2}\) is such that \(L/K\) is also a finite separable extension. Integral bases behave well with respect to composite fields provided the fields are linearly disjoint.

    \begin{proposition}\label{prop:linearly_disjoint_integral_basis}
      Let \(\smc{O}\) be an integrally closed domain with field of fractions \(K\), \(L_{1}/K\) and \(L_{2}/K\) be degree \(n_{1}\) and \(n_{2}\) separable extensions, and \(\mc{O}_{1}\) and \(\mc{O}_{2}\) be the integral closures of \(\smc{O}\) in \(L_{1}\) and \(L_{2}\) respectively. Suppose \(L_{1}\) and \(L_{2}\) are linearly disjoint over \(K\) in \(\conj{K}\) and that \(\mc{O}_{1}/\smc{O}\) and \(\mc{O}_{2}/\smc{O}\) admit integral bases \(\b_{1,1},\ldots,\b_{n_{1},1}\) and \(\b_{1,2},\ldots,\b_{n_{2},2}\) with
      \[
        \a_{1}d_{L_{1}/K}(\b_{1,1},\ldots,\b_{n_{1},1})+\a_{2}d_{L_{2}/K}(\b_{1,2},\ldots,\b_{n_{2},2}) = 1,
      \]
      for some \(\a_{1},\a_{2} \in \smc{O}\). Let \(L\) be the composite of \(L_{1}\) and \(L_{2}\) and let \(\mc{O}\) be the integral closure of \(\smc{O}\) in \(L\). Then \(\b_{1,1}\b_{1,2},\ldots,\b_{n_{1},1}\b_{n_{2},2}\) is an integral basis for \(\mc{O}/\smc{O}\) and
      \[
        \mc{O} =\mc{O}_{1}\mc{O}_{2}.
      \]
      In particular,
      \[
        d_{L/K}(\b_{1,1}\b_{1,2},\ldots,\b_{n_{1},1}\b_{n_{2},2}) = d_{L_{1}/K}(\b_{1,1},\ldots,\b_{n_{1},1})^{n_{2}}d_{L_{2}/K}(\b_{1,2},\ldots,\b_{n_{2},2})^{n_{1}}.
      \]
    \end{proposition}
    \begin{proof}
      We will start by proving \(\b_{1,1}\b_{1,2},\ldots,\b_{n_{1},1}\b_{n_{2},2}\) is an integral basis for \(\mc{O}/\smc{O}\). Since \(L_{1}\) and \(L_{2}\) are linearly disjoint over \(K\) in \(\conj{K}\), it must be that \(\b_{1,1}\b_{1,2},\ldots,\b_{n_{1},1}\b_{n_{2},2}\) is a basis for \(L/K\). Therefore any \(\b \in \mc{O}\) is of the form
      \[
        \b = \sum_{i,j}\k_{i,j}\b_{i,1}\b_{j,2},
      \]
      for some \(\k_{i,j} \in K\). We need to show that \(\k_{i,j} \in \smc{O}\). To this end, let
      \[
        \a_{i,2} = \sum_{j}\k_{i,j}\b_{j,2} \quad \text{and} \quad \a_{j,1} = \sum_{i}\k_{i,j}\b_{i,1},
      \]
      so that
      \[
        \b = \sum_{i}\a_{i,2}\b_{i,1} \quad \text{and} \quad \b = \sum_{j}\a_{j,1}\b_{j,2}.
      \]
      In particular, \(\a_{i,2} \in L_{2}\) and \(\a_{j,1} \in L_{1}\). By linear disjointness, we have
      \[
        \Hom_{K}(L,\conj{K}) \cong \Hom_{K}(L_{1},\conj{K}) \x \Hom_{K}(L_{2},\conj{K}).
      \]
      So letting \(\s_{1,1},\ldots,\s_{n_{1},1}\) and \(\s_{1,2},\ldots,\s_{n_{2},2}\) be the elements of \(\Hom_{K}(L_{1},\conj{K})\) and \(\Hom_{K}(L_{2},\conj{K})\) respectively, \(\s_{1,1}\s_{1,2},\ldots,\s_{n_{1},1}\s_{n_{2},2}\) are the elements of \(\Hom_{K}(L,\conj{K})\). In particular, we may view \(\s_{1,1},\ldots,\s_{n_{1},1}\) and \(\s_{1,2},\ldots,\s_{n_{2},2}\) as elements of \(\Hom_{K}(L,\conj{K})\) that act as the identity on \(L_{2}\) and \(L_{1}\) respectively. Then the \(n_{1}\) and \(n_{2}\) equations
      \[
        \sum_{i}\s_{i,1}(\b_{i,1})\a_{i,2} = \s_{i,1}(\b) \quad \text{and} \quad \sum_{j}\s_{j,2}(\b_{j,2})\a_{j,1} = \s_{j,2}(\b),
      \]
      are equivalent to the identities
      \[
        M(\b_{1,1},\ldots,\b_{n_{1},1})\begin{pmatrix} \a_{1,2} \\ \vdots \\ \a_{n_{1},2} \end{pmatrix} = \begin{pmatrix} \s_{1,1}(\b) \\ \vdots \\ \s_{n_{1},1}(\b) \end{pmatrix},
      \]
      and
      \[
        M(\b_{1,2},\ldots,\b_{n_{2},2})\begin{pmatrix} \a_{1,1} \\ \vdots \\ \a_{n_{2},1} \end{pmatrix} = \begin{pmatrix} \s_{1,2}(\b) \\ \vdots \\ \s_{n_{2},2}(\b) \end{pmatrix},
      \]
      respectively. The embedding matrices and column vectors on the right-hand sides all have entries in \(\mc{O}\). Cramer's rule together with \cref{disc_as_square_of_embedding_matrix} imply that \(d_{L_{1}/K}(\b_{1,1},\ldots,\b_{n_{1},1})\a_{i,2} \in \mc{O}\) and \(d_{L_{2}/K}(\b_{1,2},\ldots,\b_{n_{2},2})\a_{j,1} \in \mc{O}\). As \(\mc{O}_{1} = L_{1} \cap \mc{O}\) and \(\mc{O}_{2} = L_{2} \cap \mc{O}\), we have \(\a_{i,2} \in \mc{O}_{2}\) and \(\a_{j,1} \in \mc{O}_{1}\). Expanding \(\a_{i,2}\) and \(\a_{j,1}\) show that \(d_{L_{1}/K}(\b_{1,1},\ldots,\b_{n_{1},1})\k_{i,j} \in \smc{O}\) and \(d_{L_{2}/K}(\b_{1,2},\ldots,\b_{n_{2},2})\k_{i,j} \in \smc{O}\). From the identity
      \[
        k_{i,j} = k_{i,j}(\a_{1}d_{L_{1}/K}(\b_{1,1},\ldots,\b_{n_{1},1})+\a_{2}d_{L_{2}/K}(\b_{1,2},\ldots,\b_{n_{2},2})),
      \]
      we conclude \(k_{i,j} \in \smc{O}\). This proves \(\b_{1,1}\b_{1,2},\ldots,\b_{n_{1},1}\b_{n_{2},2}\) is an integral basis for \(\mc{O}/\smc{O}\). The fact that
      \[
        \mc{O} =\mc{O}_{1}\mc{O}_{2},
      \]
      follows at once. It remains to prove the discriminant formula which will be achieved by decomposing the embedding matrix into blocks. Recall that \(\s_{1,1}\s_{1,2},\ldots,\s_{n_{1},1}\s_{n_{2},2}\) are the elements of \(\Hom_{K}(L,\conj{K})\), we have
      \[
        M(\b_{1,1}\b_{1,2},\ldots,\b_{n_{1},1}\b_{n_{2},2}) = \begin{pmatrix} \s_{1,1}(\b_{1,1})\s_{1,2}(\b_{1,2}) & \cdots & \s_{1,1}(\b_{n_{1},1})\s_{1,2}(\b_{n_{2},2}) \\ \vdots & & \vdots \\ \s_{n_{1},1}(\b_{1,1})\s_{n_{2},2}(\b_{1,2}) & \cdots & \s_{n_{1},1}(\b_{n_{1},1})\s_{n_{2},2}(\b_{n_{2},2}) \end{pmatrix}.
      \]
      This left-hand side admits the block factorization
      \[
        \begin{pmatrix} M(\b_{1,1},\ldots,\b_{n_{1},1}) & & \\ & \ddots & \\ & & M(\b_{1,1},\ldots,\b_{n_{1},1}) \end{pmatrix}\begin{pmatrix} I\s_{1,2}(\b_{1,2}) & \cdots & I\s_{1,2}(\b_{n_{2},2}) \\ \vdots & & \vdots \\ I\s_{n_{2},2}(\b_{1,2}) & \cdots & I\s_{n_{2},2}(\b_{n_{2},2}) \end{pmatrix},
      \]
      where the second matrix is the Kronecker product \(M(\b_{1,2},\ldots,\b_{n_{2},2}) \ox I\). As this Kronecker product can be expressed as \((I \ox M(\b_{1,2},\ldots,\b_{n_{2},2}))P\) for some permutation matrix \(P\), it takes the form
      \[
        \begin{pmatrix} M(\b_{1,2},\ldots,\b_{n_{2},2}) & & \\ & \ddots & \\ & & M(\b_{1,2},\ldots,\b_{n_{2},2}) \end{pmatrix}P,
      \]
      with \(\det(P) = \pm 1\). Putting these decompositions together and applying \cref{disc_as_square_of_embedding_matrix} shows
      \[
        d_{L/K}(\b_{1,1}\b_{1,2},\ldots,\b_{n_{1},1}\b_{n_{2},2}) = d_{L_{1}/K}(\b_{1,1},\ldots,\b_{n_{1},1})^{n_{2}}d_{L_{2}/K}(\b_{1,2},\ldots,\b_{n_{2},2})^{n_{1}}.
      \]
      This proves the discriminant formula.
    \end{proof}

    It is generally a difficult problem to write down an integral basis explicitly. However, there is one instance in which this is possible. We say that \(\mc{O}/\smc{O}\) is \textbf{monogenic}\index{monogenic} if \(\mc{O} =\smc{O}[\b]\) for some \(\b \in \mc{O}\). It follows immediately that \(1,\b,\ldots,\b^{n-1}\) is an integral basis for \(\mc{O}/\smc{O}\). The discriminant of this basis is then given by \cref{equ:Vandermonde_determinant_for_discriminant}.
    
    Regardless of if \(\mc{O}/\smc{O}\) is monogenic or not, if an integral basis exists then the traces \(\Trace_{L/K}\) and \(\Trace_{\mc{O}/\smc{O}}\) and norms \(\Norm_{L/K}\) and \(\Norm_{\mc{O}/\smc{O}}\) agree.

    \begin{proposition}\label{prop:trace_and_norm_reduce_for_integral_basis}
      Let \(\smc{O}\) be an integrally closed domain with field of fractions \(K\), \(L/K\) be a degree \(n\) separable extension, and \(\mc{O}\) be the integral closure of \(\smc{O}\) in \(L\). If \(\mc{O}\) admits an integral basis over \(\smc{O}\) then
      \[
        \Trace_{L/K}(\b) = \Trace_{\mc{O}/\smc{O}}(\b) \quad \text{and} \quad \Norm_{L/K}(\b) = \Norm_{\mc{O}/\smc{O}}(\b),
      \]
      for all \(\b \in \mc{O}\).
    \end{proposition}
    \begin{proof}
      Let \(\b_{1},\ldots,\b_{n}\) be an integral basis for \(\mc{O}/\smc{O}\). Then \(\b_{1},\ldots,\b_{n}\) is also a basis for \(L/K\). It follows that the multiplication by \(\b\) map in \(L\) has the same matrix representation as it does in \(\mc{O}\). Whence
      \[
        \Trace_{L/K}(\b) = \Trace_{\mc{O}/\smc{O}}(\b) \quad \text{and} \quad \Norm_{L/K}(\b) = \Norm_{\mc{O}/\smc{O}}(\b).
      \]
    \end{proof}

    We now turn to the case of a number field \(K\) of degree \(n\). We write \(\Trace_{K} = \Trace_{K/\Q}\) and \(\Norm_{K} = \Norm_{K/\Q}\) and call these the \textbf{trace}\index{trace} and \textbf{norm}\index{norm} of \(K\). Moreover, for any \(\k \in K\) we call \(\Trace_{K}(\k)\) and \(\Norm_{K}(\k)\) the \textbf{trace}\index{trace} and \textbf{norm}\index{norm} of \(\k\). Observe from \cref{prop:trace_and_norm_AKBL,prop:unit_if_and_only_if_AKBL} that the trace and norm of algebraic integers are themselves integers and \(\k \in \mc{O}_{K}\) is a unit if and only if \(\Norm_{K}(\k) = \pm 1\). We also call the nondegenerate symmetric bilinear form
    \[
      \Trace_{K/\Q}:K \x K \to \Q \qquad (\k,\l) \mapsto \Trace_{K/\Q}(\k\l),
    \]
    the \textbf{trace form}\index{trace form} of \(K\). Moreover, since \(\Z\) is a principal ideal domain \cref{thm:integral_basis_AKBL} implies that \(\mc{O}_{K}/\Z\) admits an integral basis. Whence \(\mc{O}_{K}\) is a free abelian group of rank \(n\). Accordingly, we say that \(\a_{1},\ldots,\a_{n}\) is an \textbf{integral basis}\index{integral basis} for \(K\) if it is an integral basis for \(\mc{O}_{K}/\Z\). Accordingly, we define the \textbf{discriminant}\index{discriminant} \(\D_{K}\) of \(K\) to be the discriminant of \(\mc{O}_{K}/\Z\). As \((\Z^{\ast})^{2} = \{1\}\), \(\D_{K}\) is an integer and satisfies
    \[
      \D_{K} = d_{L/K}(\a_{1},\ldots,\a_{n}),
    \]
    for any integral basis \(\a_{1},\ldots,\a_{n}\) for \(K\). Moreover, \(\D_{K}\) is nonzero since the trace form is nondegenerate and may very well be negative. In light of \cref{disc_as_square_of_embedding_matrix}, we also have
    \[
      |\det(M(\a_{1},\ldots,\a_{n}))| = \sqrt{|\D_{K}|}.
    \]
    Lastly, we say \(K\) is \textbf{monogenic}\index{monogenic} if \(\mc{O}_{K}\) is monogenic over \(\Z\).
  \section{Dedekind Domains}
    Let \(\smc{O}\) be an integral domain with field of fractions \(K\). Any nonzero ideal \(\mf{a}\) of \(\smc{O}\) is said to be an \textbf{integral ideal}\index{integral ideal} of \(\smc{O}\). We call any prime integral ideal \(\mf{p}\) of \(\smc{O}\) a \textbf{prime}\index{prime} of \(\smc{O}\). If \(\mf{p}\) is principal with \(\mf{p} = \a\smc{O}\) for some nonzero \(\a \in K\) we will also refer to \(\a\) as the prime instead of \(\mf{p}\). In any case, an integral ideal is just a \(\smc{O}\)-submodule of \(\smc{O}\). More generally, we say \(\mf{f}\) is a \textbf{fractional ideal}\index{fractional ideal} of \(\smc{O}\) if it is \(\smc{O}\)-submodule of \(K\) such that there is a nonzero \(\d \in \smc{O}\) with \(\d\mf{f} \in \smc{O}\). This simply means \(\d\mf{f}\) is an integral ideal. Conversely, if \(\mf{a}\) is an integral ideal and \(\d \in \smc{O}\) is nonzero then  
    \[
      \mf{f} = \frac{1}{\d}\mf{a},
    \]
    is a fractional ideal. So every fractional ideal is of this form. The fractional ideal \(\mf{f}\) is said to be \textbf{principal}\index{principal} if it is generated by a single element. That is, if \(\mf{f} = \k\smc{O}\) for some nonzero \(\k \in K\). In particular, all integral ideals are fractional ideals by taking \(\d = 1\) and all principal integral ideals are principal fractional ideals.

    We will need to consider a more restrictive setting in order to develop a useful theory of integral and fractional ideals. This will allow for integral ideals to factor uniquely into a product of primes and for the fractional ideals to form a group in which \(\smc{O}\) will act as the identity. This will essentially permit us to treat integral ideas as we would integers. Accordingly, a ring \(\smc{O}\) is said to be a \textbf{Dedekind domain}\index{Dedekind domain} if the following properties are satisfied:
    \begin{enumerate}[label*=(\roman*)]
      \item \(\smc{O}\) is an integrally closed domain.
      \item \(\smc{O}\) is noetherian.
      \item Every prime of \(\smc{O}\) is maximal.
    \end{enumerate}
    Observe that \(\smc{O}\) being noetherian forces every integral ideal \(\mf{a}\) to be a finitely generated \(\smc{O}\)-module. In fact, as every fractional ideal \(\mf{f}\) is of the form \(\mf{f} = \frac{1}{\d}\mf{a}\), for some nonzero \(\d \in \smc{O}\) and integral ideal \(\mf{a}\), \(\mf{f}\) is a finitely generated \(\smc{O}\)-submodule of \(K\).

    \begin{remark}
      As \(\Z\) is a principal ideal domain which is an integrally closed domain by \cref{rem:integers_are_integrally_closed}, \(\Z\) is a Dedekind domain. The primes of \(\Z\) are exactly the primes \(p\).
    \end{remark}
    
    In view of principal ideal domains being integrally closed domains by \cref{lemma:unique_factorization_domains_are_integrally_closed}, they are Dedekind domains. In fact, Dedekind domains should be thought of as generalizations of principal ideal domains. As \(\Z\) is a prototypical example of a principal ideal domain, a Dedekind domain \(\smc{O}\) should be thought of as a generalization of \(\Z\). In fact, we will show that being a principal ideal domain is equivalent to being a unique factorization domain for \(\smc{O}\). So if \(\smc{O}\) is not a principal ideal domain then unique factorization fails. However, it will be possible to remedy most of this as one of our primary aims is to prove that integral ideals factor uniquely into a product of primes. This clearly generalizes unique factorization in \(\Z\) which will be enough. To this end, we first show inclusion in one direction for integral ideals.

    \begin{lemma}\label{lem:integral_ideal_prime_inclusion}
      Let \(\smc{O}\) be a Dedekind domain. For every integral ideal \(\mf{a}\) of \(\smc{O}\), there exist primes \(\mf{p}_{1},\ldots,\mf{p}_{k}\) of \(\smc{O}\) such that
      \[
        \mf{p}_{1}\cdots\mf{p}_{k} \subseteq \mf{a}.
      \]
    \end{lemma}
    \begin{proof}
      Let \(\mc{S}\) be the set of integral ideals which do not contain a product of primes. It suffices to show \(\mc{S}\) is empty. Assuming otherwise and ordering \(\mc{S}\) by inclusion, there exists a maximal element \(\mf{a} \in \mc{S}\) as \(\smc{O}\) is noetherian. Moreover \(\mf{a}\) cannot be prime. Then there exist \(\a_{1},\a_{2} \in \smc{O}\) with \(\a_{1}\a_{2} \in \mf{a}\) and such that \(\a_{1},\a_{2} \notin \mf{a}\). Define integral ideals
      \[
        \mf{b}_{1} = \mf{a}+\a_{1}\smc{O} \quad \text{and} \quad \mf{b}_{2} = \mf{a}+\a_{2}\smc{O}.
      \]
      Note that \(\mf{b}_{1}\) and \(\mf{b}_{2}\) strictly contain \(\mf{a}\). Moreover, \(\mf{b}_{1}\mf{b}_{2} \subseteq \mf{a}\) because
      \[
        \mf{b}_{1}\mf{b}_{2} = \mf{a}^{2}+\a_{1}\mf{a}+\a_{2}\mf{a}+\a_{1}\a_{2}\smc{O}.
      \]
      Maximality of \(\mf{a}\) implies that there exist primes \(\mf{p}_{1},\ldots,\mf{p}_{k}\) and \(\mf{q}_{1},\ldots,\mf{q}_{\ell}\) such that
      \[
        \mf{p}_{1}\cdots\mf{p}_{k} \subseteq \mf{b}_{1} \quad \text{and} \quad \mf{q}_{1}\cdots\mf{q}_{\ell} \subseteq \mf{b}_{2}.
      \]
      But then
      \[
        \mf{p}_{1}\cdots\mf{p}_{k}\mf{q}_{1}\cdots\mf{q}_{\ell} \subseteq \mf{a},
      \]
      which contradicts the fact that \(\mf{a} \in \mc{S}\). Hence \(\mc{S}\) is empty.
    \end{proof}

    More work will be necessary obtain reverse inclusion. To this end, we being by constructing a fractional ideal associated to every prime which will turn out to be the inverse. Let \(\mf{p}\) be a prime. We define \(\mf{p}^{-1}\) by
    \[
      \mf{p}^{-1} = \{\k \in K:\k\mf{p} \subseteq \smc{O}\}.
    \]
    Then \(\mf{p}^{-1}\) is a fractional ideal. Indeed, since \(\mf{p}\) is an integral ideal there exists a nonzero \(\a \in \mf{p}\). The definition of \(\mf{p}^{-1}\) implies \(\a\mf{p}^{-1} \subseteq \smc{O}\). Hence \(\a\mf{p}^{-1}\) is an integral ideal and therefore \(\mf{p}^{-1}\) is a fractional ideal. Unlike integral ideals, \(1 \in \mf{p}^{-1}\) so that \(\mf{p}^{-1}\) contains units.

    \begin{lemma}\label{lem:inverse_for_prime_ideals}
      Let \(\smc{O}\) be a Dedekind domain and \(\mf{p}\) be a prime of \(\smc{O}\). Then
      \[
        \smc{O} \subset \mf{p}^{-1} \quad \text{and} \quad \mf{p}^{-1}\mf{p} = \smc{O}.
      \]
    \end{lemma}
    \begin{proof}
      We will first prove the inclusion. As \(\smc{O} \subseteq \mf{p}^{-1}\), it suffices to show \(\mf{p}^{-1}-\smc{O}\) is nonempty. To this end, let \(\a \in \mf{p}\) be nonzero. By \cref{lem:integral_ideal_prime_inclusion} let \(k\) be the smallest positive integer such that there exist primes \(\mf{p}_{1},\ldots,\mf{p}_{k}\) with
      \[
        \mf{p}_{1} \cdots \mf{p}_{k} \subseteq \a\smc{O}.
      \]
      As \(\a\smc{O} \subseteq \mf{p}\) and \(\mf{p}\) is prime, there must be some \(\mf{p}_{i}\) such that \(\mf{p}_{i} \subseteq \mf{p}\). Without loss of generality, we may assume \(\mf{p}_{1} \subseteq \mf{p}\). As primes are maximal in \(\smc{O}\), we conclude \(\mf{p}_{1} = \mf{p}\). Minimality of \(k\) forces
      \[
        \mf{p}_{2} \cdots \mf{p}_{k} \not\subseteq \a\smc{O}.
      \]
      Hence there exists \(\b \in \mf{p}_{2} \cdots \mf{p}_{k}\) with \(\b \notin \a\smc{O}\). We claim \(\b\a^{-1} \in \mf{p}^{-1}-\smc{O}\). Since \(\mf{p}_{1} = \mf{p}\), what we have previously shown implies \(\b\mf{p} \subseteq \a\smc{O}\) whence \(\b\a^{-1}\mf{p} \in \smc{O}\) which means \(\b\a^{-1} \in \mf{p}^{-1}\). But as \(\b \notin \a\smc{O}\), we also have \(\b\a^{-1} \notin \smc{O}\). Hence \(\b\a^{-1} \in \mf{p}^{-1}-\smc{O}\) as desired.
        
      Now let us prove the identity. Observe that
      \[
        \mf{p} \subseteq \mf{p}^{-1}\mf{p} \subseteq \smc{O}.
      \]
      Since \(\mf{p}\) is maximal, it follows that \(\mf{p}^{-1}\mf{p}\) is either \(\mf{p}\) or \(\smc{O}\). So it suffices to show that the first case cannot hold. Assume by contradiction that \(\mf{p}^{-1}\mf{p} = \mf{p}\). Let \(\w_{1},\ldots,\w_{r}\) be generators of \(\mf{p}\) and let \(\a \in \mf{p}^{-1}-\smc{O}\). Then \(\a\w_{i} \in \mf{p}^{-1}\mf{p}\) whence \(\a\mf{p} \subseteq \mf{p}^{-1}\mf{p}\) and \(\a\mf{p} \subseteq \mf{p}\). But then
      \[
        \a\w_{i} = \sum_{j}\a_{i,j}\w_{j},
      \]
      with \(\a_{i,j} \in \smc{O}\). These \(r\) equations are equivalent to the identity
      \[
        \begin{pmatrix} \a-\a_{1,1} & \a_{1,2} & \cdots & -\a_{1,r} \\ -\a_{2,1} & \a-\a_{2,2} & & \\ \vdots & & \ddots & \\ -\a_{r,1} & & & \a-\a_{r,r} \end{pmatrix}\begin{pmatrix} \w_{1} \\ \w_{2} \\ \vdots \\ \w_{r} \end{pmatrix} = \mathbf{0}.
      \]
      Thus the determinant of the matrix on the left-hand side must be zero. But this means \(\a\) is a root of the characteristic polynomial \(\det(xI-(\a_{i,j}))\) which is a monic polynomial with coefficients \(\smc{O}\). As \(\smc{O}\) is an integrally closed domain, \(\a \in \smc{O}\) which is a contraction. Thus \(\mf{p}^{-1}\mf{p} = \smc{O}\).
    \end{proof}

    We can now show that every integral ideal factors uniquely into a product of primes.

    \begin{theorem}\label{thm:unique_product_prime_ideals}
      Let \(\smc{O}\) be a Dedekind domain. Then for every integral ideal \(\mf{a}\) of \(\smc{O}\) there exist primes \(\mf{p}_{1},\ldots,\mf{p}_{k}\) of \(\smc{O}\) such that \(\mf{a}\) factors as
      \[
        \mf{a} = \mf{p}_{1} \cdots \mf{p}_{k}.
      \]
      Moreover, this factorization is unique up to reordering of the factors.
    \end{theorem}
    \begin{proof}
      We first prove existence and then uniqueness. For existence, let \(\mc{S}\) be the set of integral ideals that are not a product of primes. We will show that \(\mc{S}\) is empty. Assuming otherwise and ordering \(\mc{S}\) by inclusion, there exists a maximal element \(\mf{a} \in \mc{S}\) as \(\smc{O}\) is noetherian. Necessarily \(\mf{a}\) is not prime. Since primes are maximal in \(\smc{O}\), there is some prime \(\mf{p}_{1}\) for which \(\mf{a} \subset \mf{p}_{1}\). Then \cref{lem:inverse_for_prime_ideals} implies
      \[
        \mf{a} \subset \mf{a}\mf{p}_{1}^{-1} \subset \smc{O}.
      \]
      In particular, \(\mf{a}\mf{p}_{1}^{-1}\) is an integral ideal. By maximality of \(\mf{a}\), \(\mf{a}\mf{p}_{1}^{-1}\) factors into a product of primes. That is, there exist primes \(\mf{p}_{2},\ldots,\mf{p}_{k}\) such that
      \[
        \mf{a}\mf{p}_{1}^{-1} = \mf{p}_{2},\ldots,\mf{p}_{k}.
      \]
      Hence
      \[
        \mf{a} = \mf{p}_{1},\ldots,\mf{p}_{k},
      \]
      which is a contradiction. Therefore \(\mc{S}\) is empty thus proving the existence of such a factorization.
      
      Now we prove uniqueness. Suppose \(\mf{a}\) admits factorizations
      \[
        \mf{a} = \mf{p}_{1},\ldots,\mf{p}_{k} \quad \text{and} \quad \mf{a} = \mf{q}_{1},\ldots,\mf{q}_{\ell},
      \]
      for primes \(\mf{p}_{i}\) and \(\mf{q}_{j}\). Since \(\mf{p}_{1}\) is prime, there is some \(j\) for which \(\mf{q}_{j} \subseteq \mf{p}_{1}\). Without loss of generality, we may assume \(\mf{q}_{1} \subseteq \mf{p}_{1}\) and since primes are maximal in \(\smc{O}\) we have \(\mf{q}_{1} = \mf{p}_{1}\). Then
      \[
        \mf{p}_{2},\ldots,\mf{p}_{k} = \mf{q}_{2},\ldots,\mf{q}_{\ell}.
      \]
      Repeating this process, we see that the factorizations are the same. This proves uniqueness of the factorization.
    \end{proof}

    As a near immediate corollary, fractional ideal admits analogous factorizations.

    \begin{corollary}\label{cor:fractional_ideal_prime_factorization}
      Let \(\smc{O}\) be a Dedekind domain. Then for every fractional ideal \(\mf{f}\) of \(\smc{O}\) there exist primes \(\mf{p}_{1},\ldots,\mf{p}_{k}\) and \(\mf{q}_{1},\ldots,\mf{q}_{\ell}\) of \(\smc{O}\) such that \(\mf{f}\) factors as
      \[
        \mf{f} = \mf{p}_{1} \cdots \mf{p}_{k}\mf{q}_{1}^{-1},\ldots,\mf{q}_{\ell}^{-1}.
      \]
      Moreover, this factorization is unique up to reordering of the factors.
    \end{corollary}
    \begin{proof}
      Write \(\mf{f} = \frac{1}{\d}\mf{a}\) for some nonzero \(\d \in \smc{O}\) and integral ideal \(\mf{a}\). By \cref{thm:unique_product_prime_ideals}, \(\mf{a}\) and \(\d\smc{O}\) admit unique factorizations
      \[
        \mf{a} = \mf{p}_{1} \cdots \mf{p}_{k} \quad \text{and} \quad \d\smc{O} = \mf{q}_{1},\ldots,\mf{q}_{\ell},
      \]
      for some primes \(\mf{p}_{1},\ldots,\mf{p}_{k}\) and \(\mf{q}_{1},\ldots,\mf{q}_{\ell}\) up to reordering of the factors. Hence
      \[
        \mf{f} = \mf{p}_{1} \cdots \mf{p}_{k}\mf{q}_{1}^{-1},\ldots,\mf{q}_{\ell}^{-1}.
      \]
    \end{proof}

    So for any fractional ideal \(\mf{f}\) there exist distinct prime \(\mf{p}_{1},\ldots,\mf{p}_{r}\) such that \(\mf{f}\) admits a factorization
    \[
      \mf{f} = \mf{p}_{1}^{e_{1}} \cdots \mf{p}_{r}^{e_{r}},
    \]
    for some nonzero integers \(e_{i}\). This is called the \textbf{prime factorization}\index{prime factorization} of \(\mf{f}\) with \textbf{prime factors}\index{prime factors} \(\mf{p}_{1},\ldots,\mf{p}_{r}\). In particular, the prime factorization of an integral ideal \(\mf{a}\) is of the form
    \[
      \mf{a} = \mf{p}_{1}^{e_{1}} \cdots \mf{p}_{r}^{e_{r}},
    \]
    for some positive integers \(e_{i}\). Accordingly, for any two integral ideal \(\mf{a}\) and \(\mf{b}\) we say that \(\mf{a}\) \textbf{divides}\index{divides} \(\mf{b}\) and write \(\mf{a} \mid \mf{b}\) if \(\mf{b} \subseteq \mf{a}\). Sometimes this condition is expressed as \textit{to divide is to contain}. By prime factorization, this is equivalent to the fact that every prime power factor of \(\mf{a}\) appears in the prime factorization of \(\mf{b}\). We also say that \(\mf{a}\) \textbf{exactly divides}\index{exactly divides} \(\mf{b}\) and write \(\mf{a} \mid\mid \mf{b}\) if \(\mf{a}\) divides \(\mf{b}\) but no power of \(\mf{a}\) divides \(\mf{b}\). This is equivalent to the fact that every prime power factor of \(\mf{a}\) appears in the prime factorization of \(\mf{b}\) but not for any power of \(\mf{a}\). In the case of a prime \(\mf{p}\) and an integral ideal \(\mf{a}\), \(\mf{p} \mid \mf{a}\) if and only if \(\mf{p}\) is a prime factor of \(\mf{a}\) and \(\mf{p}^{e} \mid\mid \mf{a}\) for some positive integer \(e\) if and only if \(\mf{p}^{e}\) is exactly the power of \(\mf{p}\) appearing in the prime factorization of \(\mf{a}\). Moreover, if \(\mf{a} \mid \mf{p}\) then \(\mf{a} = \mf{p}\). The \textbf{greatest common divisor}\index{greatest common divisor} \((\mf{a},\mf{b})\) of \(\mf{a}\) and \(\mf{b}\) is defined to be the integral ideal that all other common integral ideal divisors divide. Since to divide is to contain, \((\mf{a},\mf{b})\) is the smallest ideal that contains both \(\mf{a}\) and \(\mf{b}\). This is \(\mf{a}+\mf{b}\) and so \((\mf{a},\mf{b}) = \mf{a}+\mf{b}\). The \textbf{least common multiple}\index{least common multiple} \([\mf{a},\mf{b}]\) of \(\mf{a}\) and \(\mf{b}\) is defined to be the integral ideal that divides all other common multiples. Since to divide is to contain, \([\mf{a},\mf{b}]\) is the largest integral ideal that is contained in both \(\mf{a}\) and \(\mf{b}\). This is \(\mf{a} \cap \mf{b}\) and so \([\mf{a},\mf{b}] = \mf{a} \cap \mf{b}\). Lastly, we say that \(\mf{a}\) and \(\mf{b}\) are \textbf{relatively prime}\index{relatively prime} if \((\mf{a},\mf{b}) = \smc{O}\). In other words, \(\mf{a}\) and \(\mf{b}\) are comaximal which is to say
    \[
      \mf{a}+\mf{b} = \smc{O}.
    \]
    This is equivalent to the prime factorizations of \(\mf{a}\) and \(\mf{b}\) containing distinct primes. In particular, distinct primes and their powers are relatively prime. As these integral ideals are comaximal, we also have \(\mf{a} \cap \mf{b} = \mf{a}\mf{b}\). In terms of the least common multiple, this means \([\mf{a},\mf{b}] = \mf{a}\mf{b}\).
    
    Just as it is common to suppress the fundamental theorem of arithmetic and just state the prime factorization of an integer, we suppress referencing \cref{thm:unique_product_prime_ideals} and simply state the prime factorization of a fractional ideal. We can now show that for a Dedekind domain, being a principal ideal domain is equivalent to being a unique factorization domain.

    \begin{proposition}\label{prop:UDF_PID_equivalence_for_Dedekind}
      Let \(\smc{O}\) be a Dedekind domain. Then \(\smc{O}\) is a principal ideal domain if and only if it is a unique factorization domain.
    \end{proposition}
    \begin{proof}
      The forward implication is immediate as every principal ideal domain is a unique factorization domain. For the reverse implication, suppose \(\smc{O}\) is a unique factorization domain. As every integral ideal factors into a product of primes by \cref{thm:unique_product_prime_ideals} and the product of principal integral ideals is principal, it suffices to show that primes are principal. Let \(\mf{p}\) be a prime. Then there exists nonzero \(\a \in \mf{p}\) and \(\a\) is not a unit since \(\mf{p}\) a proper ideal. Since \(\smc{O}\) is a unique factorization domain, let \(\a = \e\rho_{1}^{e_{1}} \cdots \rho_{r}^{e_{r}}\) be the prime factorization of \(\a\). Since \(\mf{p}\) is prime, it follows that there is some \(i\) such that \(\rho_{i} \in \mf{p}\). Without loss of generality, we may assume \(\rho_{1} \in \mf{p}\). Then the integral ideal \(\rho_{1}\smc{O}\) satisfies \(\rho_{1}\smc{O} \subseteq \mf{p}\). As \(\rho_{1}\) is prime, \(\rho_{1}\smc{O}\) is a prime and hence maximal since prime ideals are maximal in \(\smc{O}\). Whence \(\rho_{1}\smc{O} = \mf{p}\) proving \(\mf{p}\) is principal.
    \end{proof}

    With the prime factorization in hand, we will discuss the group structure of the fractional ideals of \(\smc{O}\). Let \(I_{\smc{O}}\) denote the set of fractional ideals of \(\smc{O}\). We call \(I_{\smc{O}}\) the \textbf{ideal group}\index{ideal group} of \(\smc{O}\). This is indeed a group as the following theorem demonstrates:

    \begin{theorem}\label{thm:ideal_group_is_a_group}
      Let \(\smc{O}\) be a Dedekind domain with field of fractions \(K\). Then \(I_{\smc{O}}\) is an abelian group with identity \(\smc{O}\).
    \end{theorem}
    \begin{proof}
      It is clear that the product of fractional ideals is a fractional ideal. Associativity and commutativity of \(I_{\smc{O}}\) are also obvious. The identity is \(\smc{O}\) because every fractional ideal is a finitely generated \(\smc{O}\)-submodule of \(K\). By \cref{lem:inverse_for_prime_ideals}, we see that \(\mf{p}^{-1}\) is the inverse of any prime \(\mf{p}\). Therefore every prime is invertible. If \(\mf{f}\) is a fractional ideal it admits a prime factorization \(\mf{f} = \mf{p}_{1}^{e_{1}} \cdots \mf{p}_{r}^{e_{r}}\) and then \(\mf{f}^{-1} = \mf{p}_{1}^{-e_{1}} \cdots \mf{p}_{r}^{-e_{r}}\) is its inverse. This completes the proof.
    \end{proof}

    It is possible to deduce an explicit description for the inverse \(\mf{f}^{-1}\) of any fractional ideal \(\mf{f}\).

    \begin{proposition}\label{prop:explicit_inverse_ideal}
      Let \(\smc{O}\) be a Dedekind domain with field of fractions \(K\) and let \(\mf{f}\) be a fractional ideal of \(\smc{O}\). Then
      \[
        \mf{f}^{-1} = \{\k \in K:\k\mf{f} \subseteq \smc{O}\}.
      \]
      In particular, \(\smc{O} \subseteq \mf{f}\) if and only if \(\mf{f}^{-1}\) is an integral ideal.
    \end{proposition}
    \begin{proof}
      Let \(\mf{f}\) be a fractional ideal. Then the inverse \(\mf{f}^{-1}\) exists by \cref{thm:ideal_group_is_a_group}. In the case of an integral ideal \(\mf{a}\), we have
      \[
        \mf{a}^{-1} = \{\k \in K:\k\mf{a} \subseteq \smc{O}\},
      \]
      by the prime factorization of \(\mf{a}\) and the definition of \(\mf{p}^{-1}\) for a prime \(\mf{p}\). Write \(\mf{f} = \frac{1}{\d}\mf{a}\) for some nonzero \(\d \in \smc{O}\) and integral ideal \(\mf{a}\). Whence
      \[
        \frac{1}{\d}\mf{f}^{-1} = \{\k \in K:\k\d\mf{f} \subseteq \smc{O}\},
      \]
      which is equivalent to the first statement. For the second statement, if \(\smc{O} \subseteq \mf{f}\) then multiplying by \(\mf{f}^{-1}\) shows \(\mf{f}^{-1} \subseteq \smc{O}\) whence \(\mf{f}^{-1}\) is an integral ideal. Running this argument backwards by multiplying by \(\mf{f}\) proves the converse.
    \end{proof}

    We will now discuss applications of the Chinese remainder theorem in the context of integral ideals. With it we can prove some interesting results. First, we recall a useful fact. Suppose \(\mf{a}\) is an integral ideal with prime factorization
    \[
      \mf{a} = \mf{p}_{1}^{e_{1}} \cdots \mf{p}_{r}^{e_{r}}.
    \]
    As powers of distinct primes are relatively prime, the integral ideals \(\mf{p}_{1}^{e_{1}},\ldots,\mf{p}_{r}^{e_{r}}\) are pairwise relatively prime. Whence they are pairwise comaximal so that the Chinese remainder theorem gives an isomorphism
    \[
      \smc{O}/\mf{a} \cong \bigop_{i}\smc{O}/\mf{p}_{i}^{e_{i}},
    \]
    induced by natural inclusion. In particular, for any choice of \(\a_{i} \in \smc{O}\) there exists a \(\a \in \smc{O}\) such that
    \[
      \a \equiv \a_{i} \pmod{\mf{p}_{i}^{e_{i}}}.
    \]
    We will use the Chinese remainder theorem to prove a few useful lemmas about Dedekind domains. Our first lemma shows that given two integral ideals, we can multiply by a relatively prime integral ideal and produce a principal integral ideal.

    \begin{lemma}\label{lem:relatively_prime_and_principal}
      Let \(\smc{O}\) be a Dedekind domain and \(\mf{a}\) and \(\mf{b}\) be integral ideals of \(\smc{O}\). Then there exists an integral ideal \(\mf{c}\) of \(\smc{O}\) relatively prime to \(\mf{b}\) such that \(\mf{a}\mf{c}\) is principal.
    \end{lemma}
    \begin{proof}
      Let \(\mf{p}_{1},\ldots,\mf{p}_{r}\) be the prime factors of both \(\mf{a}\) and \(\mf{b}\) so that
      \[
        \mf{a} = \mf{p}_{1}^{e_{1}} \cdots \mf{p}_{r}^{e_{r}} \quad \text{and} \quad \mf{b} = \mf{p}_{1}^{f_{1}} \cdots \mf{p}_{r}^{f_{r}},
      \]
      for some nonnegative integers \(e_{i}\) and \(f_{i}\). By prime factorization, there exists \(\a_{i} \in \mf{p}_{i}^{e_{i}}-\mf{p}_{i}^{e_{i}+1}\). As \(\mf{p}_{1}^{e_{1}+1},\ldots,\mf{p}_{r}^{e_{r}+1}\) are pairwise relatively prime, the Chinese remainder theorem implies the existence of an \(\a \in \smc{O}\) such that \(\a \equiv \a_{i} \tmod{\mf{p}_{i}^{e_{i}+1}}\). But then
      \[
        \a \equiv 0 \mod{\mf{p}_{i}^{e_{i}}} \quad \text{and} \quad \a \equiv \a_{i} \mod{\mf{p}_{i}^{e_{i}+1}}.
      \]
      Whence \(\mf{p}_{i}^{e_{i}} \mid\mid \a\smc{O}\). It follows that \((\a\smc{O},\mf{a}\mf{b}) = \mf{a}\). Then there exists an integral ideal \(\mf{c}\) such that \(\a\smc{O} = \mf{a}\mf{c}\) and necessarily \((\mf{a}\mf{c},\mf{a}\mf{b}) = \mf{a}\). Thus \((\mf{c},\mf{b}) = \smc{O}\) which is to say that \(\mf{c}\) must be relatively prime to \(\mf{b}\).
    \end{proof}

    Our second lemma shows that multiplying by fractional ideals does not affect quotients.

    \begin{lemma}\label{lem:cancellation_isomorphism}
      Let \(\smc{O}\) be a Dedekind domain where \(\mf{f}\), \(\mf{g}\), and \(\mf{h}\) be fractional ideals of \(\smc{O}\) with \(\mf{g} \subseteq \mf{f}\). Then we have an isomorphism
      \[
        \mf{f}/\mf{g} \cong \mf{f}\mf{h}/\mf{g}\mf{h}.
      \]
      In particular, there is an isomorphism
      \[
        \mf{a}^{-1}/\smc{O} \cong \smc{O}/\mf{a},
      \]
      for any integral ideal \(\mf{a}\) of \(\smc{O}\).
    \end{lemma}
    \begin{proof}
      Write \(\mf{f} = \frac{1}{\a}\mf{a}\), \(\mf{g} = \frac{1}{\b}\mf{b}\), and \(\mf{h} = \frac{1}{\g}\mf{c}\) for some nonzero \(\a,\b,\g \in \smc{O}\) and integral ideals \(\mf{a}\), \(\mf{b}\), and \(\mf{c}\) with \(\mf{b} \subseteq \mf{a}\). In view of the isomorphisms
      \[
        \mf{f}/\mf{g} \cong \b\mf{a}/\a\mf{b} \quad \text{and} \quad \mf{f}\mf{h}/\mf{g}\mf{h} \cong \b\mf{a}\mf{c}/\a\mf{b}\mf{c},
      \]
      it suffices to show
      \[
        \mf{a}/\mf{b} \cong \mf{a}\mf{c}/\mf{b}\mf{c}.
      \]
      To this end, as \(\mf{a} \mid \mf{b}\) we conclude \(\mf{b}\mf{a}^{-1}\) is an integral ideal. Then \cref{lem:relatively_prime_and_principal} implies the existence of an integral ideal \(\mf{d}\) relatively prime to \(\mf{b}\mf{a}^{-1}\) such that \(\mf{c}\mf{d} = \d\smc{O}\) for some nonzero \(\d \in \smc{O}\). Whence
      \[
        \mf{d}+\mf{b}\mf{a}^{-1} = \smc{O} \quad \text{and} \quad \mf{d} \cap \mf{b}\mf{a}^{-1} = \mf{d}\mf{b}\mf{a}^{-1}.
      \]
      Writing \(\mf{d} = \d\mf{c}^{-1}\), short computations show that these identities imply
      \[
        \d\mf{a}+\mf{b}\mf{c} = \mf{a}\mf{c} \quad \text{and} \quad \d^{-1}\mf{b}\mf{c} \cap \mf{a} = \mf{b}.
      \]
      Consider the homomorphism
      \[
        \phi:\mf{a} \to \mf{a}\mf{c}/\mf{b}\mf{c} \qquad \a \mapsto \d\a+\mf{b}\mf{c}.
      \]
      By the first isomorphism theorem, it suffices to show \(\phi\) is surjective and \(\ker\phi = \mf{b}\). Surjectivity follows from the first identity above. As \(\ker\phi = \d^{-1}\mf{b}\mf{c} \cap \mf{a}\), the second identity above implies \(\ker\phi = \mf{b}\). This proves the first statement. For the second statement, take \(\mf{f} = \mf{a}^{-1}\), \(\mf{g} = \smc{O}\), and \(\mf{h} = \mf{a}\).
    \end{proof}

    Returning to the general case of a Dedekind domain \(\smc{O}\) and a prime \(\mf{p}\), we call \(\F_{\mf{p}} = \smc{O}/\mf{p}\) the \textbf{residue class field}\index{residue class field} of \(\smc{O}\) by \(\mf{p}\). This is field as primes are maximal in Dedekind domains. Our last lemma gives an isomorphism between the residue class field and quotients by powers of a prime.

    \begin{lemma}\label{lem:isomorphism_of_quotient_by_prime_integral_ideals}
      Let \(\smc{O}\) be a Dedekind domain. Then for any prime \(\mf{p}\) of \(\smc{O}\) and nonnegative integer \(n\), we have an isomorphism
      \[
        \F_{\mf{p}} \cong \mf{p}^{n}/\mf{p}^{n+1}.
      \]
    \end{lemma}
    \begin{proof}
      By uniqueness of prime factorizations of fractional ideals, there exists \(\b \in \mf{p}^{n}-\mf{p}^{n+1}\). Consider the homomorphism
      \[
        \phi:\smc{O} \to \mf{p}^{n}/\mf{p}^{n+1} \qquad \a \mapsto \b\a+\mf{p}^{n+1}.
      \]
      By the first isomorphism theorem, it suffices to show \(\phi\) is surjective and \(\ker\phi = \mf{p}\). By our choice of \(\b\), we find that \(\b\mf{p}^{-n}\) is an integral ideal relatively prime to \(\mf{p}\). Whence
      \[
        \b\mf{p}^{-n}+\mf{p} = \smc{O} \quad \text{and} \quad \b\mf{p}^{-n} \cap \mf{p} = \b\mf{p}^{1-n}.
      \]
      Short computations show that these identities imply
      \[
        \b\smc{O}+\mf{p}^{n+1} = \mf{p}^{n} \quad \text{and} \quad \b^{-1}\mf{p}^{n+1} \cap \smc{O} = \mf{p}.
      \]
       Surjectivity follows from the first identity. As \(\ker\phi = \b^{-1}\mf{p}^{n+1} \cap \smc{O}\), the second identity implies \(\ker\phi = \mf{p}\).
    \end{proof}

    We now state two additional interesting propositions about Dedekind domains. The first proposition is that any Dedekind domain with only finitely many primes is a principal ideal domain.

    \begin{proposition}\label{prop:Dedekind_with_finite_primes_is_PID}
      Let \(\smc{O}\) be a Dedekind domain. If there are only finitely many primes of \(\smc{O}\) then \(\smc{O}\) is a principal ideal domain.
    \end{proposition}
    \begin{proof}
      Let \(\mf{p}_{1},\ldots,\mf{p}_{r}\) be the primes of \(\smc{O}\). Then for any integral ideal \(\mf{a}\), \cref{lem:relatively_prime_and_principal} implies the existence of an integral ideal \(\mf{b}\) relatively prime to \(\mf{p}_{1} \cdots \mf{p}_{r}\) such that \(\mf{a}\mf{b} = \a\smc{O}\) for some nonzero \(\a \in \smc{O}\). As \(\mf{b}\) is relatively prime to all of the primes of \(\smc{O}\) we must have \(\mf{b} = \smc{O}\). Then \(\mf{a} = \a\smc{O}\). As \(\mf{a}\) was arbitrary, \(\smc{O}\) is a principal ideal domain.
    \end{proof}
    
    The second proposition is that any fractional ideal is generated by at most two elements.

    \begin{proposition}\label{prop:fractional_ideal_generated_by_two_elements}
      Let \(\smc{O}\) be a Dedekind domain. Then every fractional ideal \(\mf{f}\) of \(\smc{O}\) is generated by at most two elements.
    \end{proposition}
    \begin{proof}
      Writing \(\mf{f} = \frac{1}{\d}\mf{a}\) for some nonzero \(\d \in \smc{O}\) and integral ideal \(\mf{a}\), it suffices to prove the claim \(\mf{a}\). Let \(\a \in \mf{a}\) be nonzero. By \cref{lem:relatively_prime_and_principal}, there exists an integral ideal \(\mf{b}\) relatively prime to \(\a\smc{O}\) such that \(\mf{a}\mf{b} = \b\smc{O}\) for some nonzero \(\b \in \smc{O}\). Whence
      \[
        \a\smc{O}+\mf{b} = \smc{O}.
      \]
      Upon multiplying by \(\mf{a}\), a short computation shows
      \[
        \a\smc{O}+\b\smc{O} = \mf{a}. 
      \]
      Therefore \(\mf{a}\) is generated by at most two elements.
    \end{proof}

    This result shows that while a Dedekind domain \(\smc{O}\) may not be a principal ideal domain, it is not far off from one since every integral ideal needs at most two generators. We can give a more refined interpretation of the degree to which \(\smc{O}\) fails to be a principal ideal domain using the ideal group \(I_{\smc{O}}\). Let \(P_{\smc{O}}\) denote the subgroup of principal fractional ideals of \(I_{\smc{O}}\). Since \(I_{\smc{O}}\) is abelian by \cref{thm:ideal_group_is_a_group}, \(P_{\smc{O}}\) is normal. The \textbf{ideal class group}\index{ideal class group} \(\Cl(\smc{O})\) of \(\smc{O}\) is defined to be the quotient group
    \[
      \Cl(\smc{O}) = I_{\smc{O}}/P_{\smc{O}},
    \]
    An element of \(\Cl(\smc{O})\) is called an \textbf{ideal class}\index{ideal class} of \(\smc{O}\). As every fractional ideal \(\mf{f}\) can be expressed as \(\mf{f} = \frac{1}{\d}\mf{a}\) for some nonzero \(\d \in \smc{O}\) and integral ideal \(\mf{a}\), we have \(\d\mf{f} = \mf{a}\) and it follows that every ideal class can be represented by an integral ideal. The \textbf{class number}\index{class number} \(h_{\smc{O}}\) of \(\smc{O}\) is defined by
    \[
      h_{\smc{O}} = |\Cl(\smc{O})|.
    \]
    The ideal class group encodes how much \(\smc{O}\) fails to be a principal ideal domain and the class number is a measure of the degree of failure. Indeed, \(\smc{O}\) is a principal ideal domain if and only if \(h_{\smc{O}} = 1\).

    \begin{remark}\label{rem:general_class_number_not_finite}
      The class number \(h_{\smc{O}}\) need not be finite for a general Dedekind domain \(\smc{O}\).
    \end{remark}
    
    The \textbf{unit group}\index{unit group} of \(\smc{O}\) is defined to be \(\smc{O}^{\ast}\). That is, the unit group is the group of units in \(\smc{O}\). The ideal class and unit groups of \(\smc{O}\) fit into an exact sequence.

    \begin{proposition}\label{prop:ideal_class_group_exact_sequence}
      Let \(\smc{O}\) be a Dedekind domain with field of fractions \(K\). Then the sequence
      \begin{center}
        \begin{tikzcd}
          1 \arrow{r} & \smc{O}^{\ast} \arrow{r} & K^{\ast} \arrow{r} & I_{\smc{O}} \arrow{r} & \Cl(\smc{O}) \arrow{r} & 1,
        \end{tikzcd}
      \end{center}
      where the middle map takes \(\k\) to its associated principal fractional ideal \(\k\smc{O}\), is exact.
    \end{proposition}
    \begin{proof}
      As the second map is injective and the fourth map is surjective, the sequence is exact at \(\smc{O}^{\ast}\) and \(\Cl(\smc{O})\). So it suffices to prove exactness at \(K^{\ast}\) and \(I_{\smc{O}}\). For exactness at \(K^{\ast}\), recall from \cref{thm:ideal_group_is_a_group} that \(\smc{O}\) is the identity of \(I_{\smc{O}}\). A principal integral ideal is \(\smc{O}\) if and only if it is generated by a unit in \(\smc{O}\) and exactness at \(K^{\ast}\) follows. We have exactness at \(I_{\smc{O}}\) because the principal fractional ideals represent the identity class of \(\Cl(\smc{O})\) and these are generated by elements of \(K^{\ast}\).
    \end{proof}

    Thinking of the third map in this exact sequence as passing from numbers in \(K^{\ast}\) to fractional ideals in \(I_{\smc{O}}\), exactness means that unit group is measuring the contraction (how many numbers are annihilated) taking place during this process while the class group is measuring the expansion (how many fractional ideals are created).
    
    \begin{remark}
      The class number \(h_{\smc{O}}\) and unit group \(\smc{O}^{\ast}\) are two of the most difficult pieces of algebraic data of \(\smc{O}\) to compute.
    \end{remark}

    Now let \(\smc{O}\) be a Dedekind domain with field of fractions \(K\), \(L/K\) be a finite separable extension, and \(\mc{O}\) be the integral closure of \(\smc{O}\) in \(L\). It turns out that \(\mc{O}\) is also a Dedekind domain.

    \begin{proposition}\label{prop:integral_closure_of_Dedekind_is_Dedekind}
      Let \(\smc{O}\) be a Dedekind domain with field of fractions \(K\), \(L/K\) be a finite separable extension, and \(\mc{O}\) be the integral closure of \(\smc{O}\) in \(L\). Then \(\mc{O}\) is a Dedekind domain with field of fractions \(L\).
    \end{proposition}
    \begin{proof}
      By \cref{prop:field_of_fractions_AKBL}, \(\mc{O}\) is an integrally closed domain with field of fractions \(L\). To show \(\mc{O}\) is noetherian, let the degree of \(L/K\) be \(n\) and \(\l_{1},\ldots,\l_{n}\) be a basis for \(L/K\). By \cref{prop:field_of_fractions_AKBL}, we may multiply by a nonzero element of \(\smc{O}\), if necessary, to ensure that this basis belongs to \(\mc{O}\). Being a basis of \(L/K\), we find that \(d_{L/K}(\l_{1},\ldots,\l_{n})\) is nonzero by \cref{prop:discriminant_not_zero} and \cref{lem:lemma_for_integral_basis_AKBL} implies
      \[
        d_{L/K}(\l_{1},\ldots,\l_{n})\mc{O} \subseteq \smc{O}\l_{1}+\cdots+\smc{O}\l_{n}.
      \]
      Thus \(\mc{O}\) is a finitely generated \(\smc{O}\)-module. In particular, every ideal of \(\mc{O}\) is also a finitely generated \(\smc{O}\)-module and therefore also a finitely generated \(\mc{O}\)-module. Whence \(\mc{O}\) is noetherian. It remains to show that every prime of \(\mc{O}\) is maximal. This is equivalent to showing \(\mc{O}/\mf{P}\) is a field. To this end, consider the homomorphism
      \[
        \phi:\smc{O} \to \mc{O}/\mf{P} \qquad \a \mapsto \a+\mf{p},
      \]
      induced by natural inclusion. Then \(\ker\phi = \mf{P} \cap \smc{O}\) and we claim \(\mf{P} \cap \smc{O}\) is a prime of \(\smc{O}\). It is clearly an ideal of \(\smc{O}\) and is prime because \(\mf{P}\) is. To see that it is nonzero, let \(\l \in \mf{P}\) be nonzero. As \(\l\) is integral over \(\smc{O}\), we have
      \[
        \l^{n}+\a_{n-1}\l^{n-1}+\cdots+\a_{0} = 0,
      \]
      for some positive integer \(n\) and \(\a_{i} \in \smc{O}\). Taking \(n\) minimal ensures \(\a_{0} \neq 0\). Isolating \(\a_{0}\) shows \(\a_{0} \in \mf{P}\) whence \(\a_{0} \in \mf{P} \cap \smc{O}\). Therefore \(\mf{P} \cap \smc{O} = \mf{p}\) for some prime \(\mf{p}\). This means \(\ker\phi = \mf{p}\). By the first isomorphism theorem, \(\phi\) induces an embedding \(\F_{\mf{p}} \to \mc{O}/\mf{P}\). Therefore \(\mc{O}/\mf{P}\) is a finite dimensional \(\F_{\mf{p}}\)-vector space as \(\mc{O}\) is a finitely generated \(\smc{O}\)-module. By primality of \(\mf{P}\), \(\mc{O}/\mf{P}\) is a finite integral domain and thus must be a field. This proves the first statement and the second is equivalent.
    \end{proof}

    Accordingly, a ring extension \(\mc{O}/\smc{O}\) is said to be a \textbf{Dedekind extension}\index{Dedekind extension} of a finite separable extension \(L/K\) if \(\mc{O}\) and \(\smc{O}\) are Dedekind domains whose field of fractions are \(L\) and \(K\) respectively and \(\mc{O}\) is the integral closure of \(\smc{O}\) in \(L\). By the preceding result, it is enough to show that \(\smc{O}\) is a Dedekind domain and \(\mc{O}\) is the integral closure of \(\smc{O}\) in \(L\). These Dedekind domains are related via the identity
    \begin{equation}\label{equ:Dedekind_extension_intersection}
      \mc{O} \cap K = \smc{O}.
    \end{equation}
    In general, it need not be true that \(\mc{O}/\smc{O}\) admits an integral basis as \(\mc{O}\) is not guaranteed to be a free \(\smc{O}\)-module of rank \(n\). However, an integral basis will exist if \(\smc{O}\) has finitely many primes for then it is a principal ideal domain by \cref{prop:Dedekind_with_finite_primes_is_PID} and we may appeal to \cref{thm:integral_basis_AKBL}.

    We now turn to the case of a number field \(K\) for which our developments so far can be refined. By \cref{prop:integral_closure_of_Dedekind_is_Dedekind}, the ring of integers \(\mc{O}_{K}\) is a Dedekind domain as \(\Z\) is a principal ideal domain. In view of this fact, we simplify some terminology. An \textbf{integral ideal}\index{integral ideal} of \(K\) is simply an integral ideal of \(\mc{O}_{K}\), a \textbf{prime}\index{prime} of \(K\) is a prime of \(\mc{O}_{K}\), and a \textbf{fractional ideal}\index{fractional ideal} of \(K\) is a fractional ideal of \(\mc{O}_{K}\). The \textbf{ideal group}\index{ideal group} \(I_{K}\) of \(K\) is the ideal group of \(\mc{O}_{K}\), we write \(P_{K}\) for the subgroup of principal fractional ideals of \(K\), and the \textbf{ideal class group}\index{ideal class group} \(\Cl(K)\) of \(K\) is the ideal class group of \(\mc{O}_{K}\). In particular,
    \[
      \Cl(K) = I_{K}/P_{K}.
    \]
    The \textbf{class number}\index{class number} \(h_{K}\) of \(K\) is the class number of \(\mc{O}_{K}\) and so
    \[
      h_{K} = |\Cl(K)|.
    \]
    The \textbf{unit group}\index{unit group} of \(K\) is the unit group of \(\mc{O}_{K}\) and we call any element of \(\mc{O}_{K}^{\ast}\) a \textbf{unit}\index{unit} of \(K\) (with the understanding that every element of \(K\) is invertible in \(K\)). It follows from \cref{thm:unique_product_prime_ideals} that integral ideals \(\mf{a}\) of \(K\) admit prime factorizations. One of our core investigations will be to understand how the principal integral ideal \(p\mc{O}_{K}\) factors into a product of primes of \(K\) for any prime \(p\). We will also leverage geometric ideas to show that the class number is finite and completely describe the unit group.
  \section{Localization}
    Let \(\smc{O}\) be an integral domain with field of fractions \(K\). A subset \(D\) of \(\smc{O}\) is said to be \textbf{multiplicative}\index{multiplicative} if it is closed under multiplication, \(1 \in D\), and \(0 \notin D\). The set \(\smc{O}-\{0\}\) is always multiplicative but more interesting examples are when we consider strict subsets. In any case, we define the \textbf{localization}\index{localization} \(\smc{O}D^{-1}\) of \(\smc{O}\) at \(D\) by
    \[
      \smc{O}D^{-1} = \left\{\frac{\eta}{\d} \in K:\eta \in \smc{O} \text{ and } \d \in D\right\}.
    \]
    Clearly \(\smc{O}D^{-1}\) is a subring of \(K\) which is an integral domain and is obtained from \(\smc{O}\) by making the elements of \(D\) invertible. More generally, for any fractional ideal \(\mf{f}\) of \(\smc{O}\), the \textbf{localization}\index{localization} \(\mf{f}D^{-1}\) of \(\mf{f}\) at \(D\) is defined by
    \[
      \mf{f}D^{-1} = \left\{\frac{\eta}{\d} \in K:\eta \in \mf{f} \text{ and } \d \in D\right\}.
    \]
    In particular, \(\mf{a}D^{-1}\) is an integral ideal of \(\smc{O}D^{-1}\) if \(\mf{a}\) is an integral ideal of \(\smc{O}\). Writing \(\mf{f} = \frac{1}{\d}\mf{a}\) for some nonzero \(\d \in \smc{O}\) and integral ideal \(\mf{a}\), it follows that \(\mf{f}D^{-1}\) is a fractional ideal of \(\smc{O}D^{-1}\).
    
    In the case of primes, we have an exact correspondence between those of \(\smc{O}\) disjoint from \(D\) and those of \(\smc{O}D^{-1}\).

    \begin{proposition}\label{prop:localization_prime_bijection}
      Let \(\smc{O}\) be an integral domain and \(D\) be a multiplicative subset of \(\smc{O}\). Then the maps
      \[
        \mf{q} \mapsto \mf{q}D^{-1} \quad \text{and} \quad \mf{Q} \mapsto \mf{Q} \cap \smc{O}.
      \]
      are inverse inclusion-preserving bijections between the primes \(\mf{q}\) of \(\smc{O}\) disjoint from \(D\) and the primes \(\mf{Q}\) of \(\smc{O}D^{-1}\).
    \end{proposition}
    \begin{proof}
      First suppose \(\mf{q}\) is a prime of \(\smc{O}\) that is disjoint from \(D\). Then the integral ideal \(\mf{q}D^{-1}\) of \(\smc{O}D^{-1}\) is prime because \(\mf{q}\) is. Also, the prime \(\mf{q}D^{-1}\) satisfies
      \[
        \mf{q} = \mf{q}D^{-1} \cap \smc{O},
      \]
      because \(\mf{q}\) is disjoint from \(D\). Now suppose \(\mf{Q}\) is a prime of \(\smc{O}D^{-1}\). Then the integral ideal \(\mf{Q} \cap \smc{O}\) of \(\smc{O}\) is prime because \(\mf{Q}\) is. Also, \(\mf{Q} \cap \smc{O}\) is disjoint from \(D\) for otherwise it contains a unit which is a contradiction as prime ideals are proper. Moreover, the prime \(\mf{Q} \cap \smc{O}\) satisfies
      \[
        \mf{Q} = (\mf{Q} \cap \smc{O})D^{-1}.
      \]
      All of this together shows that the mappings
      \[
        \mf{q} \mapsto \mf{q}D^{-1} \quad \text{and} \quad \mf{Q} \mapsto \mf{Q} \cap \smc{O}.
      \]
      are inverse bijections between the primes \(\mf{q}\) of \(\smc{O}\) disjoint from \(D\) and the primes \(\mf{Q}\) of \(\smc{O}D^{-1}\). They are clearly inclusion-preserving.
    \end{proof}

    It follows from this result that the primes of \(\smc{O}D^{-1}\) are of the form \(\mf{p}D^{-1}\) for primes \(\mf{p}\) of \(\smc{O}\) disjoint from \(D\). Often, \(D\) is chosen so that it is the compliment of a prime \(\mf{p}\). Indeed, \(\smc{O}-\mf{p}\) is a multiplicative subset precisely because \(\mf{p}\) is prime. In fact, let \(X\) be a set of primes of \(\smc{O}\) and consider
    \[
      \smc{O}-\bigcup_{\mf{p} \in X}\mf{p}.
    \]
    Then \(\smc{O}-\bigcup_{\mf{p} \in X}\mf{p}\) is a multiplicative subset because of the identity
    \[
      \smc{O}-\bigcup_{\mf{p} \in X}\mf{p} = \bigcap_{\mf{p} \in X}(\smc{O}-\mf{p}).
    \]
    In any case, we define the \textbf{localization}\index{localization} \(\smc{O}_{\mf{p}}\) of \(\smc{O}\) at \(\mf{p}\) by
    \[
      \smc{O}_{\mf{p}} = \smc{O}(\smc{O}-\mf{p})^{-1}.
    \]
    Localizing at a prime \(\mf{p}\) should be thought of as removing all of the algebraic information about \(\smc{O}\) that has nothing to do with \(\mf{p}\). More generally, if \(\mf{f}\) is a fractional ideal of \(\smc{O}\) then the \textbf{localization}\index{localization} \(\mf{f}_{\mf{p}}\) of \(\mf{f}\) at \(\mf{p}\) is defined to be
    \[
      \mf{f}_{\mf{p}} = \mf{f}(\smc{O}-\mf{p})^{-1}.
    \]
    In both of these constructions we have localized at a single prime. It is possible to localize at more than one prime. Let \(X\) be a set of primes in \(\smc{O}\). We define the \textbf{localization}\index{localization} \(\smc{O}(X)\) of \(\smc{O}\) at \(X\) by
    \[
      \smc{O}(X) = \smc{O}\left(\smc{O}-\bigcup_{\mf{p} \in X}\mf{p}\right)^{-1}.
    \]
    If \(\mf{f}\) is a fractional ideal of \(\smc{O}\) then the \textbf{localization}\index{localization} \(\mf{f}(X)\) of \(\mf{f}\) at \(X\) is defined to be
    \[
      \mf{f}(X) = \mf{f}\left(\smc{O}-\bigcup_{\mf{p} \in X}\mf{p}\right)^{-1}.
    \]
    If \(X\) consists of a single prime \(\mf{p}\) then \(\smc{O}(X) = \smc{O}_{\mf{p}}\) and \(\mf{f}(X) = \mf{f}_{\mf{p}}\).
    
    In order for localization to be a useful tool in algebraic investigations, it must behave well with respect to what we have already developed. To this end, we will collect some useful properties about localization. Our first property says that intersections of localizations behave well with respect to fractional ideals and units.

    \begin{proposition}\label{prop:ring_is_intersection_of_all_localizations}
      Let \(\smc{O}\) be an integral domain with field of fractions \(K\). Then
      \[
        \smc{O} = \bigcap_{\mf{p}}\smc{O}_{\mf{p}} \quad \text{and} \quad \smc{O}^{\ast} = \bigcap_{\mf{p}}\smc{O}_{\mf{p}}^{\ast}.
      \]
      In particular, for every fractional ideal \(\mf{f}\) of \(\smc{O}\), we have
      \[
        \mf{f} = \bigcap_{\mf{p}}\mf{f}_{\mf{p}}.
      \]
    \end{proposition}
    \begin{proof}
      For the first identity, the forward inclusion is obvious. For the reverse inclusion, suppose \(\frac{\eta}{\d} \in \bigcap_{\mf{p}}\smc{O}_{\mf{p}}\) and set
      \[
        \mf{a} = \{\a \in \smc{O}:\eta\a \in \d\smc{O}\}.
      \]
      Then \(\mf{a}\) is an integral ideal of \(\smc{O}\) and contains \(\d\). As \(\d\) is not contained in any prime of \(\smc{O}\), it follows that \(\mf{a}\) cannot be contained in any prime of \(\smc{O}\). Every proper ideal is contained in a maximal ideal which is necessarily prime. Whence \(\mf{a}\) is not proper and so \(\mf{a} = \smc{O}\). Therefore \(\eta \in \d\smc{O}\) which implies \(\frac{\a}{\b} \in \smc{O}\) proving the reverse inclusion. As for the second identity, the forward inclusion is clear. For the reverse inclusion, suppose \(\frac{\eta}{\d} \in \bigcap_{\mf{p}}\smc{O}_{\mf{p}}^{\ast}\). We have already shown \(\frac{\eta}{\d} \in \smc{O}\) so it suffices to show \(\frac{\eta}{\d}\) is a unit. As \(\d\) is not contained in any prime of \(\smc{O}\), it cannot be contained in any maximal ideal as maximal ideals are prime. This means \(\d\) is a unit. Interchanging the roles of \(\eta\) and \(\d\) shows that \(\eta\) is a unit as well. Hence \(\frac{\eta}{\d}\) is a unit and the reverse inclusion follows. The second statement is immediate from the first upon multiplying by \(\mf{f}\).
    \end{proof}
    
    Our second property is that localization respects integral closure.

    \begin{proposition}\label{prop:localization_of_integral_closure_is_integral_closure}
      Let \(\smc{O}\) be an integrally closed domain with field of fractions \(K\), \(L/K\) be a finite separable extension, and \(\mc{O}\) be the integral closure of \(\smc{O}\) in \(L\). Then for any multiplicative set \(D\) of \(\smc{O}\), \(\mc{O}D^{-1}\) is the integral closure of \(\smc{O}D^{-1}\) in \(L\). In particular, \(\mc{O}D^{-1}\) is an integrally closed domain with field of fractions \(L\).
    \end{proposition}
    \begin{proof}
      To prove the first statement, we must show \(\mc{O}D^{-1} = \conj{\smc{O}D^{-1}}\). For the forward inclusion, let \(\frac{\eta}{\d} \in \mc{O}D^{-1}\). As \(\mc{O}\) is the integral closure of \(\smc{O}\) in \(L\), \(\eta\) is integral over \(\smc{O}\) so that
      \[
        \eta^{n}+\a_{n-1}\eta^{n-1}+\cdots+\a_{0} = 0,
      \]
      for some positive integer \(n\) and \(\a_{i} \in \smc{O}\). Diving by \(\d^{n}\), we obtain
      \[
        \left(\frac{\eta}{\d}\right)^{n}+\frac{\a_{n-1}}{\d}\left(\frac{\eta}{\d}\right)^{n-1}+\cdots+\frac{\a_{0}}{\d^{n}} = 0.
      \]
      Thus \(\frac{\eta}{\d}\) is the root of a monic polynomial with coefficients in \(\smc{O}D^{-1}\) whence \(\frac{\eta}{\d} \in \conj{\smc{O}D^{-1}}\). This proves the forward inclusion. For the reverse inclusion, suppose \(\l \in \conj{\smc{O}D^{-1}}\). Then
      \[
        \l^{n}+\frac{\a_{n-1}}{\d_{n-1}}\l^{n-1}+\cdots+\frac{\a_{0}}{\d_{0}} = 0,
      \]
      for some positive integer \(n\) and \(\frac{\a_{i}}{\d_{i}} \in \smc{O}D^{-1}\). Letting \(\d = \d_{0}\cdots\d_{n-1}\) and multiplying by \(\d^{n}\), we obtain
      \[
        (\l\d)^{n}+\frac{\a_{n-1}\d}{\d_{n-1}}(\l\d)^{n-1}+\cdots+\frac{\a_{0}\d^{n}}{\d_{0}} = 0.
      \]
      It follows that \(\l\d\) is the root of a monic polynomial with coefficients in \(\smc{O}\). As \(\mc{O}\) is the integral closure of \(\smc{O}\) in \(L\), we have \(\l\d \in \mc{O}\) and thus \(\l \in \mc{O}D^{-1}\). This proves the reverse inclusion. For the second statement, observe that \(\mc{O}D^{-1}\) is an integral domain as it is a subring of a field. Moreover, the field of fractions of \(\mc{O}D^{-1}\) is \(L\) since the same is true for \(\mc{O}\) by \cref{prop:field_of_fractions_AKBL}. The second statement follows.
    \end{proof}

    Our third property is that the localization of a Dedekind domain is again a Dedekind domain.

    \begin{proposition}\label{prop:localization_of_Dedekind_is_Dedekind}
      Let \(\smc{O}\) be a Dedekind domain with field of fractions \(K\) and \(D\) be a multiplicative subset of \(\smc{O}\). Then \(\smc{O}D^{-1}\) is a Dedekind domain.
    \end{proposition}
    \begin{proof}
      We see from \cref{prop:localization_of_integral_closure_is_integral_closure} that \(\smc{O}D^{-1}\) is an integrally closed domain. To show it is noetherian, let \(\mf{A}\) be an ideal of \(\smc{O}D^{-1}\) and set \(\mf{a} = \mf{A} \cap \smc{O}\). Then
      \[
        \mf{A} = \mf{a}D^{-1}.
      \]
      Since \(\smc{O}\) is a Dedekind domain, \(\mf{a}\) is a finitely generated \(\smc{O}\)-module and hence \(\mf{A}\) is a finitely generated \(\smc{O}D^{-1}\)-module by the identity we have just proved. Therefore \(\smc{O}D^{-1}\) is noetherian. It remains to show that every prime of \(\smc{O}D^{-1}\) is maximal. By \cref{prop:localization_prime_bijection}, every prime is of the form \(\mf{p}D^{-1}\) for some prime \(\mf{p}\) of \(\smc{O}\). But then \(\mf{p}D^{-1}\) is maximal because \(\mf{p}\) is and the bijections in \cref{prop:localization_prime_bijection} are inclusion-preserving.
    \end{proof}

    An immediate consequence of this result is that the localizations \(\smc{O}_{\mf{p}}\) and \(\smc{O}(X)\) are Dedekind domains if \(\smc{O}\) is. In fact, from \cref{prop:integral_closure_of_Dedekind_is_Dedekind,prop:localization_of_integral_closure_is_integral_closure,prop:localization_of_Dedekind_is_Dedekind} we see that \(\mc{O}D^{-1}/\smc{O}D^{-1}\) is a Dedekind extension of a finite separable extension \(L/K\) if \(\mc{O}/\smc{O}\) is. In this case we setup some additional notation. If \(\mf{p}\) is a prime of \(\smc{O}\), we define the \textbf{localization}\index{localization} \(\mc{O}_{\mf{p}}\) of \(\mc{O}\) at \(\mf{p}\) by
    \[
      \mc{O}_{\mf{p}} = \mc{O}(\smc{O}-\mf{p})^{-1}.
    \]
    More generally, if \(\mf{F}\) is a fractional ideal of \(\mc{O}\) then the \textbf{localization}\index{localization} \(\mf{F}_{\mf{p}}\) of \(\mf{F}\) at \(\mf{p}\) is defined by
    \[
      \mf{F}_{\mf{p}} = \mf{F}(\smc{O}-\mf{p})^{-1}.
    \]
    We can also consider the case of more than a single prime. Let \(X\) be a set of primes of \(\smc{O}\). We define the \textbf{localization}\index{localization} \(\mc{O}(X)\) of \(\mc{O}\) at \(X\) by
    \[
      \mc{O}(X) = \mc{O}\left(\smc{O}-\bigcup_{\mf{p} \in X}\mf{p}\right)^{-1}.
    \]
    More generally, if \(\mf{F}\) is a fractional ideal of \(\mc{O}\) then the \textbf{localization}\index{localization} \(\mf{F}(X)\) of \(\mf{F}\) at \(X\) is defined by
    \[
      \mf{F}(X) = \mf{F}\left(\smc{O}-\bigcup_{\mf{p} \in X}\mf{p}\right)^{-1}.
    \]
    If \(X\) consists of a single prime \(\mf{p}\) then \(\mc{O}(X) = \mc{O}_{\mf{p}}\) and \(\mf{F}(X) = \mf{F}_{\mf{p}}\). Moreover, \(\mc{O}_{\mf{p}}/\smc{O}_{\mf{p}}\) and \(\mc{O}(X)/\smc{O}(X)\) are Dedekind extensions.

    Returning to the general setting, an integral domain \(\smc{O}\) is said to be a \textbf{local}\index{local} if it has a unique maximal ideal. As may be expected, the resulting integral domain after localizing at a prime will be local. In particular, \(\smc{O}_{\mf{p}}\) is local where the maximal ideal is \(\mf{p}_{\mf{p}}\). Indeed, recall from \cref{prop:localization_prime_bijection} that the map
    \[
      \mf{q} \to \mf{q}_{\mf{p}},
    \]
    is an inclusion-preserving bijection between the primes of \(\smc{O}\) contained in \(\mf{p}\) the primes of \(\smc{O}_{\mf{p}}\). As maximal ideals are prime, \(\mf{p}_{\mf{p}}\) is the unique maximal ideal of \(\smc{O}_{\mf{p}}\). As for consequences, we have
    \[
      \smc{O}_{\mf{p}}^{\ast} = \smc{O}_{\mf{p}}-\mf{p}_{\mf{p}} \quad \text{and} \quad \smc{O}_{\mf{p}}^{\ast}+\mf{p}_{\mf{p}} = \smc{O}_{\mf{p}}^{\ast}.
    \]
    In other words, the units of \(\smc{O}_{\mf{p}}\) are precisely the elements not in \(\mf{p}_{\mf{p}}\) and the sum of a unit of \(\smc{O}_{\mf{p}}\) and an element of \(\mf{p}_{\mf{p}}\) is again a unit of \(\smc{O}_{\mf{p}}\). If \(\mf{p}\) itself is maximal we can say more.

    \begin{proposition}\label{prop:localization_at_prime_is_local}
      Let \(\smc{O}\) be an integral domain and \(\mf{p}\) be a prime of \(\smc{O}\). Then there is an embedding
      \[
        \smc{O}/\mf{p}^{n} \to \smc{O}_{\mf{p}}/\mf{p}_{\mf{p}}^{n},
      \]
      induced by natural inclusion for every positive integer \(n\). In particular, this embedding identifies \(\smc{O}_{\mf{p}}/\mf{p}_{\mf{p}}\) with the field of fractions of \(\smc{O}/\mf{p}\) when \(n  = 1\). Moreover, if \(\mf{p}\) is maximal we have an isomorphism
      \[
        \smc{O}/\mf{p}^{n} \cong \smc{O}_{\mf{p}}/\mf{p}_{\mf{p}}^{n},
      \]
      induced by natural inclusion.
    \end{proposition}
    \begin{proof}
      Consider the homomorphism
      \[
        \phi:\smc{O} \to \smc{O}_{\mf{p}}/\mf{p}_{\mf{p}}^{n} \qquad \a \mapsto \a+\mf{p}_{\mf{p}}^{n},
      \]
      induced by natural inclusion. Note that \(\ker\phi = \mf{p}_{\mf{p}}^{n} \cap \smc{O}\). By \cref{prop:localization_prime_bijection}, \(\mf{p}_{\mf{p}} \cap \smc{O} = \mf{p}\) and by induction
      \[
        \mf{p}_{\mf{p}}^{n} \cap \smc{O} = \mf{p}^{n}.
      \]
      Whence \(\ker\phi = \mf{p}^{n}\). Then the first isomorphism theorem induces the desired embedding. As \(\mf{p}_{\mf{p}}\) is maximal, \(\smc{O}_{\mf{p}}/\mf{p}_{\mf{p}}\) is a field and thus must be the field of fractions of \(\smc{O}/\mf{p}\) under this embedding when \(n = 1\). It remains to prove the last statement. So suppose \(\mf{p}\) is maximal. It suffices to show \(\phi\) is surjective. As \(\smc{O}_{\mf{p}}\) is local, every element not in \(\mf{p}_{\mf{p}}\) is a unit which is to say
      \[
         \smc{O}_{\mf{p}}^{\ast}+\mf{p}_{\mf{p}} = \smc{O}_{\mf{p}}.
      \]
      As \(\mf{p}\) is maximal, \(\smc{O}/\mf{p}\) is a field whence \(\smc{O}_{\mf{p}}^{\ast} \subseteq \smc{O}+\mf{p}_{\mf{p}} \subseteq \smc{O}_{p}\). This implies \(\smc{O}+\mf{p}_{\mf{p}} = \smc{O}_{\mf{p}}\). It follows by induction that
      \[
        \smc{O}+\mf{p}_{\mf{p}}^{n} = \smc{O}_{\mf{p}}.
      \]
      This shows surjectivity.
    \end{proof}
    
    We wil now restrict to principal ideal domains rather than integral domains. An integral domain \(\smc{O}\) is said to be a \textbf{discrete valuation ring}\index{discrete valuation ring} if it is a principal ideal domain with a unique maximal ideal. Equivalently, \(\smc{O}\) is a local principal ideal domain or a Dedekind domain with exactly one prime.
    
    Let \(\mf{p}\) be the unique maximal ideal of \(\smc{O}\). In particular, \(\mf{p}\) is prime and of the form \(\mf{p} = \pi\smc{O}\) for some prime \(\pi \in \smc{O}\). We call \(\pi\) a \textbf{uniformizer}\index{uniformizer} of \(\smc{O}\) and it is uniquely defined up to multiplication by a unit. On the one hand, \(\smc{O}\) is local so every element not in \(\mf{p}\) is a unit. On the other hand, \(\smc{O}\) is a principal ideal domain so that \(\pi\) is the only prime of \(\smc{O}\). Since \(\smc{O}\) is necessarily a unique factorization domain, these facts together imply that every \(\a \in \smc{O}\) is of the form
    \[
      \a = \e\pi^{n},
    \]
    for some nonnegative integer \(n\) and unit \(\e\). Moreover, \(\a\) generates the integral ideal \(\mf{p}^{n}\) and every integral ideal is of this form. If \(K\) is the field of fractions of \(\smc{O}\), it follows that every nonzero \(\k \in K\) can be uniquely expressed as
    \[
      \k = \e\pi^{n},
    \]
    for some integer \(n\) and unit \(\e\). The most important data of a discrete valuation ring is its valuation. The \textbf{valuation}\index{valuation} \(v\) associated to \(\smc{O}\) on \(K\) is the function defined by
    \[
      v:K \to \Z \cup \{\infty\} \qquad \k \mapsto v(\k) = \begin{cases} n & \text{if \(\k = \e\pi^{n}\)}, \\ \infty & \text{if \(\k = 0\)}. \end{cases}
    \]
    We call \(v(\k)\) the \textbf{valuation}\index{valuation} of \(\k\) with respect to \(\smc{O}\). Note that \(v(\k) = 0\) if and only if \(\k\) is a unit in \(\smc{O}\). If \(\k \in \smc{O}\) then the \(v(\k)\) is characterized by the equation
    \[
      \k\smc{O} = \mf{p}^{v(\k)},
    \]
    since \(\k = \e\pi^{v(\k)}\). Moreover, if \(\k = \e\pi^{n}\) and \(\eta = \d\pi^{m}\), we have
    \[
      \k\eta = \e\d\pi^{n+m} \quad \text{and} \quad \k+\eta = (\e\pi^{n-\min(n,m)}+\d\pi^{m-\min(n,m)})\pi^{\min(n,m)},
    \]
    where \(\e\pi^{n-\min(n,m)}+\d\pi^{m-\min(n,m)}\) is a unit. These identities imply that \(v\) satisfies the properties
    \[
      v(\k\eta) = v(\k)+v(\eta) \quad \text{and} \quad v(\k+\eta) \ge \min(v(\k),v(\eta)).
    \]
    The first of which shows that
    \[
      v:K^{\ast} \to \Z \qquad \k \mapsto v(\k),
    \]
    is a surjective homomorphism.

    If \(\smc{O}\) is a Dedekind domain then it is local if and only if it is a discrete valuation ring. Indeed, any Dedekind domain with finitely many primes is a principal ideal domain by \cref{prop:Dedekind_with_finite_primes_is_PID}. This will allow us to establish a connection between Dedekind domains and discrete valuation rings. In particular, a noetherian domain is a Dedekind domain if and only if its localization at any prime is a discrete valuation ring.

    \begin{theorem}\label{thm:Dedekind_if_and_only_if_all_localizations_are_discrete_valuation_rings}
      Let \(\smc{O}\) be a noetherian domain. Then \(\smc{O}\) is a Dedekind domain if and only if \(\smc{O}_{\mf{p}}\) is a discrete valuation ring for all primes \(\mf{p}\).
    \end{theorem}
    \begin{proof}
      For the forward implication, suppose \(\smc{O}\) is a Dedekind domain and let \(\mf{p}\) be a prime. Then \(\smc{O}_{\mf{p}}\) is a local Dedekind domain and hence a discrete valuation ring as we have seen. This proves the forward implication. For the reverse implication, suppose \(\smc{O}\) is a noetherian domain and \(\smc{O}_{\mf{p}}\) is a discrete valuation ring for all primes \(\mf{p}\). As \(\smc{O}_{\mf{p}}\) is a principal ideal domain, \(\smc{O}_{\mf{p}}\) is an integrally closed domain by \cref{lemma:unique_factorization_domains_are_integrally_closed}. Then \cref{prop:ring_is_intersection_of_all_localizations} implies \(\smc{O}\) is an integrally closed domain too. As \(\smc{O}\) is noetherian by assumption, it remains to show that every prime is maximal. So let \(\mf{q}\) be prime. As \(\mf{q}\) is proper, \(\mf{q}\) is contained in some maximal ideal \(\mf{p}\) which is necessarily prime. The inclusion-preserving bijections in \cref{prop:localization_prime_bijection} show that \(\mf{q}_{\mf{p}}\) and \(\mf{p}_{\mf{p}}\) are primes of \(\smc{O}_{\mf{p}}\). As \(\smc{O}_{\mf{p}}\) is a discrete valuation ring it is a Dedekind domain with exactly one prime. Whence \(\mf{q}_{\mf{p}} = \mf{p}_{\mf{p}}\) and the aforementioned inclusion-preserving bijections imply that \(\mf{q} = \mf{p}\). Hence \(\mf{q}\) is maximal. This proves the reverse implication.
    \end{proof}

    It follows from this result that the Dedekind extension \(\mc{O}_{\mf{p}}/\smc{O}_{\mf{p}}\) is an extension of principal ideal domains. Whence it admits an integral basis by \cref{thm:integral_basis_AKBL}. This is often a useful reduction for developing further results or applications in conjunction with \cref{prop:ring_is_intersection_of_all_localizations}.

    Let \(v_{\mf{p}}\) denote the valuation of \(\smc{O}_{\mf{p}}\). We call \(v_{\mf{p}}\) the \textbf{valuation}\index{valuation} associated to the prime \(\mf{p}\) of \(\smc{O}\). These valuations are intimately connected to prime factorizations. Indeed, prime factorization implies that for any \(\k \in K^{\ast}\), we have
    \[
      \k\smc{O} = \prod_{\mf{q}}\mf{q}^{e_{\mf{q}}},
    \]
    for some integers \(e_{\mf{q}}\) all but finitely many of which are zero. We claim that \(v_{\mf{p}}(\k) = e_{\mf{p}}\). To see this, first observe that if \(\mf{p}\) and \(\mf{q}\) are distinct primes then \(\mf{q}_{\mf{p}} = \smc{O}_{\mf{p}}\). Indeed, as these primes are distinct choose \(\a \in \mf{q}-\mf{p}\). Then \(\a \in \mf{q}_{\mf{p}}-\mf{p}_{\mf{p}}\) by primality of \(\mf{p}\). As \(\smc{O}_{\mf{p}}\) is local, \(\a\) must be a unit whence \(\mf{q}_{\mf{p}} = \smc{O}_{\mf{p}}\). This forces
    \[
      \k\smc{O}_{\mf{p}} = \mf{p}_{\mf{p}}^{e_{\mf{p}}},
    \]
    and we readily see that \(v_{\mf{p}}(\k) = e_{\mf{p}}\). In particular, \(v_{\mf{p}}(\k) = 0\) for all but finitely many primes \(\mf{p}\). In view of these properties, \(v_{\mf{p}}\) is also called an \textbf{exponential valuation}\index{exponential valuation}.

    Continue to let \(\smc{O}\) be a Dedekind domain with field of fractions \(K\) and let \(X\) be a set of all but finitely many primes of \(\smc{O}\). Then \(\smc{O}(X)\) is a Dedekind domain. By \cref{prop:localization_prime_bijection}, the primes \(\mf{p}_{X}\) of \(\smc{O}(X)\) are of the form \(\mf{p}_{X} = \mf{p}(X)\) for \(\mf{p} \in X\). Moreover, \(\smc{O}\) and \(\smc{O}(X)\) have the same localizations at \(\mf{p}\) and \(\mf{p}_{X}\) respectively because the only elements which are not inverted are those of \(\mf{p}\). In particular, this means
    \begin{equation}\label{equ:localizing_at_primes_for_X_is_the_same_as_localizing_at_primes} 
      \smc{O}_{\mf{p}} = \smc{O}(X)_{\mf{p}_{X}} \quad \text{and} \quad \mf{f}_{\mf{p}} = \mf{f}(X)_{\mf{p}_{X}}.
    \end{equation}
    The relationship between \(\smc{O}\) and \(\smc{O}(X)\) can be expressed via an exact sequence.

     \begin{proposition}\label{prop:ideal_class_group_localization_exact_sequence}
      Let \(\smc{O}\) be a Dedekind domain with field of fractions \(K\) and let \(X\) be a set of all but finitely many primes of \(\smc{O}\). Then the sequence
      \begin{center}
        \begin{tikzcd}
          1 \arrow{r} & \smc{O}^{\ast} \arrow{r} & \smc{O}(X)^{\ast} \arrow{r} & \displaystyle{\bigop_{\mf{p} \notin X}K^{\ast}/\smc{O}_{\mf{p}}^{\ast}} \arrow{r} & \Cl(\smc{O}) \arrow{r} & \Cl(\smc{O}(X)) \arrow{r} & 1,
        \end{tikzcd}
      \end{center}
      where the fourth map takes the representative \((\k_{\mf{p}})_{\mf{p} \notin X}\) to the ideal class represented by \(\prod_{\mf{p} \notin X}\mf{p}^{v_{\mf{p}}(\k_{\mf{p}})}\) and the fifth map takes the representative \(\mf{a}\) to the ideal class represented by \(\mf{a}(X)\), is exact. Moreover, \(K^{\ast}/\smc{O}_{\mf{p}}^{\ast} \cong \Z\) for all primes \(\mf{p}\) of \(\smc{O}\).
    \end{proposition}
    \begin{proof}
      As the second map is injective and the fifth map is surjective, the sequence is exact at \(\smc{O}^{\ast}\) and \(\Cl(\smc{O}(X))\). So we must show exactness at \(\smc{O}(X)^{\ast}\), \(\bigop_{\mf{p} \notin X}K^{\ast}/\smc{O}_{\mf{p}}^{\ast}\), and \(\Cl(\smc{O})\). For exactness at \(\smc{O}(X)^{\ast}\), \(\a \in \smc{O}(X)^{\ast}\) represents the zero coset in under the third map \(\bigop_{\mf{p} \notin X}K^{\ast}/\smc{O}_{\mf{p}}^{\ast}\) if and only if \(\a \in \smc{O}_{\mf{p}}^{\ast}\) for all \(\mf{p} \notin X\). But \cref{equ:localizing_at_primes_for_X_is_the_same_as_localizing_at_primes} implies \(\a \in \smc{O}_{\mf{p}}^{\ast}\) for all primes \(\mf{p} \in X\). Whence \(\a\) represents the zero coset if and only if \(\a \in \smc{O}^{\ast}\) by \cref{prop:ring_is_intersection_of_all_localizations} proving exactness at \(\smc{O}(X)^{\ast}\). For exactness at \(\bigop_{\mf{p} \notin X}K^{\ast}/\smc{O}_{\mf{p}}^{\ast}\), a representative \((\k_{\mf{p}})_{\mf{p} \notin X}\) of a coset in \(\bigop_{\mf{p} \notin X}K^{\ast}/\smc{O}_{\mf{p}}^{\ast}\) represents the principal class in \(\Cl(\smc{O})\) under the fourth map if and only if there is a \(\k \in K^{\ast}\) such that
      \[
        \prod_{\mf{p} \notin X}\mf{p}^{v_{\mf{p}}(\k_{\mf{p}})} = \prod_{\mf{p}}\mf{p}^{v_{\mf{p}}(\k)}.
      \]
      By prime factorization, \(v_{\mf{p}}(\k) = v_{\mf{p}}(\k_{\mf{p}})\). In particular, \(v_{\mf{p}}(\k) = 0\) for all \(\mf{p} \in X\) and \(v_{\mf{p}}(\k) = v_{\mf{p}}(\k_{\mf{p}})\) for all \(\mf{p} \notin X\). As \(v_{\mf{p}}(\k) = 0\) for all \(\mf{p} \in X\), we have \(\k \in \smc{O}_{\mf{p}}^{\ast}\) for these primes as well. Because the primes of \(\smc{O}(X)\) are \(\mf{p}_{X}\) for \(\mf{p} \in X\), \cref{equ:localizing_at_primes_for_X_is_the_same_as_localizing_at_primes,prop:ring_is_intersection_of_all_localizations} together imply \(\k \in \smc{O}(X)^{\ast}\). Exactness at \(\bigop_{\mf{p} \notin X}K^{\ast}/\smc{O}_{\mf{p}}^{\ast}\) follows. For exactness at \(\Cl(\smc{O})\), an integral ideal \(\mf{a}\) representing a class in \(\Cl(\smc{O})\) represents the principal class under the fifth map if and only if there is a \(\k \in K^{\ast}\) such that
      \[
        \mf{a}(X) = \k\smc{O}(X).
      \]
      In view of \cref{equ:localizing_at_primes_for_X_is_the_same_as_localizing_at_primes,prop:ring_is_intersection_of_all_localizations} again, taking the intersection with the localizations at \(\mf{p}_{X}\) for all \(\mf{p} \notin X\) shows
      \[
        \mf{a} = \k\smc{O}.
      \]
      Therefore \((v_{\mf{p}}(\k))_{\mf{p} \notin X}\) is a representative of a coset in \(\bigop_{\mf{p} \notin X}K^{\ast}/\smc{O}_{\mf{p}}^{\ast}\) whose image under the fourth map is \(\mf{a}\) proving exactness at \(\Cl(\smc{O})\). 
      
      The last statement follows from the first isomorphism theorem since the valuation \(v_{\mf{p}}\) restricted to \(K^{\ast}\) is a surjective homomorphism and the kernel is exactly \(\smc{O}_{\mf{p}}^{\ast}\). This completes the proof.
    \end{proof}

    We now turn to the setting of a number field \(K\). Let \(S\) denote a finite set of primes of \(\mc{O}_{K}\) and let \(X\) denote the set of all primes that do not belong to \(S\). The \textbf{ring of \(S\)-integers}\index{ring of \(S\)-integers} \(\mc{O}_{K}^{S}\) of \(K\) is defined by
    \[
      \mc{O}_{K}^{S} = \mc{O}_{K}(X).
    \]
    We call any \(\a \in \mc{O}_{K}^{S}\) an \textbf{algebraic \(S\)-integer}\index{algebraic \(S\)-integer}. The \textbf{\(S\)-class group}\index{\(S\)-class group} \(\Cl^{S}(K)\) of \(K\) is the ideal class group of \(\mc{O}_{K}^{S}\). The \textbf{\(S\)-class number}\index{\(S\)-class number} \(h_{K}^{S}\) of \(K\) is the class number of \(\Cl^{S}(K)\). The \textbf{\(S\)-unit group}\index{\(S\)-unit group} of \(K\) is the unit group \((\mc{O}_{K}^{S})^{\ast}\) of \(\mc{O}_{K}^{S}\) and we call any element of \((\mc{O}_{K}^{S})^{\ast}\) an \textbf{\(S\)-unit}\index{\(S\)-unit} of \(K\).
  \iffalse\section{\todo{Orders}}
    We will often consider generalizations of Dedekind domains by relaxing the condition that they are integrally closed in their field of fractions. As every prime of \(\smc{O}\) being maximal is equivalent to \(\smc{O}\) having Krull dimension one, these generalizations are one-dimensional noetherian domains.
    
    An important fact about one-dimensional noetherian domains is that a generalization of the Chinese remainder theorem holds.

    \begin{proposition}\label{prop:localized_Chinese_remainder_theorem}
      Let \(\smc{O}\) be a one-dimensional noetherian domain. Then for any integral ideal \(\mf{a}\) of \(\smc{O}\), we have an isomorphism
      \[
        \smc{O}/\mf{a} \cong \bigop_{\mf{p}}\smc{O}_{\mf{p}}/\mf{a}_{\mf{p}},
      \]
      induced by natural inclusion.
    \end{proposition}
    \begin{proof}
      \todo{fix proof using one-dimensional noetherian assumption}
      Let \(\wtilde{\mf{a}}_{\mf{p}} = \mf{a}_{\mf{p}} \cap \smc{O}\). Then \(\wtilde{\mf{a}}_{\mf{p}}\) is an integral ideal of \(\smc{O}\). If \(\mf{p}\) is not a prime factor of \(\mf{a}\), then \(\mf{a} \subsetneq \mf{p}\). Whence \(\mf{a}\) contains an element of \(\smc{O}-\mf{p}\). This forces \(\mf{a}_{\mf{p}} = \smc{O}_{\mf{p}}\) and so \(\wtilde{\mf{a}}_{\mf{p}} = \smc{O}\). If \(\mf{p}\) is a prime factor of \(\mf{a}\), then the inclusion-preserving bijections in \cref{prop:localization_prime_bijection} show that \(\mf{p}\) is the only prime containing \(\wtilde{\mf{a}}_{\mf{p}}\) and \(\mf{a}_{p}\) is maximal in \(\smc{O}_{\mf{p}}\) as \(\mf{p}\) is maximal. In either case, these facts together imply that the \(\wtilde{\mf{a}}_{\mf{p}}\) are pairwise comaximal. Now we claim
      \[
        \mf{a} = \bigcap_{\mf{p}}\wtilde{\mf{a}}_{\mf{p}}.
      \]
      The forward inclusion is obvious. For the reverse inclusion, suppose \(\frac{\eta}{\d} \in \bigcap_{\mf{p}}\wtilde{\mf{a}}_{\mf{p}}\) and set
      \[
        \mf{b} = \{\b \in \smc{O}:\b\eta \in \d\mf{a}\}.
      \]
      Then \(\mf{b}\) is an integral ideal of \(\smc{O}\) and contains \(\d\). As \(\d\) is not contained in any prime of \(\smc{O}\), it follows that \(\mf{b}\) cannot be contained in any prime of \(\smc{O}\). Every proper ideal is contained in a maximal ideal which is necessarily prime. Whence \(\mf{b}\) is not proper and so \(\mf{b} = \smc{O}\). In particular, \(1 \in \mf{b}\). Therefore \(\eta \in \d\mf{a}\) which implies \(\frac{\eta}{\d} \in \mf{a}\) proving the reverse inclusion. Altogether, the Chinese remainder theorem gives an isomorphism
      \[
        \smc{O}/\mf{a} \cong \bigop_{\mf{p}}\smc{O}/\wtilde{\mf{a}}_{\mf{p}},
      \]
      induced by natural inclusion. Now consider the homomorphism
      \[
        \phi:\smc{O} \to \smc{O}_{\mf{p}}/\mf{a}_{\mf{p}} \qquad \a \mapsto \a+\mf{a}_{\mf{p}},
      \]
      induced by natural inclusion. By the first isomorphism theorem, it suffices to show \(\phi\) is surjective and \(\ker\phi = \wtilde{\mf{a}}_{\mf{p}}\). Surjectivity follows as
      \[
        \smc{O}+\mf{a}_{\mf{p}} = \smc{O}_{\mf{p}},
      \]
      since \(\mf{a}_{\mf{p}}\) is either \(\smc{O}_{\mf{p}}\) or maximal. Moreover, \(\ker\phi = \wtilde{\mf{a}}_{\mf{p}}\) from the definition of \(\wtilde{\mf{a}}_{\mf{p}}\).
    \end{proof}
  \fi

  \iffalse If \(\t\) is a primitive element for \(K/\Q\) contained in \(\mc{O}_{K}\) then the \textbf{conductor}\index{conductor} \(\mf{q}_{K}\) of \(K\) relative to \(\t\) is the conductor of \(\mc{O}_{K}/\Z\) relative to \(\t\). Even though \(K\) admits an integral basis, \(\mc{O}_{K}\) is not necessarily monogenic so that \(\mf{q}_{K}\) may not be \(\mc{O}_{K}\). \todo{say something about Galois number fields...}
  \fi