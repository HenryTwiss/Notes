\chapter{Splitting of Primes and Ramification}
  Throughout, all rings will be understood to be commutative and with unity.
  \section{Splitting of Primes}
    We begin with an important proposition regarding a Dedekind extension \(\mc{O}/\smc{O}\) of a finite separable extension \(L/K\). This is essentially a collection of some results in the proof of \cref{prop:integral_closure_of_Dedekind_is_Dedekind}

    \begin{proposition}\label{prop:Dedekind_extension_facts}
      Let \(\mc{O}/\smc{O}\) be a Dedekind extension of a degree \(n\) separable extension \(L/K\). Then \(\mc{O}\) is a finitely generated \(\smc{O}\)-module of rank at most \(n\). Moreover, every prime \(\mf{P}\) of \(\mc{O}\) satisfies
      \[
        \mf{P} \cap \smc{O} = \mf{p},
      \]
      for some prime \(\mf{p}\) of \(\smc{O}\) and \(\F_{\mf{P}}/\F_{\mf{p}}\) is a field extension of degree at most \(n\).
    \end{proposition}
    \begin{proof}
      Let \(\l_{1},\ldots,\l_{n}\) be a basis for \(L/K\). By \cref{prop:field_of_fractions_AKBL}, we may multiply by a nonzero element of \(\smc{O}\), if necessary, to ensure that this basis belongs to \(\mc{O}\). Being a basis of \(L/K\), \(d_{L/K}(\l_{1},\ldots,\l_{n})\) is nonzero by \cref{prop:discriminant_not_zero} and \cref{lem:lemma_for_integral_basis_AKBL} implies
      \[
        d_{L/K}(\l_{1},\ldots,\l_{n})\mc{O} \subseteq \smc{O}\l_{1}+\cdots+\smc{O}\l_{n}.
      \]
      Thus \(\mc{O}\) is a finitely generated \(\smc{O}\)-module of rank at most \(n\). This proves the first statement.
      
      For the second statement, let \(\mf{P}\) be a prime of \(\mc{O}\). Then \(\mf{P} \cap \smc{O}\) is an ideal of \(\smc{O}\) and is prime because \(\mf{P}\) is. To see that it is nonzero, let \(\l \in \mf{P}\) be nonzero. As \(\l\) is integral over \(\smc{O}\), we have
      \[
        \l^{n}+\a_{n-1}\l^{n-1}+\cdots+\a_{0} = 0,
      \]
      for some positive integer \(n\) and \(\a_{i} \in \smc{O}\). Taking \(n\) minimal ensures \(\a_{0} \neq 0\). Isolating \(\a_{0}\) shows \(\a_{0} \in \mf{P}\) whence \(\a_{0} \in \mf{P} \cap \smc{O}\). Therefore \(\mf{P} \cap \smc{O}\) is a prime of \(\smc{O}\) which is to say
      \[
        \mf{P} \cap \smc{O} = \mf{p}.
      \]
      for some prime \(\mf{p}\). Now consider the homomorphism
      \[
        \phi:\smc{O} \to \F_{\mf{P}} \qquad \a \mapsto \a+\mf{p},
      \]
      We have \(\ker\phi = \mf{P} \cap \smc{O}\) and thus \(\ker\phi = \mf{p}\) by the previous identity. The first isomorphism theorem implies that \(\phi\) induces an embedding \(\F_{\mf{p}} \to \F_{\mf{P}}\). This means \(\F_{\mf{P}}\) is a \(\F_{\mf{p}}\)-vector space of dimension at most \(n\) as \(\mc{O}\) is a finitely generated \(\smc{O}\)-module of rank at most \(n\). In other words, \(\F_{\mf{P}}/\F_{\mf{p}}\) is a field extension of degree at most \(n\).
    \end{proof}

    In view of this result, if \(\mf{f}\) is a fractional ideal of \(\smc{O}\) then \(\mf{f}\mc{O}\) is a fractional ideal of \(\mc{O}\) because \(\mf{f}\) is a finitely generated \(\smc{O}\)-module and hence a finitely generated \(\mc{O}\)-module. In particular, \(\mf{a}\mc{O}\) is an integral ideal of \(\mc{O}\) for any integral ideal \(\mf{a}\) of \(\smc{O}\).
    
    We can say more in the case of primes. Suppose \(\mf{P}\) and \(\mf{p}\) are primes of \(\mc{O}\) and \(\smc{O}\) respectively. We say that \(\mf{P}\) is \textbf{above}\index{above} \(\mf{p}\), or equivalently, \(\mf{p}\) is \textbf{below}\index{below} \(\mf{P}\) if
    \[
      \mf{P} \cap \smc{O} = \mf{p}.
    \]
    \cref{prop:Dedekind_extension_facts} implies that every prime of \(\mc{O}\) is above exactly one prime of \(\smc{O}\). If \(\mf{P}\) is above \(\mf{p}\) then \(\mf{P} \mid \mf{p}\mc{O}\). Indeed, since \(\mf{p} \subseteq \mf{P}\) we have \(\mf{p}\mc{O} \subseteq \mf{P}\) which is to say that \(\mf{P}\) divides \(\mf{p}\mc{O}\). This implies that only finitely many primes \(\mf{P}\) can lie above a prime \(\mf{p}\) and they are exactly the prime factors of \(\mf{p}\mc{O}\). Moreover, every prime \(\mf{p}\) satisfies
    \[
      \mf{p}\mc{O} \neq \mc{O}.
    \]
    Assume by contradiction that \(\mf{p}\mc{O} = \mc{O}\). By prime factorization, let \(\a \in \mf{p}-\mf{p}^{2}\) so that \(\a\smc{O} = \mf{a}\mf{p}\) for some integral ideal \(\mf{a}\) relatively prime to \(\mf{p}\). Then \(\mf{a}+\mf{p} = \smc{O}\) and so there exist \(\b \in \mf{a}\) and \(\g \in \mf{p}\) both nonzero and satisfying \(\b+\g = 1\). As \(\b\mf{p} \subseteq \a\smc{O}\), it follows that \(\b\mc{O} \subseteq \a\mc{O}\) which implies \(\b = \a\d\) for some nonzero \(\d \in \mc{O}\). In view of \cref{equ:Dedekind_extension_intersection}, \(\d \in \smc{O}\). Whence \(\a\d+\g = 1\) and thus \(1 \in \mf{p}\) contradicting that \(\mf{p}\) is proper. Therefore \(\mf{p}\mc{O}\) is always a integral ideal of \(\mc{O}\). In fact, we have
    \begin{equation}\label{equ:contraction_for_primes}
      \mf{p}\mc{O} \cap \smc{O} = \mf{p}.
    \end{equation}
    The reverse inclusion is obvious. For the forward inclusion, as \(\mf{p}\mc{O} \neq \mc{O}\) there exists a prime factor \(\mf{P}\) of \(\mf{p}\mc{O}\) so that \(\mf{p}\mc{O} \subseteq \mf{P}\). But then \(\mf{P}\) is above \(\mf{p}\) so that \(\mf{P} \cap \smc{O} = \mf{p}\) and the forward inclusion follows. Whence the primes above \(\mf{p}\) are exactly the prime factors \(\mf{P}\) of \(\mf{p}\mc{O}\).

    In fact, a more general identity holds. Let \(\mf{f}\) be a fractional ideal of \(\smc{O}\). Write \(\mf{f} = \frac{1}{\d}\mf{a}\) for some nonzero \(\d \in \smc{O}\) and integral ideal \(\mf{a}\). Expressing \(\mf{a}\) as its prime factorization and repeatedly applying \cref{equ:contraction_for_primes} shows that
    \[
      \mf{f}\mc{O} \cap \smc{O} = \mf{f}.
    \]
    This identity is called the \textbf{contraction}\index{contraction} of the fractional ideal \(\mf{f}\mc{O}\) to \(\smc{O}\). In practice, it is useful for relating information about the fractional ideal \(\mf{f}\mc{O}\) of \(\mc{O}\) to the fraction ideal \(\mf{f}\) of \(\smc{O}\). 




    The \textbf{ramification index}\index{ramification index} \(e_{\mf{p}}(\mf{P})\) of \(\mf{P}\) relative to \(\mf{p}\) is defined to be the power of \(\mf{P}\) appearing in the prime factorization of \(\mf{p}\mc{O}\). If \(\mf{p}\mc{O}\) has prime factors \(\mf{P}_{1},\ldots,\mf{P}_{r}\) then the prime factorization of \(\mf{p}\mc{O}\) is
    \[
      \mf{p}\mc{O} = \mf{P}_{1}^{e_{\mf{p}}(\mf{P}_{1})} \cdots \mf{P}_{r}^{e_{\mf{p}}(\mf{P}_{r})}.
    \]

    Something can also be said about the residue class fields \(\F_{\mf{P}}\) and \(\F_{\mf{p}}\). The former is a finite extension of the latter of degree at most \(n\) by \cref{prop:Dedekind_extension_facts}. Accordingly, we call \(\F_{\mf{P}}/\F_{\mf{p}}\) the \textbf{residue class extension}\index{residue class extension} of \(\mf{P}\) relative to \(\mf{p}\). We define the \textbf{inertia degree}\index{inertia degree} \(f_{\mf{p}}(\mf{P})\) of \(\mf{P}\) relative to \(\mf{p}\) by
    \[
      f_{\mf{p}}(\mf{P}) = [\F_{\mf{P}}:\F_{\mf{p}}].
    \]
    That is, \(f_{\mf{p}}(\mf{P})\) is the dimension of the residue field \(\F_{\mf{P}}\) as a \(\F_{\mf{p}}\)-vector space. More generally, recall from \cref{prop:Dedekind_extension_facts} that \(\mc{O}\) is at finitely generated \(\smc{O}\)-module of rank at most \(n\). Therefore the quotient ring \(\mf{B}/\mf{A}\) is a \(\F_{\mf{p}}\)-vector space of dimension at most \(n\) for any integral ideals \(\mf{A}\) and \(\mf{B}\) with \(\mf{p} \subseteq \mf{A} \subseteq \mf{B}\). We obtain the residue class field \(\F_{\mf{P}}\) when \(\mf{A} = \mf{P}\) and \(\mf{B} = \mc{O}\).

    Suppose \(\wtilde{\mc{O}}/\mc{O}/\smc{O}\) be a tower of Dedekind extensions for a finite separable tower \(M/L/K\). Let \(\wtilde{\mf{P}}\), \(\mf{P}\), and \(\mf{p}\) be primes of \(\wtilde{\mc{O}}\), \(\mc{O}\), and \(\smc{O}\) respectively with \(\wtilde{\mf{P}}\) above \(\mf{P}\) and \(\mf{P}\) above \(\mf{p}\). We illustrate this relationship via the diagram
    \begin{center}
      \begin{tikzcd}[row sep=large, column sep=large]
        \wtilde{\mf{P}} \subset \wtilde{\mc{O}} \subseteq M \arrow[dash]{d} \\
        \mf{P} \subset \mc{O} \subseteq L \arrow[dash]{d} \\
        \mf{p} \subset \smc{O} \subseteq K.
      \end{tikzcd}
    \end{center} 
    Then \(\mf{P}^{e_{\mf{p}}(\mf{P})} \mid\mid \mf{p}\mc{O}\) and \(\wtilde{\mf{P}}^{e_{\mf{P}}(\wtilde{\mf{P}})} \mid\mid \mf{P}\wtilde{\mc{O}}\). We also have the residue extensions \(\F_{\wtilde{\mf{P}}}/\F_{\mf{P}}\) and \(\F_{\mf{P}}/\F_{\mf{p}}\). It follows that
    \begin{equation}\label{equ:ramification_and_interia_towers}
      e_{\mf{p}}(\wtilde{\mf{P}}) = e_{\mf{P}}(\wtilde{\mf{P}})e_{\mf{p}}(\mf{P}) \quad \text{and} \quad f_{\mf{p}}(\wtilde{\mf{P}}) = f_{\mf{P}}(\wtilde{\mf{P}})f_{\mf{p}}(\mf{P}).
    \end{equation}
    In other words, the ramification indices and inertia degrees are multiplicative with respect to towers.

    The ramification indices and inertia degrees satisfy a simple relationship to the degree of \(L/K\) that is fundamental to they study of Dedekind extensions. In order to state it, we will require a lemma.

    \begin{lemma}\label{lem:quotient_by_prime_power_isomorphism}
      Let \(\mc{O}/\smc{O}\) be a Dedekind extension of a degree \(n\) separable extension \(L/K\). Suppose \(\mf{P}\) is a prime above \(\mf{p}\). Then for any positive integer \(e\), we have
      \[
        \mc{O}/\mf{P}^{e} \cong \bigop_{e}\F_{\mf{P}}.
      \]
      In particular, the dimension of \(\mc{O}/\mf{P}^{e}\) as a \(\F_{\mf{p}}\)-vector space is \(ef_{\mf{p}}(\mf{P})\).
    \end{lemma}
    \begin{proof}
      Consider the descending chain
      \[
        \mc{O}/\mf{P}^{e} \supseteq \mf{P}/\mf{P}^{e} \supseteq \cdots \supseteq \mf{P}^{e-1}/\mf{P}^{e} \supseteq \mf{P}^{e}/\mf{P}^{e},
      \]
      of \(\F_{\mf{p}}\)-vector spaces. By the third isomorphism theorem, these quotients are of the form \(\mf{P}^{i}/\mf{P}^{i+1}\) and are all isomorphic to \(\F_{\mf{P}}\) by \cref{lem:isomorphism_of_quotient_by_prime_integral_ideals}. Therefore we have a decomposition
      \[
        \mf{P}^{i}/\mf{P}^{e} \cong \F_{\mf{P}} \op (\mf{P}^{i+1}/\mf{P}^{e}).
      \]
      Iteratively applying this isomorphism \(e-1\) times gives
      \[
        \mc{O}/\mf{P}^{e} \cong \bigop_{e}\F_{\mf{P}}.
      \]
      This proves the first statement. Since the dimension of \(\F_{\mf{P}}\) as a \(\F_{\mf{p}}\)-vector space is \(f_{\mf{p}}(\mf{P})\) by definition, it follows that \(\mc{O}/\mf{P}^{e}\) is a \(\F_{\mf{p}}\)-vector space of dimension \(ef_{\mf{p}}(\mf{P})\). This proves the second statement.
    \end{proof}
    
    We now describe the relationship between ramification indices and inertia degrees which is known as the \textbf{fundamental equality}\index{fundamental equality}.

    \begin{theorem*}[Fundamental equality]
      Let \(\mc{O}/\smc{O}\) be a Dedekind extension of a degree \(n\) separable extension \(L/K\). Suppose \(\mf{p}\) is a prime of \(\smc{O}\) and \(\mf{p}\mc{O}\) has prime factorization
      \[
        \mf{p}\mc{O} = \mf{P}_{1}^{e_{\mf{p}}(\mf{P}_{1})} \cdots \mf{P}_{r}^{e_{\mf{p}}(\mf{P}_{r})}.
      \]
      Then
      \[
        n = \sum_{i}e_{\mf{p}}(\mf{P}_{i})f_{\mf{p}}(\mf{P}_{i}).
      \]
    \end{theorem*}
    \begin{proof}
      Since distinct primes are relatively prime, the Chinese remainder theorem gives an isomorphism
      \[
        \mc{O}/\mf{p}\mc{O} \cong \bigop_{i} \mc{O}/\mf{P}_{i}^{e_{\mf{p}}(\mf{P}_{i})}.
      \]
      As \(\mc{O}/\mf{p}\mc{O}\) and \(\mc{O}/\mf{P}_{i}^{e_{\mf{p}}(\mf{P}_{i})}\) are \(\F_{\mf{p}}\)-vector spaces, it suffices to show \(\mc{O}/\mf{p}\mc{O}\) is of dimension \(n\) and \(\mc{O}/\mf{P}_{i}^{e_{\mf{p}}(\mf{P}_{i})}\) is of dimension \(e_{\mf{p}}(\mf{P}_{i})f_{\mf{p}}(\mf{P}_{i})\). For \(\mc{O}/\mf{p}\mc{O}\), we already know it is a \(\F_{\mf{p}}\)-vector space of dimension at most \(n\) as \(\mc{O}\) is a finitely generated \(\smc{O}\)-module of rank at most \(n\) by \cref{prop:Dedekind_extension_facts}. Therefore we must show that the dimension is exactly \(n\). Let \(\conj{\l_{1}},\ldots,\conj{\l_{m}}\) be a basis for \(\mc{O}/\mf{p}\mc{O}\) as a \(\F_{\mf{p}}\)-vector space and let \(\l_{1},\ldots,\l_{m}\) be any lift of this basis to \(\mc{O}\). As \(m \le n\), it suffices to show \(\l_{1},\ldots,\l_{m}\) spans \(L/K\). Let \(M\) be the \(\smc{O}\)-module generated by \(\l_{1},\ldots,\l_{m}\) and set \(N = \mc{O}/M\). Then \(\mc{O} = M+\mf{p}\mc{O}\) since \(\l_{1},\ldots,\l_{m}\) is a lift of a basis for \((\mc{O}/\mf{p}\mc{O})/\F_{\mf{p}}\). Whence \(N = \mf{p}N\) and therefore \(N\) is a finitely generated \(\smc{O}\)-module of rank at most \(n\) since \(\mc{O}\) is by \cref{prop:Dedekind_extension_facts}. Let \(\w_{1},\ldots,\w_{r}\) be generators for \(N\). As \(N = \mf{p}N\), we have
      \[
        \w_{i} = \sum_{j}\a_{i,j}\w_{j},
      \]
      for some \(\a_{i,j} \in \mf{p}\). These \(r\) equations are equivalent to the identity
      \[
        \begin{pmatrix} 1-\a_{1,1} & \a_{1,2} & \cdots & -\a_{1,r} \\ -\a_{2,1} & 1-\a_{2,2} & & \\ \vdots & & \ddots & \\ -\a_{r,1} & & & 1-\a_{r,r} \end{pmatrix}\begin{pmatrix} \w_{1} \\ \w_{2} \\ \vdots \\ \w_{r} \end{pmatrix} = \mathbf{0}.
      \]
      Let \(d = \det(I-(\a_{i,j}))\). Then \(d \neq 0\) because expanding the determinant shows \(d \equiv 1 \tmod{\mf{p}}\) as \(\a_{i,j} \in \mf{p}\). Cramer's rule implies that \(d\) annihilates \(N\) which is to say that \(d\mc{O} \subseteq M\). Equivalently,
      \[
        d\mc{O} \subseteq \l_{1}\smc{O}+\cdots+\l_{m}\smc{O}.
      \]
      By \cref{prop:field_of_fractions_AKBL}, multiplication by \(K\) shows
      \[
        L = \l_{1}K+\cdots+\l_{m}K.
      \]
      Hence \(\l_{1},\ldots,\l_{m}\) spans \(L/K\) whence \(\mc{O}/\mf{p}\mc{O}\) is a \(\F_{\mf{p}}\)-vector space of dimension \(n\). For \(\mc{O}/\mf{P}_{i}^{e_{\mf{p}}(\mf{P}_{i})}\), the dimensionality claim follows from \cref{lem:quotient_by_prime_power_isomorphism}. So our dimension computations combine to give
      \[
        n = \sum_{i}e_{\mf{p}}(\mf{P}_{i})f_{\mf{p}}(\mf{P}_{i}).
      \]
    \end{proof}

    It is useful to classify primes according to specific cases of the fundamental equality. We say \(\mf{p}\) is \textbf{inert}\index{inert} in \(\mc{O}/\smc{O}\) if there is a prime \(\mf{P}\) above \(\mf{p}\) with \(f_{\mf{p}}(\mf{P}) = n\). By the fundamental equality, this means \(e_{\mf{p}}(\mf{P}) = 1\) and \(r = 1\). Whence
    \[
      \mf{p}\mc{O} = \mf{P},
    \]
    which is to say \(\mf{p}\mc{O}\) is prime. More generally, \(\mf{p}\) is said to be \textbf{nonsplit}\index{nonsplit} in \(\mc{O}/\smc{O}\) if there is a prime \(\mf{P}\) above \(\mf{p}\) satisfying \(e_{\mf{p}}(\mf{P})f_{\mf{p}}(\mf{P}) = n\) and is said to be \textbf{split}\index{split} in \(\mc{O}/\smc{O}\) otherwise. Therefore,
    \[
      \mf{p}\mc{O} = \mf{P}^{e_{\mf{p}}(\mf{P})} \quad \text{or} \quad \mf{p}\mc{O} = \mf{P}_{1}^{e_{\mf{p}}(\mf{P}_{1})} \cdots \mf{P}_{r}^{e_{\mf{p}}(\mf{P}_{r})},
    \]
    with \(r\) at least \(2\), according to if \(\mf{p}\) is nonsplit or split accordingly. Moreover, we say \(\mf{p}\) is \textbf{totally split}\index{totally split} in \(\mc{O}/\smc{O}\) if every prime \(\mf{P}\) above \(\mf{p}\) satisfies \(e_{\mf{p}}(\mf{P}) = f_{\mf{p}}(\mf{P}) = 1\) and so \(r = n\) by the fundamental equality. Then
    \[
      \mf{p}\mc{O} = \mf{P}_{1} \cdots \mf{P}_{n}.
    \]
    In terms of inertia degrees, being inert or totally split in \(\mc{O}/\smc{O}\) are antithetical properties. In particular, the smaller the inertia degrees are the greater the tendency for \(\mf{p}\mc{O}\) to factor into distinct primes.
    
    Now let us introduce ramification of primes. If \(\mf{P}\) is a prime of \(\mc{O}\) above \(\mf{p}\), we say \(\mf{P}\) is \textbf{unramified}\index{unramified} in \(\mc{O}/\smc{O}\) if \(e_{\mf{p}}(\mf{P}) = 1\) and the residue class extension \(\F_{\mf{P}}/\F_{\mf{p}}\) is separable. Otherwise, we say \(\mf{P}\) is \textbf{ramified}\index{ramified} in \(\mc{O}/\smc{O}\). Moreover, \(\mf{P}\) is \textbf{totally ramified}\index{totally ramified} in \(\mc{O}/\smc{O}\) if in addition to being ramified we have \(e_{\mf{p}}(\mf{P}) = n\) whence \(f_{\mf{p}}(\mf{P}) = 1\). Similarly, we say that a prime \(\mf{p}\) of \(\smc{O}\) is \textbf{unramified}\index{unramified} in \(\mc{O}/\smc{O}\) if every prime \(\mf{P}\) above it is unramified and is \textbf{ramified}\index{ramified} in \(\mc{O}/\smc{O}\) otherwise. Also, \(\mf{p}\) is said to be \textbf{totally ramified}\index{totally ramified} in \(\mc{O}/\smc{O}\) if a prime above it is totally ramified. In this case, the fundamental equality implies that there is only one prime \(\mf{P}\) above \(\mf{p}\) and
    \[
      \mf{p}\mc{O} = \mf{P}^{n}.
    \]

    We now turn to the setting where \(L/K\) is an extension of number fields. Then \(L/K\) is a finite separable extension and \(\mc{O}_{L}/\mc{O}_{K}\) is a Dedekind extension for \(L/K\). Then every residue class extension is separable as \(\F_{p}\) is perfect. A prime \(\mf{p}\) of \(K\) is said to be \textbf{inert}\index{inert}, \textbf{nonsplit}\index{nonsplit}, \textbf{split}\index{split}, or \textbf{totally split}\index{totally split} in \(L/K\) if it is in \(\mc{O}_{L}/\mc{O}_{K}\). Primes \(\mf{P}\) and \(\mf{p}\) of \(L\) and \(K\) are said to be \textbf{unramified}\index{unramified}, \textbf{ramified}\index{ramified}, or \textbf{totally ramified}\index{totally ramified} in \(L/K\) if they are in \(\mc{O}_{L}/\mc{O}_{K}\). In particular, a prime \(\mf{P}\) above \(\mf{p}\) is unramified in \(L/K\) if and only if \(e_{\mf{p}}(\mf{P}) = 1\).
    
    \iffalse
    The Dedekind extension \(\mc{O}/\smc{O}\) itself is said to be \textbf{unramified}\index{unramified} if every prime of \(\smc{O}\) is unramified in \(\mc{O}/\smc{O}\) and is said to be \textbf{ramified}\index{ramified} otherwise.
    \fi

    \iffalse
    It is desirable to understand how \(\mf{p}\mc{O}\) factors into primes. This is tractable for most primes if we know a sufficient amount of information about the minimal polynomial of certain primitive elements for \(L/K\). Let \(\t\) be a primitive element for \(L/K\) so that \(L = K(\t)\). In view of \cref{prop:field_of_fractions_AKBL} we can multiply by a nonzero element of \(\smc{O}\), if necessary, to ensure \(\t \in \mc{O}\) and hence its minimal polynomial \(m_{\t}(x)\) over \(K\) has coefficients in \(\smc{O}\). For such a primitive element \(\t\), we define the \textbf{conductor}\index{conductor} \(\mf{Q}_{\mc{O}/\smc{O}}(\t)\) of \(\mc{O}/\smc{O}\) relative to \(\t\) by
    \[
      \mf{Q}_{\mc{O}/\smc{O}}(\t) = \{\a \in \mc{O}:\a\mc{O} \subseteq \smc{O}[\t]\}.
    \]
    This is an integral ideal of \(\mc{O}\) provided it is nonzero. To see that it is nonzero, recall that \(\mc{O}\) is a finitely generated \(\smc{O}\) module by \cref{prop:Dedekind_extension_facts}. Let \(\w_{1},\ldots,\w_{r}\) be generators. As \(L/K\) is algebraic, \(\t\) is algebraic over \(K\) and hence \(L = K[\t]\). Then
    \[
      \w_{i} = \sum_{j}\k_{i,j}\t^{j},
    \]
    with \(\k_{i,j} \in K\). As \(K\) is the field of fractions of \(\smc{O}\), \(\k_{i,j} = \frac{\a_{i,j}}{\d_{i,j}}\) with \(\a_{i,j},\d_{i,j} \in \smc{O}\) and \(\d_{i,j}\) nonzero. Setting \(\d = \prod_{i,j}\d_{i,j}\), we find that \(\d \in \smc{O}\) is nonzero and satisfies \(\d\w_{i} \in \smc{O}[\t]\). As \(\w_{1},\ldots,\w_{r}\) generate \(\mc{O}\) as a \(\smc{O}\)-module, we conclude \(\d \in \mf{Q}_{\mc{O}/\smc{O}}(\t)\). Therefore \(\mf{Q}_{\mc{O}/\smc{O}}(\t)\) is nonzero and hence an integral ideal of \(\mc{O}\). The \textbf{Dedekind-Kummer theorem}\index{Dedekind-Kummer theorem} describes the factorization of \(\mf{p}\mc{O}\) provided it is relatively prime to the conductor \(\mf{Q}_{\mc{O}/\smc{O}}(\t)\):

    \begin{theorem*}[Dedekind-Kummer theorem]
      Let \(\mc{O}/\smc{O}\) be a Dedekind extension of a degree \(n\) separable extension \(L/K\), \(\t\) be a primitive element of \(L/K\) contained in \(\mc{O}\) with minimal polynomial \(m_{\t}(x)\) over \(K\), and \(\mf{p}\) be a prime of \(\smc{O}\) such that \(\mf{p}\mc{O}\) is relatively prime to \(\mf{Q}_{\mc{O}/\smc{O}}\). Suppose
      \[
        \conj{m_{\t}}(x) = \conj{m_{1}}(x)^{e_{1}} \cdots \conj{m_{r}}(x)^{e_{r}},
      \]
      is the prime factorization of \(\conj{m_{\t}}(x)\) in \(\F_{\mf{p}}[x]\). Let \(m_{i}(x)\) be any lift of \(\conj{m_{i}}(x)\) to \(\smc{O}[x]\) and set
      \[
        \mf{P}_{i} = \mf{p}\mc{O}+m_{i}(\t)\mc{O},
      \]
      for \(1 \le i \le r\). Then \(\mf{P}_{i}\) is a prime of \(\mc{O}\) for all \(i\),
      \[
        \mf{p}\mc{O} = \mf{P}_{1}^{e_{1}} \cdots \mf{P}_{r}^{e_{r}},
      \]
      is the prime factorization of \(\mf{p}\mc{O}\), and \(f_{\mf{p}}(\mf{P}_{i}) = \deg(\conj{m_{i}}(x))\) for all \(i\).
    \end{theorem*}
    \begin{proof}
      First consider the homomorphism
      \[
        \phi:\smc{O}[\t] \to \mc{O}/\mf{p}\mc{O} \qquad \a \mapsto \a+\mf{p}\mc{O}.
      \]
      We have \(\mf{p}\mc{O}+\mf{Q}_{\mc{O}/\smc{O}} = \mc{O}\) because \(\mf{p}\mc{O}\) is relatively prime to \(\mf{Q}_{\mc{O}/\smc{O}}\). As \(\mf{Q}_{\mc{O}/\smc{O}} \subseteq \smc{O}[\t]\), it follows that \(\mf{p}\mc{O}+\smc{O}[\t] = \mc{O}\) which shows \(\phi\) is surjective. Now \(\ker\phi = \smc{O}[\t] \cap \mf{p}\mc{O}\) and we claim \(\smc{O}[\t] \cap \mf{p}\mc{O} = \mf{p}\smc{O}[\t]\). The reverse inclusion is clear since \(\mf{p}\) is an integral ideal of \(\smc{O}\). For the forward inclusion, intersecting both sides of \(\mf{p}\mc{O}+\mf{Q}_{\mc{O}/\smc{O}} = \mc{O}\) with \(\smc{O}\) gives \(\mf{p} \cap \mf{Q}_{\mc{O}/\smc{O}} = \smc{O}\) because \(\mf{p}\mc{O} \cap \smc{O} = \mf{p}\). Hence
      \[
        \smc{O}[\t] \cap \mf{p}\mc{O} = (\mf{p} \cap \mf{Q}_{\mc{O}/\smc{O}})(\smc{O}[\t] \cap \mf{p}\mc{O}) = (\mf{p}\smc{O}[\t] \cap \mf{p}\mc{O})+(\mf{Q}_{\mc{O}/\smc{O}}\smc{O}[\t] \cap \mf{Q}_{\mc{O}/\smc{O}}\mf{p}\mc{O}) \subseteq \mf{p}\smc{O}[\t],
      \]
      where the inclusion follows by the definition of the conductor \(\mf{Q}_{\mc{O}/\smc{O}}\). This proves the reverse containment so that \(\ker\phi = \mf{p}\smc{O}[\t]\). By first isomorphism theorem, we obtain
      \[
        \smc{O}[\t]/\mf{p}\smc{O}[\t] \cong \mc{O}/\mf{p}\mc{O}.
      \]
      Since \(m_{\t}(x)\) is the minimal polynomial for \(\t\) over \(K\), we have an isomorphism \(\smc{O}[x]/m_{\t}(x)\smc{O}[x] \cong \smc{O}[\t]\) given by evaluation at \(\t\). Then we have the chain of isomorphism
      \[
        \smc{O}[\t]/\mf{p}\smc{O}[\t] \cong (\smc{O}[x]/m_{\t}(x)\smc{O}[x])/(\mf{p}(\smc{O}[x]/m_{\t}(x)\smc{O}[x])) \cong \smc{O}[x]/(\mf{p}\smc{O}[x]+m_{\t}(x)\smc{O}[x]) \cong \F_{\mf{p}}[x]/\conj{m_{\t}}(x)\F_{\mf{p}}[x],
      \]
      where the second and third isomorphisms follow by taking \(\smc{O}[x]/(\mf{p}\smc{O}[x]+m_{\t}(x)\smc{O}[x])\) and reducing elements of \(\smc{O}[x]\) modulo \(m_{\t}(x)\smc{O}[x]\) or their coefficients modulo \(\mf{p}\) respectively. Therefore the inverse isomorphism is given by sending any representative \(\conj{f}(x)\) of a coset in \(\F_{\mf{p}}[x]/\conj{m_{\t}}(x)\F_{\mf{p}}[x]\) to a lift \(f(x)\) in \(\smc{O}[x]\) and then to \(\conj{f(\t)}\) by reducing \(f(\t)\) modulo \(\mf{p}\mc{O}\). Now set \(A = \F_{\mf{p}}[x]/\conj{m_{\t}}(x)\F_{\mf{p}}[x]\). The Chinese remainder theorem gives an isomorphism
      \[
        A \cong \bigop_{1 \le i \le r}\F_{\mf{p}}[x]/\conj{m_{i}}(x)^{e_{i}}\F_{\mf{p}}[x].
      \]
      As \(\conj{m_{i}}(x)\) is irreducible, \(\conj{m_{i}}(x)\F_{\mf{p}}[x]\) is maximal and hence \(\F_{\mf{p}}[x]/\conj{m_{i}}(x)\F_{\mf{p}}[x]\) is a field. By the third isomorphism theorem, \(\conj{m_{i}}(x)\F_{\mf{p}}[x]/\conj{m_{i}}(x)^{e_{i}}\F_{\mf{p}}[x]\) is a maximal ideal of \(\F_{\mf{p}}[x]/\conj{m_{i}}(x)^{e_{i}}\F_{\mf{p}}[x]\). It follows that the maximal ideals of \(\smc{O}\) are precisely the \(\conj{m_{i}}(x)A\) and we have an isomorphism
      \[
        A/\conj{m_{i}}(x)A \cong \F_{\mf{p}}[x]/\conj{m_{i}}(x)\F_{\mf{p}}[x],
      \]
      for all \(i\). Via the isomorphism \(\mc{O}/\mf{p}\mc{O} \cong A\) described above, the maximal ideals of \(\mc{O}/\mf{p}\mc{O}\) are exactly \(\conj{m_{i}(\t)}(\mc{O}/\mf{p}\mc{O})\). We now show that the \(\mf{P}_{i}\) are prime. To see this, consider the surjective homomorphism
      \[
        \pi:\mc{O} \to \mc{O}/\mf{p}\mc{O} \qquad \a \mapsto \a+\mf{p}\mc{O}.
      \]
      Then the image of \(\mf{P}_{i}\) under \(\pi\) is \(\conj{m_{i}(\t)}(\mc{O}/\mf{p}\mc{O})\). As this ideal is maximal and hence prime, the preimage \(\mf{P}_{i}\) is prime too. Moreover, the \(\mf{P}_{i}\) are all distinct since the \(\conj{m_{i}(\t)}\mc{O}/\mf{p}\mc{O}\) are which are all distinct because the \(\conj{m_{i}}(x)A\) are (using the isomorphism \(\mc{O}/\mf{p}\mc{O} \cong A\)). In particular, they are also relatively prime. By construction, \(\mf{P}_{i} \subseteq \mf{p}\mc{O}\) so that the \(\mf{P}_{i}\) are prime factors of \(\mf{p}\mc{O}\). These are the only prime factors of \(\mf{p}\mc{O}\) because the image of any prime under \(\pi\) contained in \(\mf{p}\mc{O}\) must be a maximal ideal of \(\mc{O}/\mf{p}\mc{O}\) (since primes are maximal and by the fourth isomorphism theorem) and every maximal ideal is one of the \(\conj{m_{i}(\t)}(\mc{O}/\mf{p}\mc{O})\). Together, all of this means that \(\mf{p}\mc{O}\) admits the prime factorization
      \[
        \mf{p}\mc{O} = \mf{P}_{1}^{e_{\mf{p}}(\mf{P}_{1})} \cdots \mf{P}_{r}^{e_{\mf{p}}(\mf{P}_{r})},
      \]
      for some ramification indices \(e_{\mf{p}}(\mf{P}_{i})\). We will be done if we can show \(e_{\mf{p}}(\mf{P}_{i}) = e_{i}\) and \(f_{\mf{p}}(\mf{P}_{i}) = \deg(\conj{m_{i}}(x))\) for all \(i\). To accomplish this, observe that we have an isomorphism
      \[
        \F_{\mf{P}_{i}} \cong (\mc{O}/\mf{p}\mc{O})/(\conj{m_{i}(\t)}(\mc{O}/\mf{p}\mc{O})) \cong \F_{\mf{p}}[x]/\conj{m_{i}}(x)\F_{\mf{p}}[x],
      \]
      where the first isomorphism follow by taking \(\F_{\mf{P}_{i}}\) and reducing \(\mc{O}\) modulo \(\mf{p}\) and the second isomorphism follows from \(\mc{O}/\mf{p}\mc{O} \cong A\) and that the image of the maximal ideal \(\conj{m_{i}(\t)}(\mc{O}/\mf{p}\mc{O})\) under this isomorphism is \(\conj{m_{i}}(x)A\). Now \(\F_{\mf{p}}[x]/\conj{m_{i}}(x)\F_{\mf{p}}[x]\) is a \(\F_{\mf{p}}\)-vector space of degree \(\deg(\conj{m_{i}}(x))\). Hence \(f_{p}(\mf{P}_{i}) = \deg(\conj{m_{i}}(x))\) for all \(i\) as desired. The ideal \(\conj{m_{i}}(x)^{e_{i}}A\) under the isomorphism \(A \cong \mc{O}/\mf{p}\mc{O}\) is the ideal \(\conj{m_{i}(\t)}^{e_{i}}(\mc{O}/\mf{p}\mc{O})\). As the image of \(\mf{P}_{i}\) under \(\pi\) is \(\conj{m_{i}(\t)}(\mc{O}/\mf{p}\mc{O})\), we have that \(\mf{P}_{i}^{e_{i}}\) is contained in the preimage of \(\conj{m_{i}(\t)}^{e_{i}}(\mc{O}/\mf{p}\mc{O})\) under \(\pi\). As \(\conj{m_{\t}(\t)}(\mc{O}/\mf{p}\mc{O}) = 0\), it follows that
      \[
        \mf{p}\mc{O} = \pi^{-1}(0) \supseteq \mf{p}_{1}^{e_{1}} \cdots \mf{p}_{r}^{e_{r}}.
      \]
      Since the \(\mf{P}_{i}\) are prime, we have \(e_{\mf{p}}(\mf{P}_{i}) \le e_{i}\) for all \(i\). But the fundamental equality then gives
      \[
        n = \sum_{1 \le i \le r}e_{\mf{p}}(\mf{P}_{i})f_{p}(\mf{P}_{i}) \le \sum_{1 \le i \le r}e_{i}f_{p}(\mf{P}_{i}) \le \sum_{1 \le i \le r}e_{i}\deg(\conj{m_{i}}(x)) \le n,
      \]
      where the last equality follows by the prime factorization of \(\conj{m_{\t}}(x)\) and that \(\deg(\conj{m_{\t}}(x)) = \deg(m_{\t}(x))\) because \(m_{\t}(x)\) is monic. This shows \(e_{\mf{p}}(\mf{P}_{i}) = e_{i}\) for all \(i\) which completes the proof.
    \end{proof}

    Let \(\mc{O}/\smc{O}\) be a Dedekind extension of a degree \(n\) separable extension \(L/K\). The Dedekind-Kummer theorem allows us to compute the prime factorization of \(\mf{p}\mc{O}\) provided this integral ideal is relatively prime to the conductor \(\mf{Q}_{\mc{O}/\smc{O}}\). By the prime factorization of fractional ideals, \(\mf{Q}_{\mc{O}/\smc{O}}\) has finitely many prime factors so we only have to avoid finitely many primes of \(\mc{O}\). In fact, if the conductor is \(\mc{O}\) then we do not have to avoid any primes at all. This occurs when \(\mc{O}/\smc{O}\) is monogenic. Indeed, Suppose \(\mc{O} = \smc{O}[\a]\) for some \(\a \in \mc{O}\). Then \(1,\a,\ldots,\a^{n-1}\) is an integral basis for \(\mc{O}/\smc{O}\) and is necessarily a basis for \(L/K\). But then \(\a\) is also a primitive element for \(L/K\) which implies
    \[
      \mf{Q}_{\mc{O}/\smc{O}} = \mc{O}.
    \]
    \fi
  \section{Ramification}
    Much more can be said about the ramification of primes in a Dedekind extension \(\mc{O}/\smc{O}\) when the associated extension \(L/K\) is Galois. Since \(L/K\) is assumed to be finite and separable, this amounts to further assuming \(L/K\) is normal. In this case, the Galois group is
    \[
      \Gal(L/K) = \Hom_{K}(L,\conj{K}),
    \]
    so that every \(K\)-embedding of \(L\) into \(\conj{K}\) is an automorphism of \(L\). By \cref{prop:formulas_for_trace_and_norm}, the norm and trace of any \(\l \in L\) are given by
    \[
      \Norm_{L/K}(\l) = \prod_{\s \in \Gal(L/K)}\s(\l) \quad \text{and} \quad \Trace_{L/K}(\l) = \sum_{\s \in \Gal(L/K)}\s(\l).
    \]
    As \(\smc{O} \subseteq K\), \(\Gal(L/K)\) fixes \(\smc{O}\) pointwise and hence every fractional ideal of \(\smc{O}\) as well. In fact, for any \(\a \in \mc{O}\), we see that \(\Gal(L/K)\) permutes the roots of the monic polynomial in \(\smc{O}[x]\) which \(\a\) satisfies. Whence there is an action
    \[
      \Gal(L/K) \x \mc{O} \to \mc{O} \qquad (\s,\a) \mapsto \s(\a).
    \]
    In particular, each \(K\)-automorphism \(\s\) restricts to an automorphism of \(\mc{O}\) and therefore \(\s(\mf{P})\) is a prime of \(\mc{O}\) if \(\mf{P}\) is. Moreover, if \(\mf{P}\) is above \(\mf{p}\) then so is \(\s(\mf{P})\) as it is a prime containing \(\mf{p}\). Whence
    \[
      \s(\mf{P}) \cap \smc{O} = \mf{p}.
    \]
    Accordingly, we say that \(\s(\mf{P})\) is \textbf{conjugate}\index{conjugate} to \(\mf{P}\). It turns out that the Galois group acts transitively on the primes above a given prime.

    \begin{proposition}\label{prop:Galois_action_on_primes_is_transitive}
      Let \(\mc{O}/\smc{O}\) be a Dedekind extension of a finite Galois extension \(L/K\) and let \(\mf{p}\) be a prime of \(\smc{O}\). Then \(\Gal(L/K)\) acts transitively on the primes above \(\mf{p}\). Moreover, if \(\mf{p}\mc{O}\) has prime factorization
      \[
        \mf{p}\mc{O} = \mf{P}_{1}^{e_{\mf{p}}(\mf{P}_{1})} \cdots \mf{P}_{r}^{e_{\mf{p}}(\mf{P}_{r})},
      \]
      then
      \[
        f_{\mf{p}}(\mf{P}_{1}) = \cdots = f_{\mf{p}}(\mf{P}_{r}) \quad \text{and} \quad e_{\mf{p}}(\mf{P}_{1}) = \cdots = e_{\mf{p}}(\mf{P}_{r}).
      \]
    \end{proposition}
    \begin{proof}
      Assume by contradiction that there exist distinct primes \(\mf{P}_{i}\) and \(\mf{P}_{j}\) above \(\mf{p}\) for which \(\mf{P}_{i} \neq \mf{P}_{j}\) for any \(K\)-automorphism \(\s\). As distinct primes are relatively prime, the Chinese remainder theorem implies the existence of an \(\a \in \mc{O}\) such that
      \[
        \a \equiv 1 \tmod{\s(\mf{P}_{i})} \quad \text{and} \quad \a \equiv 0 \tmod{\mf{P}_{j}},
      \]
      Now recall As \(\Norm_{L/K}(\a) \in \smc{O}\) by \cref{prop:trace_and_norm_AKBL}. On the one hand, \(\a \notin \s(\mf{P}_{i})\) whence \(\s(\a) \notin \mf{P}_{i}\). Then \(\Norm_{L/K}(\a) \notin \mf{P}_{i}\) by primality of \(\mf{P}_{i}\) and hence \(\Norm_{L/K}(\a) \notin \mf{p}\) because \(\mf{P}_{i}\) is above \(\mf{p}\). On the other hand, \(\a \in \mf{P}_{j}\) which implies \(\a \in \s(\mf{P}_{j})\) when \(\s\) is the identity. Therefore \(\Norm_{L/K}(\a) \in \mf{P}_{j}\) and so \(\Norm_{L/K}(\a) \in \mf{p}\) because \(\mf{P}_{j}\) is above \(\mf{p}\). This gives a contradiction from which we conclude the action is transitive. 
      
      It remains to show that the ramification indices and inertia degrees of \(\mf{P}_{i}\) and \(\mf{P}_{j}\) are equal. As the action is transitive, there exists a \(K\)-automorphism \(\s\) satisfying \(\s(\mf{P}_{i}) = \mf{P}_{j}\). Now \(\s\) is an automorphism of \(\mc{O}\) fixing \(\smc{O}\), and hence \(\mf{p}\), pointwise. Thus \(\s(\mf{p}\mc{O}) = \mf{p}\mc{O}\) which implies \(\mf{P}_{i}^{e} \mid\mid \mf{p}\mc{O}\) if and only if \(\mf{P}_{j}^{e} \mid\mid \mf{p}\mc{O}\). Therefore the ramification indices of \(\mf{P}_{i}\) and \(\mf{P}_{j}\) are equal. Also, being an automorphism of \(\mc{O}\), \(\s\) induces an isomorphism
      \[
        \F_{\mf{P}_{i}} \cong \F_{\mf{P}_{j}}.
      \]
      Therefore the inertia degrees of \(\mf{P}_{i}\) and \(\mf{P}_{j}\) are equal.
    \end{proof}

    Another way to phrase this result is that the primes above \(\mf{p}\) are all conjugate to each other and their ramification indices and inertia degrees are all equal. We call the common ramification index \(e\) the \textbf{ramification index}\index{ramification index} of \(\mf{p}\) in \(\mc{O}/\smc{O}\) and the common inertia degree \(f\) the \textbf{inertia degree}\index{inertia degree} of \(\mf{p}\) in \(\mc{O}/\smc{O}\). If there are \(r\) primes above \(\mf{p}\), the fundamental equality takes the particularly simple form
    \[
      n = ref.
    \]

    We define the \textbf{decomposition group}\index{decomposition group} \(D_{\mf{p}}(\mf{P})\) of \(\mf{P}\) over \(\mf{p}\) is defined to by
    \[
      D_{\mf{p}}(\mf{P}) = \{\tau \in \Gal(L/K):\tau(\mf{P}) = \mf{P}\}.
    \]
    Equivalently, the decomposition group is the stabilizer subgroup of \(\mf{P}\) in the Galois group. The associated \textbf{decomposition field}\index{decomposition field} \(L^{D_{\mf{p}}(\mf{P})}\) of \(\mf{P}\) over \(\mf{p}\) is defined by
    \[
      L^{D_{\mf{p}}(\mf{P})} = \{\l \in L:\tau(\l) = \l \text{ for all } \tau \in D_{\mf{p}}(\mf{P})\}.
    \]
    In other words, the decomposition field is the fixed field of \(L\) by the decomposition group and hence an intermediate field of \(L/K\). In this case, the fundamental theorem of Galois theory says
    \[
      \Gal(L/L^{D_{\mf{p}}(\mf{P})}) = D_{\mf{p}}(\mf{P}).
    \]
    Let \(\mc{O}^{D_{\mf{p}}(\mf{P})}\) be the integral closure of \(\smc{O}\) in \(L^{D_{\mf{p}}(\mf{P})}\). Then \cref{prop:integral_closure_is_transitive,prop:integral_closure_of_Dedekind_is_Dedekind} together imply that \(\mc{O}/\mc{O}^{D_{\mf{p}}(\mf{P})}/\smc{O}\) is a tower of Dedekind extensions for the finite separable tower \(L/L^{D_{\mf{p}}(\mf{P})}/K\). Also, write \(\mf{P}^{D}\) for the prime of \(L^{D_{\mf{p}}(\mf{P})}\) below \(\mf{P}\). We illustrate this relationship via the diagram
    \begin{center}
      \begin{tikzcd}[row sep=large, column sep=large]
        \mf{P} \subset \mc{O} \subseteq L \arrow[dash]{d} \\
        \mf{P}^{D} \subset \mc{O}^{D_{\mf{p}}(\mf{P})} \subseteq L^{D_{\mf{p}}(\mf{P})} \arrow[dash]{d} \\
        \mf{p} \subset \smc{O} \subseteq K.
      \end{tikzcd}
    \end{center}
    The decomposition group encodes how \(\mf{p}\mc{O}\) splits into distinct prime factors. Indeed, by the orbit-stabilizer theorem the number of cosets in \(\Gal(L/K)/D_{\mf{p}}(\mf{P})\) is equal to the size of the orbit of \(\mf{P}\) under the action of \(\Gal(L/K)\). As the action is transitive by \cref{prop:Galois_action_on_primes_is_transitive}, \(\s(\mf{P})\) runs over the primes above \(\mf{p}\) as \(\s\) runs over a complete set of representatives for this coset space. Whence
    \[
      \mf{p}\mc{O} = \left(\prod_{\s \in \Gal(L/K)/D_{\mf{p}}(\mf{P})}\s(\mf{P})\right)^{e}.
    \]
    Then the order of \(\Gal(L/K)/D_{\mf{p}}(\mf{P})\) is exactly the number of prime factors of \(\mf{p}\mc{O}\). That is,
    \[
      |\Gal(L/K)/D_{\mf{p}}(\mf{P})| = r.
    \]
    In particular, \(\mf{p}\) is inert in \(\mc{O}/\smc{O}\) if and only if \(D_{\mf{p}}(\mf{P}) = \Gal(L/K)\) which is equivalent to \(L^{D_{\mf{p}}(\mf{P})} = K\). Antithetically, \(\mf{p}\) is totally split in \(\mc{O}/\smc{O}\) if and only if \(D_{\mf{p}}(\mf{P}) = \{1\}\) which is equivalent to \(L^{D_{\mf{p}}(\mf{P})} = L\). More generally, the previous identity and fundamental equality together imply
    \begin{equation}\label{equ:decomposition_group_e_f_relation}
      |D_{\mf{p}}(\mf{P})| = ef.
    \end{equation}
    Therefore \(L/L^{D_{\mf{p}}(\mf{P})}\) has degree \(ef\) and \(L^{D_{\mf{p}}(\mf{P})}/K\) has degree \(r\). If \(\mf{p}\) inert in \(\mc{O}/\smc{O}\) then \(n = ef\) implying \(e = 1\) and \(f = n\) as we have seen. If \(\mf{p}\) is totally split in \(\mc{O}/\smc{O}\) then \(1 = ef\) so that \(e = f = 1\) as we have also seen. Moreover, the decomposition group of a conjugate prime is the conjugate of decomposition group. More precisely, we have
    \[
      D_{\mf{p}}(\s(\mf{P})) = \s D_{\mf{p}}(\mf{P})\s^{-1}.
    \]
    This is simply because \(\tau(\mf{P}) = \mf{P}\) if and only if \(\s\tau\s^{-1}(\s(\mf{P})) = \s(\mf{P})\). Also, the fundamental theorem of Galois theory implies that \(D_{\mf{p}}(\mf{P})\) is normal if and only if \(L^{D_{\mf{p}}(\mf{P})}/K\) is Galois in which case
    \[
      \Gal(L^{D_{\mf{p}}(\mf{P})}/K) \cong \Gal(L/K)/D_{\mf{p}}(\mf{P}).
    \]
    Regardless, the decomposition field contains all of the information about the prime factors that \(\mf{p}\mc{O}\) splits into. In particular, \(\mf{P}^{D}\) is nonsplit in \(\mc{O}/\mc{O}^{D_{\mf{p}}(\mf{P})}\) and \(\mf{p}\) is totally split in \(\mc{O}^{D_{\mf{p}}(\mf{P})}/\smc{O}\).

    \begin{proposition}\label{prop:decomposition_field_e_and_f}
      Let \(\mc{O}/\smc{O}\) be a Dedekind extension of a degree \(n\) Galois extension \(L/K\), let \(\mf{p}\) be a prime of \(\smc{O}\) with inertia degree \(f\) and ramification index \(e\), and let \(\mf{P}\) be a prime of \(\mc{O}\) above \(\mf{p}\). Then \(\mf{P}^{D}\) is nonsplit in \(\mc{O}/\mc{O}^{D_{\mf{p}}(\mf{P})}\) and \(\mf{p}\) is totally split in \(\mc{O}^{D_{\mf{p}}(\mf{P})}/\smc{O}\). Moreover,
      \[
        e_{\mf{P}^{D}}(\mf{P}) = e \quad \text{and} \quad f_{\mf{P}^{D}}(\mf{P}) = f.
      \]
    \end{proposition}
    \begin{proof}
      Recall \(\Gal(L/L^{D_{\mf{p}}(\mf{P})}) = D_{\mf{p}}(\mf{P})\). It follows from \cref{prop:Galois_action_on_primes_is_transitive} that the primes of \(\mc{O}\) above \(\mf{P}^{D}\) are of the form \(\tau(\mf{P})\) for \(\tau \in D_{\mf{p}}(\mf{P})\). These \(\tau\) leave \(\mf{P}\) invariant which means \(\mf{P}^{D}\) is nonsplit in \(\mc{O}/\mc{O}^{D_{\mf{p}}(\mf{P})}\). In view of this fact and that \(L/L^{D_{\mf{p}}(\mf{P})}\) has degree \(ef\), the fundamental equality gives
      \[
        ef = e_{\mf{P}^{D}}(\mf{P})f_{\mf{P}^{D}}(\mf{P}).
      \]
      As \(\mc{O}/\mc{O}^{D_{\mf{p}}(\mf{P})}/\smc{O}\) is a tower of Dedekind extensions, \cref{equ:ramification_and_interia_towers} implies
      \[
        e = e_{\mf{P}^{D}}(\mf{P})e_{\mf{p}}(\mf{P}^{D}) \quad \text{and} \quad f = f_{\mf{P}^{D}}(\mf{P})f_{\mf{p}}(\mf{P}^{D}).
      \]
      Combining all of these identities results in
      \[
        1 = e_{\mf{p}}(\mf{P}^{D})f_{\mf{p}}(\mf{P}^{D}).
      \]
      Whence \(e_{\mf{p}}(\mf{P}^{D}) = f_{\mf{p}}(\mf{P}^{D}) = 1\). As \(\mf{P}^{D}\) is above \(\mf{p}\), \cref{prop:Galois_action_on_primes_is_transitive} implies that \(\mf{p}\) is totally split in \(\mc{O}^{D_{\mf{p}}(\mf{P})}/\smc{O}\). This proves the first statement. The second statement follows at once from the identities above.
    \end{proof}
    
    We can view this result as a statement that \(\mc{O}^{D_{\mf{p}}(\mf{P})}/\smc{O}\) encodes all of the splitting information about \(\mf{p}\mc{O}\) while \(\mc{O}/\mc{O}^{D_{\mf{p}}(\mf{P})}\) contains all of the ramification and inertia information. We will need to do more work to unpack the latter information. As \(\tau \in D_{\mf{p}}(\mf{P})\) leaves \(\mc{O}\) and \(\mf{P}\) invariant, it induces an automorphism \(\conj{\tau}\) of the residue class field \(\F_{\mf{P}}\) defined by
    \[
      \conj{\tau}:\F_{\mf{P}}\to \F_{\mf{P}}\qquad \a \mapsto \tau(\a)+\mf{P}.
    \]
    We then obtain a homomorphism
    \[
      T_{\mf{p}}^{\mf{P}}:D_{\mf{p}}(\mf{P}) \to \Aut(\F_{\mf{P}}/\F_{\mf{p}}) \qquad \tau \mapsto \conj{\tau}.
    \]
    It turns out that the residue class extensions \(\F_{\mf{P}}/\F_{\mf{p}}\) are normal and that \(T_{\mf{p}}^{\mf{P}}\) is surjective.
    
    \begin{proposition}\label{prop:T_map_is_surjective}
      Let \(\mc{O}/\smc{O}\) be a Dedekind extension of a finite Galois extension \(L/K\), let \(\mf{p}\) be a prime of \(\smc{O}\), and let \(\mf{P}\) be a prime of \(\mc{O}\) above \(\mf{p}\). Then the residue class extension \(\F_{\mf{P}}/\F_{\mf{p}}\) is normal. Moreover, the homomorphism \(T_{\mf{p}}^{\mf{P}}\) is surjective.
    \end{proposition}
    \begin{proof}
      As \(f_{\mf{p}}(\mf{P}^{D}) = 1\) by \cref{prop:decomposition_field_e_and_f}, we have an isomorphism \(\F_{\mf{P}^{D}} \cong \F_{\mf{p}}\). So without loss of generality, we may assume \(L^{D_{\mf{p}}(\mf{P})} = K\) which is to say \(D_{\mf{p}}(\mf{P}) = \Gal(L/K)\). For any \(\a \in \mc{O}\) let \(\conj{\a} \in \F_{\mf{P}}\) be its image under natural projection. Also let \(m_{\a}(x)\) and \(m_{\conj{\a}}(x)\) be the minimal polynomials of \(\a\) and \(\conj{\a}\) over \(K\) and \(\F_{\mf{p}}\) respectively. As \(m_{\a}(x)\) necessarily has coefficients in \(\smc{O}\) and hence \(\mc{O}\), let \(\conj{m_{\a}}(x) \in \F_{\mf{P}}[x]\) be the image of \(m_{\a}(x)\) under natural projection. Note that as \(\conj{\a}\) is a root of \(\conj{m_{\a}}(x)\), \(m_{\conj{\a}}(x)\) divides \(\conj{m_{\a}}(x)\) in \(\F_{\mf{P}}[x]\).

      For the first statement, let \(\a \in \mc{O}\). Then we must show that \(m_{\conj{\a}}(x)\) splits into linear factors over \(\F_{\mf{P}}\). As \(m_{\conj{\a}}(x)\) divides \(\conj{m_{\a}}(x)\) in \(\F_{\mf{P}}[x]\), it suffices to show \(\conj{m_{\a}}(x)\) splits into linear factors over \(\F_{\mf{P}}\). As \(L/K\) is Galois and thus normal, \(m_{\a}(x)\) splits into linear factors over \(L\). These linear factors belong to \(\mc{O}[x]\) since their roots are the Galois conjugates of \(\a\) which belong to \(\mc{O}\). It follows that \(\conj{m_{\a}}(x)\) splits into linear factors over \(\F_{\mf{P}}\) proving the first statement.
      
      For the second statement, consider the maximal separable subextension of \(\F_{\mf{P}}/\F_{\mf{p}}\). As this extension is finite, it is simple by the primitive element theorem so let \(\t \in \mc{O}\) be such that \(\conj{\t}\) is a primitive element. Then this subextension is \(\F_{\mf{p}}(\conj{\t})/\F_{\mf{p}}\). Now let \(\conj{\tau} \in \Aut(\F_{\mf{P}}/\F_{\mf{p}})\). Since \(\conj{\tau}\) fixes \(\F_{\mf{p}}\) pointwise, \(\conj{\tau}(\conj{\t})\) is a root of \(m_{\conj{\t}}(x)\). As \(m_{\conj{\t}}(x)\) divides \(\conj{m_{\t}}(x)\) in \(\F_{\mf{P}}[x]\), we find that \(\conj{\tau}(\conj{\t})\) is also a root of \(\conj{m_{\t}}(x)\). Therefore there is a root \(\t'\) of \(m_{\t}(x)\) whose image under natural projection is \(\conj{\tau}(\conj{\t})\). Because \(L/K\) is Galois with Galois group \(D_{\mf{p}}(\mf{P})\), we can find \(\tau \in D_{\mf{p}}(\mf{P})\) such that \(\tau(\t) = \t'\). Then the image of \(\conj{\t}\) under \(T_{\mf{p}}^{\mf{P}}(\tau)\) and \(\conj{\tau}\) are the same whence these automorphisms are identical on \(\F_{\mf{p}}\). As \(\F_{\mf{P}}/\F_{\mf{p}}(\conj{\t})\) is purely inseparable, it has trivial automorphism group. This forces \(T_{\mf{p}}^{\mf{P}}(\tau) = \conj{\tau}\) proving surjectivity.
    \end{proof}

    As this result shows that \(T_{\mf{p}}^{\mf{P}}\) is surjective, we define the \textbf{inertia group}\index{inertia group} \(I_{\mf{p}}(\mf{P})\) of \(\mf{P}\)  over \(\mf{p}\) by
    \[
      I_{\mf{p}}(\mf{P}) = \ker{T_{\mf{p}}^{\mf{P}}}.
    \]
    Being the kernel of a homomorphism, \(I_{\mf{p}}(\mf{P})\) is a normal subgroup of \(D_{\mf{p}}(\mf{P})\). The associated \textbf{inertia field}\index{inertia field} \(L^{I_{\mf{p}}(\mf{P})}\) of \(\mf{P}\) over \(\mf{p}\) is defined to be
    \[
      L^{I_{\mf{p}}(\mf{P})} = \{\l \in L:\tau(\l) = \l \text{ for all } \tau \in I_{\mf{p}}(\mf{P})\}.
    \]
    In other words, the inertia field of is the fixed field of \(L\) by the inertia group and hence an intermediate field of \(L/K\). In particular, the fundamental theorem of Galois theory gives
    \[
      \Gal(L/L^{I_{\mf{p}}(\mf{P})}) = I_{\mf{p}}(\mf{P}),
    \]
    \(L^{I_{\mf{p}}(\mf{P})}\) is an intermediate field of the Galois extension \(L/L^{D_{\mf{p}}(\mf{P})}\) as \(I_{\mf{p}}(\mf{P})\) is a normal subgroup of \(D_{\mf{p}}(\mf{P})\), and
    \[
      \Gal(L^{I_{\mf{p}}(\mf{P})}/L^{D_{\mf{p}}(\mf{P})}) \cong D_{\mf{p}}(\mf{P})/I_{\mf{p}}(\mf{P}).
    \]
    Let \(\mc{O}^{I_{\mf{p}}(\mf{P})}\) be the integral closure of \(\smc{O}\) in \(L^{I_{\mf{p}}(\mf{P})}\). Then \cref{prop:integral_closure_is_transitive,prop:integral_closure_of_Dedekind_is_Dedekind} together imply that \(\mc{O}/\mc{O}^{I_{\mf{p}}(\mf{P})}/\mc{O}^{D_{\mf{p}}(\mf{P})}\) is a tower of Dedekind extensions for the finite separable tower \(L/L^{I_{\mf{p}}(\mf{P})}/L^{D_{\mf{p}}(\mf{P})}\). We illustrate this relationship via the diagram
    \begin{center}
      \begin{tikzcd}[row sep=large, column sep=large]
        \mf{P} \subset \mc{O} \subseteq L \arrow[dash]{d} \\
        \mf{P}^{I} \subset \mc{O}^{I_{\mf{p}}(\mf{P})} \subseteq L^{I_{\mf{p}}(\mf{P})} \arrow[dash]{d} \\
        \mf{P}^{D} \subset \mc{O}^{D_{\mf{p}}(\mf{P})} \subseteq L^{D_{\mf{p}}(\mf{P})}.
      \end{tikzcd}
    \end{center}
    The inertia field is closely related to the automorphism group of the residue class extension \(\F_{\mf{P}}/\F_{\mf{p}}\). Indeed, as \(T_{\mf{p}}^{\mf{P}}\) is surjective by \cref{prop:T_map_is_surjective}, the first isomorphism theorem shows
    \[
      \Gal(L^{I_{\mf{p}}(\mf{P})}/L^{D_{\mf{p}}(\mf{P})}) \cong \Aut(\F_{\mf{P}}/\F_{\mf{p}}).
    \]
    In fact, the decomposition and inertia groups of \(\mf{P}\) over \(\mf{p}\) fit into an exact sequence.

    \begin{proposition}\label{prop:inertia_group_exact_sequence}
      Let \(\mc{O}/\smc{O}\) be a Dedekind extension of a finite Galois extension \(L/K\), let \(\mf{p}\) be a prime of \(\smc{O}\), and let \(\mf{P}\) be a prime of \(\mc{O}\) above \(\mf{p}\). Then the sequence
      \begin{center}
        \begin{tikzcd}
          1 \arrow{r} & I_{\mf{p}}(\mf{P}) \arrow{r} & D_{\mf{p}}(\mf{P}) \arrow{r} & \Aut(\F_{\mf{P}}/\F_{\mf{p}}) \arrow{r} & 1,
        \end{tikzcd}
      \end{center}
      where the third map map takes any \(\tau\) to its induced map \(T_{\mf{p}}^{\mf{P}}(\tau)\), is exact.
    \end{proposition}
    \begin{proof}
      As the second map is injective and the fourth map is surjective by \cref{prop:T_map_is_surjective}, the sequence is exact at \(I_{\mf{p}}(\mf{P})\) and \(\Aut(\F_{\mf{P}}/\F_{\mf{p}})\). So it remains to prove exactness at \(D_{\mf{p}}(\mf{P})\). This follows because \(I_{\mf{p}}(\mf{P})\) is the kernel of \(T_{\mf{p}}^{\mf{P}}\) by definition.
    \end{proof}

    We can say more when the residue class extension \(\F_{\mf{P}}/\F_{\mf{p}}\) is separable. For in this case, \cref{prop:T_map_is_surjective} implies \(\F_{\mf{P}}/\F_{\mf{p}}\) is Galois. Whence
    \[
      \Gal(L^{I_{\mf{p}}(\mf{P})}/L^{D_{\mf{p}}(\mf{P})}) \cong \Gal(\F_{\mf{P}}/\F_{\mf{p}}).
    \]
    Then \cref{equ:decomposition_group_e_f_relation} and this isomorphism together imply 
    \[
      |\Gal(L^{I_{\mf{p}}(\mf{P})}/L^{D_{\mf{p}}(\mf{P})})| = f \quad \text{and} \quad |I_{\mf{p}}(\mf{P})| = e.
    \]
    It follows that \(L^{I_{\mf{p}}(\mf{P})}/L^{D_{\mf{p}}(\mf{P})}\) has degree \(f\) and \(L/L^{I_{\mf{p}}(\mf{P})}\) has degree \(e\) as \(L/L^{D_{\mf{p}}(\mf{P})}\) has degree \(ef\). In particular, \(\mf{P}\) is totally ramified in \(\mc{O}/\mc{O}^{I_{\mf{p}}(\mf{P})}\) and \(\mf{P}^{D}\) is inert in \(\mc{O}^{I_{\mf{p}}(\mf{P})}/\mc{O}^{D_{\mf{p}}(\mf{P})}\).

    \begin{proposition}\label{prop:inertia_field_e_and_f}
      Let \(\mc{O}/\smc{O}\) be a Dedekind extension of a finite Galois extension \(L/K\), let \(\mf{p}\) be a prime of \(\smc{O}\) with inertia degree \(f\) and ramification index \(e\), and let \(\mf{P}\) be a prime of \(\mc{O}\) above \(\mf{p}\). Moreover, suppose the residue class extension \(\F_{\mf{P}}/\F_{\mf{p}}\) is separable. Then \(\mf{P}\) is totally ramified in \(\mc{O}/\mc{O}^{I_{\mf{p}}(\mf{P})}\) and \(\mf{P}^{D}\) is inert in \(\mc{O}^{I_{\mf{p}}(\mf{P})}/\mc{O}^{D_{\mf{p}}(\mf{P})}\).
    \end{proposition}
    \begin{proof}
      Recall that \(\Gal(L/L^{I_{\mf{p}}(\mf{P})}) = I_{\mf{p}}(\mf{P})\). By \cref{prop:Galois_action_on_primes_is_transitive}, the primes of \(\mc{O}\) above \(\mf{P}^{I}\) are of the form \(\tau(\mf{P})\) for \(\tau \in I_{\mf{p}}(\mf{P})\). These \(\tau\) leave \(\mf{P}\) invariant whence \(\mf{P}^{I}\) is nonsplit in \(\mc{O}/\mc{O}^{I_{\mf{p}}(\mf{P})}\). In view of this fact and that \(L/L^{I_{\mf{p}}(\mf{P})}\) has degree \(e\), the fundamental equality implies
      \[
        e = e_{\mf{P}^{I}}(\mf{P})f_{\mf{P}^{I}}(\mf{P}).
      \]
      So \(\mf{P}^{I}\) is totally ramified in \(\mc{O}/\mc{O}^{I_{\mf{p}}(\mf{P})}\) if \(f_{\mf{P}^{I}}(\mf{P}) = 1\). As \(\mf{P}^{I}\) is nonsplit in \(\mc{O}/\mc{O}^{I_{\mf{p}}(\mf{P})}\), we have \(D_{\mf{P}^{I}}(\mf{P}) = I_{\mf{p}}(\mf{P})\). Whence \(I_{\mf{P}^{I}}(\mf{P}) = I_{\mf{p}}(\mf{P})\). In short, the inertia and decomposition groups are the same. By \cref{prop:inertia_group_exact_sequence}, the only way this is possible is if \(\Gal(\F_{\mf{P}^{I}}/\F_{\mf{P}}) = \{1\}\). This forces \(f_{\mf{P}^{I}}(\mf{P}) = 1\) as desired. As \(L/L^{D_{\mf{p}}(\mf{P})}\) has degree \(ef\) and \(\mf{P}^{D}\) is nonsplit in \(\mc{O}/\mc{O}^{D_{\mf{p}}(\mf{P})}\) by \cref{prop:decomposition_field_e_and_f}, the fundamental equality implies
      \[
        ef = e_{\mf{P}^{D}}(\mf{P})f_{\mf{P}^{D}}(\mf{P}).
      \]
      Since \(\mc{O}/\mc{O}^{I_{\mf{p}}(\mf{P})}/\mc{O}^{D_{\mf{p}}(\mf{P})}\) is a tower of Dedekind extensions, \cref{equ:ramification_and_interia_towers} gives
      \[
        e_{\mf{P}^{D}}(\mf{P}) = e_{\mf{P}^{I}}(\mf{P})e_{\mf{P}^{D}}(\mf{P}^{I}) \quad \text{and} \quad f_{\mf{P}^{D}}(\mf{P}) = f_{\mf{P}^{I}}(\mf{P})f_{\mf{P}^{D}}(\mf{P}^{I}).
      \]
      Upon combining these identities and our previous work, \(f_{\mf{P}^{D}}(\mf{P}^{I}) = f\) which is the degree of \(L^{I_{\mf{p}}(\mf{P})}/L^{D_{\mf{p}}(\mf{P})}\). Therefore \(\mf{P}^{D}\) is inert in \(\mc{O}^{I_{\mf{p}}(\mf{P})}/\mc{O}^{D_{\mf{p}}(\mf{P})}\).
    \end{proof}

    In the case the residue class extension \(\F_{\mf{P}}/\F_{\mf{p}}\) is separable, we fit the decomposition and inertia fields into the diagram
    \begin{center}
      \begin{tikzcd}[row sep=large, column sep=large]
        \mf{P} \subset \mc{O} \subseteq L \arrow[dash]{d}{1}[swap]{e} \\
        \mf{P}^{I} \subset \mc{O}^{I_{\mf{p}}(\mf{P})} \subseteq L^{I_{\mf{p}}(\mf{P})} \arrow[dash]{d}{f}[swap]{1} \\
        \mf{P}^{D} \subset \mc{O}^{D_{\mf{p}}(\mf{P})} \subseteq L^{D_{\mf{p}}(\mf{P})} \arrow[dash]{d}{1}[swap]{1} \\
        \mf{p} \subset \smc{O} \subseteq K.
      \end{tikzcd}
    \end{center}
    The labels on left and right of the extensions represent the corresponding ramification indices and inertia degrees respectively which come from \cref{prop:decomposition_field_e_and_f,prop:inertia_field_e_and_f}. So the extension \(L^{D_{\mf{p}}(\mf{P})}/K\) contains all of the information about the splitting, the extension \(L^{I_{\mf{p}}(\mf{P})}/L^{D_{\mf{p}}(\mf{P})}\) contains all of the information about the inertia degrees, and the extension \(L/L^{I_{\mf{p}}(\mf{P})}\) contains all of the information about the ramification indices. In particular, \(\mf{p}\) is unramified in \(\mc{O}/\smc{O}\) if and only if \(L^{I_{\mf{p}}(\mf{P})} = L\) which is equivalent to \(I_{\mf{p}}(\mf{P}) = \{1\}\).

    In the case of a Galois extension of number fields \(L/K\) and a prime \(\mf{p}\) of \(K\), the \textbf{ramification index}\index{ramification index} \(e\) and \textbf{inertia degree}\index{inertia degree} \(f\) of \(\mf{p}\) in \(L/K\) is that of \(\mc{O}_{L}/\mc{O}_{K}\). Recall that all of the residue class field extensions are separable. This means that the above diagram applies to \(L/K\).
  \iffalse\section{\todo{The Different and Discriminant}}
    Let \(\mc{O}/\smc{O}\) be a Dedekind extension of a degree \(n\) separable extension \(L/K\). It is exceptionally rare for a prime \(\mf{p}\) of \(K\) to ramify. We will construct two integral ideals, one of \(\mc{O}\) and the other of \(\smc{O}\) which will tell us which primes ramify in \(L\) or \(K\). These integral ideals are the different and discriminant respectively. To describe them, we will need the concept of lattices in Dedekind domains. We say that \(\mf{L}\) is a \textbf{\(\smc{O}\)-lattice}\index{\(\smc{O}\)-lattice} if it is a finitely generated \(\smc{O}\)-submodule of \(L\). Moreover, we say that \(\mf{L}\) is \textbf{complete}\index{complete} if it spans \(L/K\). That is, \(\mf{L}\) contains a basis of \(L/K\).

    \begin{remark}
      A \(\Z\)-lattice is an integral lattice and a complete \(\Z\)-lattice is a complete integral lattice.
    \end{remark}

    A \(\smc{O}\)-lattice \(\mf{L}\) need not be a fractional ideal of \(\smc{O}\) since it does not need to be a \(\mc{O}\)-submodule of \(L\). However, every fractional ideal \(\mf{F}\) of \(\mc{O}\) is a complete \(\smc{O}\)-lattice. Indeed, by \cref{prop:Dedekind_extension_facts} \(\mc{O}\) is a finitely generated \(\smc{O}\)-module and so \(\mf{F}\) is a finitely generated \(\smc{O}\)-module as well. Moreover, \(\mf{F}\) contains a basis of \(L/K\). For if \(\l_{1},\ldots,\l_{n}\) is a basis for \(L/K\) we may use \cref{prop:field_of_fractions_AKBL} to multiply by a nonzero element of \(\smc{O}\), if necessary, to ensure that this basis is contained in \(\mc{O}\). Choosing any nonzero \(\a \in \mf{F}\), \(\a\l_{1},\ldots,\a\l_{n}\) is a basis for \(L/K\) inside \(\mf{F}\).

    \begin{remark}
      A \(\smc{O}\)-lattice \(\mc{L}\) need not be a free \(\smc{O}\)-submodule of \(L\). Indeed, we are only guaranteed that \(\mc{L}\) is a finitely generated \(\smc{O}\)-submodule of \(L\) not that it is a free finitely generated \(\smc{O}\)-submodule of \(L\).
    \end{remark}
    
    Recall by \cref{lem:trace_is_nondegenerate} that there is a nondegenerate symmetric bilinear form on \(K\) given by
    \[
      \Trace_{L/K}:L \x L \to K \qquad (\l,\eta) \mapsto \Trace_{L/K}(\l\eta).
    \]
    We call this bilinear form the \textbf{trace form}\index{trace form} of \(L/K\). The trace form makes \(L\) into a nondegenerate inner product space over \(K\) and so every basis \(\l_{1},\ldots,\l_{n}\) admits a dual basis \(\l_{1}^{\vee},\ldots,\l_{n}^{\vee}\) with respect to \(\Trace_{L/K}\) defined by
    \[
      \Trace_{L/K}(\l_{i}\l_{j}^{\vee}) = \d_{i,j},
    \]
    for \(1 \le i,j \le n\). The trace form will allow us to introduce duals. Indeed, if \(\mf{F}\) is a fractional ideal \(\mc{O}\), the \textbf{dual}\index{dual} \(\mf{F}^{\vee}\) of \(\mf{F}\) is defined by
    \[
      \mf{F}^{\vee} = \{\l \in L:\Trace_{L/K}(\l\mf{F}) \subseteq \smc{O}\}.
    \]
    We say that \(\mf{F}\) is \textbf{self-dual}\index{self-dual} if \(\mf{F}^{\vee} = \mf{F}\). The following proposition shows that the dual \(\mf{F}^{\vee}\) is indeed a fractional ideal:

    \begin{proposition}\label{prop:dual_lattice_is_fractional}
      Let \(\mc{O}/\smc{O}\) be a Dedekind extension of a degree \(n\) separable extension \(L/K\) and let \(\mf{F}\) be a fractional ideal of \(\mc{O}\). Then \(\mf{F}^{\vee}\) is also a fractional ideal of \(\mc{O}\) and
      \[
        \mf{F}^{\vee} = \mf{F}^{-1}\mc{O}^{\vee}.
      \]
    \end{proposition}
    \begin{proof}
      We will first show that \(\mf{F}^{\vee}\) is a finitely generated \(\smc{O}\)-submodule of \(L\). Let \(\l_{1},\ldots,\l_{n}\) be a basis for \(L/K\). Using \cref{prop:field_of_fractions_AKBL} to multiply by a nonzero element of \(\smc{O}\), if necessary, we can ensure that this basis is contained in \(\mc{O}\). Now choose some nonzero \(\a \in \mf{F} \cap \smc{O}\) (such an element exists since every prime \(\mf{P}\) of \(\mc{O}\) is above a prime \(\mf{p}\) of \(\smc{O}\) and \(\mf{F}\) is of the form \(\mf{F} = \frac{1}{\d}\mf{A}\) for some nonzero \(\d \in \mc{O}\) and integral ideal \(\mf{A}\) so that \(\mf{A} \subseteq \mf{F}\)). Now suppose \(\l \in \mf{F}^{\vee}\) and write \(\l = \k_{1}\l_{1}+\cdots+\k_{n}\l_{n}\) with \(\k_{i} \in K\) for \(1 \le i \le n\). Then linearity of the trace implies
      \[
        \sum_{1 \le j \le n}\a\k_{j}\Trace_{L/K}(\l_{i}\l_{j}) = \Trace_{L/K}(\a\l_{i}\l),
      \]
      for \(1 \le i \le n\). These \(n\) equations are equivalent to the identity
      \[
        \Trace_{L/K}(\l_{1},\ldots,\l_{n})\begin{pmatrix} \a\k_{1} \\ \vdots \\ \a\k_{n} \end{pmatrix} = \begin{pmatrix} \Trace_{L/K}(\a\l_{1}\l) \\ \vdots \\ \Trace_{L/K}(\a\l_{n}\l) \end{pmatrix}.
      \]
      Multiplying on the left by the adjugate \(\adj(\Trace_{L/K}(\l_{1},\ldots,\l_{n}))\) of \(\Trace_{L/K}(\l_{1},\ldots,\l_{n})\) and recalling that a matrix times its adjugate is its determinant times the identity, we see that
      \[
        d_{L/K}(\l_{1},\ldots,\l_{n})\begin{pmatrix} \a\k_{1} \\ \vdots \\ \a\k_{n} \end{pmatrix} = \adj(\Trace_{L/K}(\l_{1},\ldots,\l_{n}))\begin{pmatrix} \Trace_{L/K}(\a\l_{1}\l) \\ \vdots \\ \Trace_{L/K}(\a\l_{n}\l) \end{pmatrix}.
      \]
      Since \(\l_{1},\ldots,\l_{n}\) is a basis for \(L/K\) that is contained in \(\mc{O}\), the matrix \(\Trace_{L/K}(\l_{1},\ldots,\l_{n})\) has entries in \(\smc{O}\) by \cref{prop:trace_and_norm_AKBL} and therefore \(\adj(\Trace_{L/K}(\l_{1},\ldots,\l_{n}))\) does too. As \(\a\l_{i} \in \mf{F}\) (because \(\l_{j} \in \mc{O}\) for all \(j\)) and \(\l \in \mf{F}^{\vee}\), \cref{prop:trace_and_norm_AKBL} again implies \(\Trace_{L/K}(\a\l_{i}\l) \in \smc{O}\) for all \(i\). So the right-hand side has entires in \(\smc{O}\) and hence the left-hand side must as well. This means \(d_{L/K}(\l_{1},\ldots,\l_{n})\a\k_{i} \in \smc{O}\) for all \(i\). Since \(\l \in \mf{F}^{\vee}\) was arbitrary, we have
      \[
        \a d_{L/K}(\l_{1},\ldots,\l_{n})\mf{F}^{\vee} \subseteq \mc{O}.
      \]
      As \(\mc{O}\) is a finitely generated \(\smc{O}\)-module by \cref{prop:Dedekind_extension_facts}, it follows that \(\mf{F}^{\vee}\) is a finitely generated \(\smc{O}\)-submodule of \(L\). Therefore \(\mf{F}^{\vee}\) is also a finitely generated \(\mc{O}\)-submodule of \(L\) if it is preserved under multiplication by \(\mc{O}\). Let \(\a \in \mc{O}\) and \(\b \in \mf{F}^{\vee}\). Then we must show \(\a\b \in \mf{F}^{\vee}\). To see this, observe that \(\Trace_{L/K}(\a\b\mf{F}) \subseteq \Trace_{L/K}(\b\mf{F}) \subseteq \smc{O}\) by \cref{prop:trace_and_norm_AKBL} since \(\a\mf{F} \subseteq \mf{F}\) and \(\b \in \mf{F}^{\vee}\). Therefore \(\a\b \in \mf{F}^{\vee}\) and hence \(\mf{F}^{\vee}\) is a fractional ideal proving the first statement. To prove the second statement we will show containment in both directions. For the forward containment, first suppose \(\a \in \mf{F}^{\vee}\) and \(\b \in \mf{F}\). Then \(\Trace_{L/K}(\a\b\mc{O}) \subseteq \Trace_{L/K}(\a\mf{F}) \subseteq \smc{O}\) by \cref{prop:trace_and_norm_AKBL} since \(\b\mc{O} \subseteq \mf{F}\) and \(\a \in \mf{F}^{\vee}\). Hence \(\a\b \in \mc{O}^{\vee}\) so that \(\mf{F}^{\vee}\mf{F} \subseteq \mc{O}^{\vee}\) and therefore \(\mf{F}^{\vee} \subseteq \mf{F}^{-1}\mc{O}^{\vee}\). This proves the forward containment. For the reverse containment, suppose \(\a \in \mf{F}^{-1}\) and \(\b \in \mc{O}^{\vee}\). Then \(\Trace_{L/K}(\a\b\mf{F}) \subseteq \Trace_{L/K}(\b\mc{O}) \subseteq \smc{O}\) by \cref{prop:trace_and_norm_AKBL} since \(\a\mf{F} \subseteq \mc{O}\) and \(\b \in \mc{O}^{\vee}\). Thus \(\a\b \in \mf{F}^{\vee}\) implying \(\mf{F}^{-1}\mc{O}^{\vee} \subseteq \mf{F}^{\vee}\) and proving the reverse containment. This completes the proof.
    \end{proof}

    As the dual \(\mf{F}^{\vee}\) is a fractional ideal by \cref{prop:dual_lattice_is_fractional}, it is also a complete \(\smc{O}\)-lattice. As we might hope, localization respects duals:

    \begin{proposition}\label{prop:localization_of_dual_is_dual_of_localization}
      Let \(\mc{O}/\smc{O}\) be a Dedekind extension of a degree \(n\) separable extension \(L/K\) and let \(D\) be a multiplicative subset of \(\smc{O}\). Then for any fractional ideal \(\mf{F}\) of \(\mc{O}\), we have
      \[
        (\mf{F}D^{-1})^{\vee} = (\mf{F}^{\vee})D^{-1}.
      \]
    \end{proposition}
    \begin{proof}
      For the forward containment, suppose \(\l \in (\mf{F}D^{-1})^{\vee}\). Then \(\Trace_{L/K}(\l\mf{F}D^{-1}) \subseteq \smc{O}D^{-1}\). As \(D \subset K\), linearity of the trace implies \(\l = \frac{\a}{\d}\) with \(\Trace_{L/K}(\a\mf{F})\d \subseteq \smc{O}D^{-1}\) so that \(\a \in \mf{F}^{\vee}\) and \(\d \in D\). Hence \(\l \in (\mf{F}^{\vee})D^{-1}\) and the forward containment follows. For the reverse containment, suppose \(\frac{\a}{\d} \in (\mf{F}^{\vee})D^{-1}\). Then \(\a \in \mf{F}^{\vee}\) so that \(\Trace_{L/K}(\a\mf{F}) \subseteq \smc{O}\) and \(\d \in D\). As \(D \subset K\), linearity of the trace implies \(\Trace_{L/K}(\l\mf{F}) \subseteq \smc{O}D^{-1}\) with \(\l = \frac{\a}{\d}\). Hence \(\frac{\a}{\d} \in (\mf{F}D^{-1})^{\vee}\) and the reverse containment holds.
    \end{proof}
    
    We will now introduce the different and the discriminant. We define the \textbf{complement}\index{complement} \(\mf{C}_{\mc{O}/\smc{O}}\) of \(\mc{O}/\smc{O}\) by
    \[
      \mf{C}_{\mc{O}/\smc{O}} = \mc{O}^{\vee}.  
    \]
    This is a fractional ideal by \cref{prop:dual_lattice_is_fractional}. The \textbf{different}\index{different} \(\mf{D}_{\mc{O}/\smc{O}}\) of \(\mc{O}/\smc{O}\) is defined to be the inverse of the complement \(\mf{C}_{\mc{O}/\smc{O}}\):
    \[
      \mf{D}_{\mc{O}/\smc{O}} = (\mc{O}^{\vee})^{-1}.  
    \]
    As \(\mc{O} \subseteq \mc{O}^{\vee}\) by \cref{prop:trace_and_norm_AKBL}, it follows from \cref{prop:explicit_inverse_ideal} that \(\mf{D}_{\mc{O}/\smc{O}}\) is an integral ideal and
    \[
      \mf{D}_{\mc{O}/\smc{O}} = \{\l \in L: \l\mc{O}^{\vee} \subseteq \mc{O}\}.  
    \]
    Also, \cref{lem:cancellation_isomorphism} gives the first isomorphism in the following chain:
    \begin{equation}\label{equ:different_isomorphism_chain}
      \mc{O}/\mf{D}_{\mc{O}/\smc{O}} \cong \mf{D}_{\mc{O}/\smc{O}}^{-1}/\mc{O} \cong \mc{O}^{\vee}/\mc{O}.
    \end{equation}
    From this chain of isomorphisms we find that
    \[
      \mf{D}_{\mc{O}/\smc{O}} \subseteq \mc{O} \subseteq \mc{O}^{\vee}.
    \]
    Therefore the different \(\mf{D}_{\mc{O}/\smc{O}}\) is a measure of how much \(\mc{O}\) fails to be self-dual for if \(\mf{D}_{\mc{O}/\smc{O}} = \mc{O}^{\vee}\) then \(\mc{O}\) must be self-dual. Moreover, by \cref{prop:dual_lattice_is_fractional} we can express the dual fractional ideal \(\mf{F}^{\vee}\) of a fractional ideal \(\mf{F}\) of \(\smc{O}\) in terms of the different as
    \[
      \mf{F}^{\vee} = \mf{F}^{-1}\mf{D}_{\mc{O}/\smc{O}}^{-1}.
    \]
    Now for the discriminant. We define the \textbf{discriminant}\index{discriminant} \(\mf{d}_{\mc{O}/\smc{O}}\) of \(\mc{O}/\smc{O}\) to be the ideal of \(\smc{O}\) generated by all discriminants \(d_{L/K}(\a_{1},\ldots,\a_{n})\) as \(\a_{1},\ldots,\a_{n}\) run over all bases of \(L/K\) contained in \(\mc{O}\). Note that \(\mf{d}_{\mc{O}/\smc{O}}\) is necessarily an integral ideal. Recall that if \(\mc{O}/\smc{O}\) admits an integral then \(\mc{O}\) is a free \(\smc{O}\)-module of rank \(n\). In this case, bases of \(L/K\) contained in \(\mc{O}\) are precisely the integral bases of \(\mc{O}/\smc{O}\). Then
    \[
      \mf{d}_{\mc{O}/\smc{O}} = d_{\smc{O}}(\mc{O})\smc{O},
    \]
    so that the discriminant \(\mf{d}_{\mc{O}/\smc{O}}\) is generated by the discriminant \(d_{\smc{O}}(\mc{O})\). In particular, \(\mf{d}_{\mc{O}/\smc{O}}\) is a principal integral ideal if \(\mc{O}/\smc{O}\) admits an integral basis. By \cref{thm:integral_basis_AKBL} this will hold if \(\smc{O}\) is a principal ideal domain but not necessarily in general. The different and discriminant also respect localization:

    \begin{proposition}\label{prop:different_and_discriminant_respect_localization}
      Let \(\mc{O}/\smc{O}\) be a Dedekind extension of a degree \(n\) separable extension \(L/K\) and let \(D\) be a multiplicative subset of \(\smc{O}\). Then
      \[
        \mf{D}_{\mc{O}D^{-1}/\smc{O}D^{-1}} = \mf{D}_{\mc{O}/\smc{O}}D^{-1} \quad \text{and} \quad \mf{d}_{\mc{O}D^{-1}/\smc{O}D^{-1}} = \mf{d}_{\mc{O}/\smc{O}}D^{-1}
      \]
    \end{proposition}
    \begin{proof}
      For the first identity, it is equivalent to show
      \[
        ((\mc{O}D^{-1})^{\vee})^{-1} = (\mc{O}^{\vee})^{-1}D^{-1},
      \]
      by definition of the different. Applying \cref{prop:localization_of_inverse_is_inverse_of_localization,prop:localization_of_dual_is_dual_of_localization}, we see that the right-had side is equal to the left-hand side as desired. For the second identity, we will show containment in both directions. For the forward containment, suppose \(\frac{\a_{1}}{\d_{1}},\ldots,\frac{\a_{n}}{\d_{n}}\) is a basis of \(L/K\) contained in \(\mc{O}D^{-1}\). Setting \(\d = \d_{1} \cdots \d_{n}\), we see that \(\frac{\a_{1}\d}{\d_{1}},\ldots,\frac{\a_{n}\d}{\d_{n}}\) is a basis of \(L/K\) contained in \(\mc{O}\). As \(D \subset K\) and \(d_{L/K}\left(\frac{\a_{1}\d}{\d_{1}},\ldots,\frac{\a_{n}\d}{\d_{n}}\right) \in \mf{d}_{\mc{O}/\smc{O}}\), linearity of the trace implies \(d_{L/K}\left(\frac{\a_{1}}{\d_{1}},\ldots,\frac{\a_{n}}{\d_{n}}\right) \in \mf{d}_{\mc{O}/\smc{O}}D^{-1}\). This shows the forward containment. For the reverse containment, suppose \(\a_{1},\ldots,\a_{n}\) is a basis of \(L/K\) contained in \(\mc{O}\) and let \(\d \in D^{-1}\). Then \(\frac{\a_{1}}{\d},\ldots,\frac{\a_{n}}{\d}\) is a basis for \(L/K\) contained in \(\mc{O}D^{-1}\). As \(D \subseteq K\) and \(d_{L/K}\left(\frac{\a_{1}}{\d},\ldots,\frac{\a_{n}}{\d}\right) \in \mf{d}_{\mc{O}D^{-1}/\smc{O}D^{-1}}\), linearity of the trace again implies \(d_{L/K}(\a_{1}\ldots,\a_{n}) \in \mf{d}_{\mc{O}D^{-1}/\smc{O}D^{-1}}\) proving the reverse containment.
    \end{proof}

    We want to show that a prime \(\mf{P}\) of \(\mc{O}\) ramifies in \(\mc{O}/\smc{O}\) if and only if it divides the different \(\mf{D}_{\mc{O}/\smc{O}}\) provided the residue class extensions are separable. We first require a lemma:

    \begin{lemma}\label{lem:different_containment_lemma}
      Let \(\mc{O}/\smc{O}\) be a Dedekind extension of a degree \(n\) separable extension \(L/K\) and let \(\mf{A}\) be an integral ideal of \(\mc{O}\). Then \(\mf{A} \mid \mf{D}_{\mc{O}/\smc{O}}\) if and only if \(\Trace_{L/K}(\mf{A}^{-1}) \subseteq \smc{O}\).
    \end{lemma}
    \begin{proof}
      \(\mf{A}\) divides \(\mf{D}_{\mc{O}/\smc{O}}\) if and only if \(\mf{D}_{\mc{O}/\smc{O}} \subseteq \mf{A}\). This equivalent to \((\mc{O}^{\vee})^{-1} \subseteq \mf{A}\) and inverting shows the further equivalence \(\mf{A}^{-1} \subseteq \mc{O}^{\vee}\). This last containment is equivalent to \(\Trace_{L/K}(\mf{A}^{-1}) \subseteq \smc{O}\) which completes the proof.
    \end{proof}

    With \cref{lem:different_containment_lemma} in hand, we can now prove our claim:

    \begin{theorem}\label{thm:ramifies_if_and_only_if_divides_the_different}
      Let \(\mc{O}/\smc{O}\) be a Dedekind extension of a degree \(n\) separable extension \(L/K\) and assume that all residue class extensions are separable. Then a prime \(\mf{P}\) of \(\mc{O}\) ramifies in \(\mc{O}/\smc{O}\) if and only if \(\mf{P}\) divides \(\mf{D}_{\mc{O}/\smc{O}}\). In particular, if \(\mf{P}\) is above \(\mf{p}\) then the following hold:
      \begin{enumerate}[label*=(\roman*)]
        \item \(\mf{P}^{e_{\mf{p}}(\mf{P})} \mid \mf{D}_{\mc{O}/\smc{O}}\) if and only if \(e_{\mf{p}}(\mf{P}) \equiv 0 \tmod{\mf{p}}\).
        \item \(\mf{P}^{e_{\mf{p}}(\mf{P})-1} \mid\mid \mf{D}_{\mc{O}/\smc{O}}\) if and only if \(e_{\mf{p}}(\mf{P}) \not\equiv 0 \tmod{\mf{p}}\).
      \end{enumerate}
    \end{theorem}
    \begin{proof}
      We first show that (i) and (ii) together imply that \(\mf{P}\) ramifies in \(\mc{O}/\smc{O}\) if and only if \(\mf{P} \mid \mf{D}_{\mc{O}/\smc{O}}\). Since the residue class extensions are separable by assumption, \(\mf{P}\) ramifies in \(\mc{O}/\smc{O}\) if and only if \(e_{\mf{p}}(\mf{P}) \ge 2\). Therefore (i) and (ii) together show that \(\mf{P}\) ramifies in \(\mc{O}/\smc{O}\) if and only if \(\mf{P} \mid \mf{D}_{\mc{O}/\smc{O}}\) (note that \(1 \not\equiv 0 \tmod{\mf{p}}\) for any prime \(\mf{p}\)). Therefore it remains to prove (i) and (ii). In view of \cref{prop:ramification_setup_respects_localization,prop:different_and_discriminant_respect_localization}, it suffices to assume \(\mc{O}/\smc{O}\) is a local Dedekind extension. Therefore \(\smc{O}\) is a discrete valuation ring, \(\mc{O}\) is a principal ideal domain, and \(\mc{O}/\smc{O}\) admits an integral basis \(\a_{1},\ldots,\a_{n}\) making \(\mc{O}\) a free \(\smc{O}\)-module of rank \(n\). We will first show \(\mf{P}^{e_{\mf{p}}(\mf{P})-1} \mid \mf{D}_{\mc{O}/\smc{O}}\) independent of \(e_{\mf{p}}(\mf{P})\) modulo \(\mf{p}\) (note that this is satisfied by both (i) and (ii)). To this end, write \(\mf{p}\mc{O} = \mf{P}^{e_{\mf{p}}(\mf{P})-1}\mf{A}\) for some integral ideal \(\mf{A}\) of \(\mc{O}\) so that \(\mf{P} \mid\mid \mf{A}\). By \cref{lem:different_containment_lemma}, \(\mf{P}^{e_{\mf{p}}(\mf{P})-1} \mid \mf{D}_{\mc{O}/\smc{O}}\) if and only if \(\Trace_{L/K}(\mf{P}^{1-e_{\mf{p}}(\mf{P})}) \subseteq \smc{O}\). Since \(\mf{P}^{1-e_{\mf{p}}(\mf{P})} = \mf{p}^{-1}\mf{A}\), linearity of the trace implies that this is equivalent to \(\Trace_{L/K}(\mf{A}) \subseteq \mf{p}\). By \cref{prop:trace_and_norm_reduce_for_integral_basis}, \(\Trace_{K/L}(\mf{A}) = \Trace_{\mc{O}/\smc{O}}(\mf{A})\) and so it is further equivalent to show
      \[
        \Trace_{\mc{O}/\smc{O}}(\mf{A}) \subseteq \mf{p},
      \]
      Let \(\a \in \mf{A}\) and \(T_{\a}\) be the multiplication by \(\a\) map. Then \(\mf{p}\mc{O}\) is \(T_{\a}\)-invariant because it is an ideal of \(\mc{O}\). This induces a multiplication by \(\conj{\a}\) map \(T_{\conj{\a}}\) on \(\mc{O}/\mf{p}\mc{O}\) as a \(\F_{\mf{p}}\)-vector space. As the classes \(\conj{\a_{1}},\ldots,\conj{\a_{n}}\) are a basis for \(\mc{O}/\mf{p}\mc{O}\) as a \(\F_{\mf{p}}\)-vector space, it follows that
      \[
        \Trace_{\mc{O}/\smc{O}}(\a) \equiv \Trace_{(\mc{O}/\mf{p}\mc{O})/\F_{\mf{p}}}(\conj{\a}) \tmod{\mf{p}}.
      \]
      Therefore it is yet further equivalent to show
      \[
        \Trace_{(\mc{O}/\mf{p}\mc{O})/\F_{\mf{p}}}(\conj{\a}) \equiv 0 \tmod{\mf{p}},
      \]
      for all \(\a \in \mf{A}\). To prove this, observe that \(\conj{\a}\) runs over to the subring \(\mf{A}/\mf{p}\mc{O}\) of \(\mc{O}/\mf{p}\mc{O}\) as \(\a\) runs over \(\mf{A}\). By construction, \(\mf{A}\) is divisible by every prime factor of \(\mf{p}\mc{O}\) and so a power, say \(k\), of \(\mf{A}\) is divisible by \(\mf{p}\mc{O}\). But this means \(T_{\conj{\a}}^{k}\) is the zero operator so that it is nilpotent. As nilpotent maps have trace zero, \(\Trace_{(\mc{O}/\mf{p}\mc{O})/\F_{\mf{p}}}(\conj{\a}) = 0\). But then \(\Trace_{\mc{O}/\smc{O}}(\a) \equiv 0 \tmod{p}\) and, as \(\a\) was arbitrary, the claim is justified. We now prove (i) and (ii):
      \begin{enumerate}[label*=(\roman*)]
        \item Begin by writing \(\mf{p}\mc{O} = \mf{P}^{e_{\mf{p}}(\mf{P})}\mf{A}\) for some integral ideal \(\mf{A}\) of \(\mc{O}\) so that \(\mf{A}\) is relatively prime to \(\mf{P}\). Arguing as before, we find that \(\mf{P}^{e_{\mf{p}}(\mf{P})} \mid \mf{D}_{\mc{O}/\smc{O}}\) if and only if
        \[
          \Trace_{(\mc{O}/\mf{p}\mc{O})/\F_{\mf{p}}}(\conj{\a}) = 0,
        \]
        for all \(\a \in \mf{A}\). Since \(\mf{P}^{e_{\mf{p}}(\mf{P})}\) and \(\mf{A}\) are relatively prime, the Chinese remainder theorem implies
        \[
          \mc{O}/\mf{p}\mc{O} \cong (\mc{O}/\mf{P}^{e_{\mf{p}}(\mf{P})}) \op (\mc{O}/\mf{A}).
        \]
        So there exists \(\b,\g \in \mc{O}\) such that \(\b \equiv \a \tmod{\mf{P}^{e_{\mf{p}}(\mf{P})}}\) and \(\g \equiv \a \tmod{\mf{A}}\). Then \cref{equ:trace_and_norm_direct_sums} implies
        \[
          \Trace_{(\mc{O}/\mf{p}\mc{O})/\F_{\mf{p}}}(\conj{\a}) = \Trace_{(\mc{O}/\mf{P}^{e_{\mf{p}}(\mf{P})})/\F_{\mf{p}}}(\conj{\b})+\Trace_{(\mc{O}/\mf{A})/\F_{\mf{p}}}(\conj{\g}).
        \]
        As \(\a \in \mf{A}\), \(\conj{\g} = 0\) and so \(\Trace_{(\mc{O}/\mf{A})/\F_{\mf{p}}}(\conj{\g}) = 0\). Therefore it is further equivalent to show
        \[
          \Trace_{(\mc{O}/\mf{P}^{e_{\mf{p}}(\mf{P})})/\F_{\mf{p}}}(\conj{\b}) = 0,
        \]
        for all \(\b \in \mc{O}\). We will show that this occurs if and only if \(e_{\mf{p}}(\mf{P}) \equiv 0 \tmod{\mf{p}}\). So let \(\b \in \mc{O}\). By \cref{lem:quotient_by_prime_power_isomorphism}, we have an isomorphism
        \[
          \mc{O}/\mf{P}^{e_{\mf{p}}(\mf{P})} \cong \bigop_{0 \le e \le e_{\mf{p}}(\mf{P})-1}\mc{O}/\mf{P}.
        \]
        Therefore there exists \(\b_{e} \in \mc{O}\) such that \(\b \equiv \b_{e} \tmod{\mf{P}}\) for all \(e\). But then \(\Trace_{\F_{\mf{P}}/\F_{\mf{p}}}(\conj{\b}_{e}) = \Trace_{\F_{\mf{P}}/\F_{\mf{p}}}(\conj{\b})\) for all \(e\), and combining with \cref{equ:trace_and_norm_direct_sums} gives
        \[
          \Trace_{(\mc{O}/\mf{P}^{e_{\mf{p}}(\mf{P})})/\F_{\mf{p}}}(\conj{\b}) = e_{\mf{p}}(\mf{P})\Trace_{\F_{\mf{P}}/\F_{\mf{p}}}(\conj{\b}),
        \]
        which we recall is an element of \(\F_{\mf{p}}\). As the residue class extensions are assumed to be separable, \cref{lem:trace_is_nondegenerate} implies that \(\Trace_{\F_{\mf{P}}/\F_{\mf{p}}}(\conj{\b})\) cannot be zero for all \(\b \in \mc{O}\). So it must be the case that \(\Trace_{(\mc{O}/\mf{P}^{e_{\mf{p}}(\mf{P})})/\F_{\mf{p}}}(\conj{\b}) = 0\) for all \(\b \in \mc{O}\) if and only if \(e_{\mf{p}}(\mf{P}) \equiv 0 \tmod{\mf{p}}\). This proves (i).
        \item As we have already shown \(\mf{P}^{e_{\mf{p}}(\mf{P})-1} \mid \mf{D}_{\mc{O}/\smc{O}}\), (ii) follows from (i).
      \end{enumerate}
    \end{proof}

    Note that \cref{thm:ramifies_if_and_only_if_divides_the_different} does not tell us the degree to which \(\mf{P}\) divides \(\mf{D}_{\mc{O}/\smc{O}}\) in the case \(e_{\mf{p}}(\mf{P}) \equiv 0 \tmod{p}\). It only tells us that the degree is at least \(e_{\mf{p}}(\mf{P})\). Also, only finitely many primes can ramify in \(L\) as a corollary of \cref{thm:ramifies_if_and_only_if_divides_the_different}:

    \begin{corollary}\label{cor:finitely_many_primes_ramify_L}
      Let \(\mc{O}/\smc{O}\) be a Dedekind extension of a finite separable extension \(L/K\) and assume that all residue class extensions are separable. Then only finitely many primes ramify in \(L\).
    \end{corollary}
    \begin{proof}
      There are only finitely many prime factors of \(\mf{D}_{\mc{O}/\smc{O}}\) by the prime factorization of fractional ideals. Therefore only finitely many primes can ramify in \(L\) by \cref{thm:ramifies_if_and_only_if_divides_the_different}.
    \end{proof}

    Similarly, a prime \(\mf{p}\) of \(\smc{O}\) ramifies in \(\mc{O}/\smc{O}\) if and only if it divides the discriminant \(\mf{d}_{\mc{O}/\smc{O}}\) provided the residue class extensions are separable:

    \begin{theorem}\label{thm:ramifies_if_and_only_if_divides_the_discriminant}
      Let \(\mc{O}/\smc{O}\) be a Dedekind extension of a degree \(n\) separable extension \(L/K\) and assume that all residue class extensions are separable. Then a prime \(\mf{p}\) of \(\smc{O}\) ramifies in \(\mc{O}/\smc{O}\) if and only if \(\mf{p}\) divides \(\mf{d}_{\mc{O}/\smc{O}}\).
    \end{theorem}
    \begin{proof}
      As the residue class extensions are assumed to be separable, \(\mf{p}\) ramifies in \(\mc{O}/\smc{O}\) if and only if \(e_{\mf{p}}(\mf{P}) \ge 2\) for some prime \(\mf{P}\) above \(\mf{p}\). In view of \cref{prop:ramification_setup_respects_localization,prop:different_and_discriminant_respect_localization}, it suffices to assume \(\mc{O}/\smc{O}\) is a local Dedekind extension. Therefore \(\smc{O}\) is a discrete valuation ring, \(\mc{O}\) is a principal ideal domain, and \(\mc{O}/\smc{O}\) admits an integral basis \(\a_{1},\ldots,\a_{n}\) making \(\mc{O}\) a free \(\smc{O}\)-module of rank \(n\). Then, as we have see, \(\mf{d}_{\mc{O}/\smc{O}}\) is principal and
      \[
        \mf{d}_{\mc{O}/\smc{O}} = d_{\smc{O}}(\mc{O})\smc{O}.
      \]
      Therefore \(\mf{p}\) divides \(\mf{d}_{\mc{O}/\smc{O}}\) if and only if \(d_{\smc{O}}(\mc{O}) \equiv 0 \tmod{\mf{p}}\). Since \(\conj{\a_{1}},\ldots,\conj{\a_{n}}\) is a basis for \(\mc{O}/\mf{p}\mc{O}\) as a \(\F_{\mf{p}}\)-vector space, it follows that
      \[
        d_{\smc{O}}(\mc{O}) \equiv d_{\F_{\mf{p}}}(\mc{O}/\mf{p}\mc{O}) \pmod{\mf{p}}.
      \]
      So \(\mf{p}\) divides \(\mf{d}_{\mc{O}/\smc{O}}\) if and only if \(d_{\F_{\mf{p}}}(\mc{O}/\mf{p}\mc{O}) = 0\). Now suppose \(\mf{p}\mc{O}\) has prime factorization
      \[
        \mf{p}\mc{O} = \mf{P}_{1}^{e_{\mf{p}}(\mf{P}_{1})} \cdots \mf{P}_{r}^{e_{\mf{p}}(\mf{P}_{r})}.
      \]
      As the \(\mf{P}_{1}^{e_{\mf{p}}(\mf{P}_{1})},\ldots,\mf{P}_{r}^{e_{\mf{p}}(\mf{P}_{r})}\) are pairwise relatively prime, the Chinese remainder theorem gives
      \[
        \mc{O}/\mf{p}\mc{O} \cong \bigop_{1 \le i \le r} \mc{O}/\mf{P}_{i}^{e_{\mf{p}}(\mf{P}_{i})}.
      \]
      Then \cref{prop:discriminant_and_direct_sums} further implies
      \[
        d_{\F_{\mf{p}}}(\mc{O}/\mf{p}\mc{O}) = \prod_{1 \le i \le r}d_{\F_{\mf{p}}}(\mc{O}/\mf{P}_{i}^{e_{\mf{p}}(\mf{P}_{i})}),
      \]
      Therefore \(d_{\F_{\mf{p}}}(\mc{O}/\mf{p}\mc{O}) = 0\) if and only if \(d_{\F_{\mf{p}}}(\mc{O}/\mf{P}_{i}^{e_{\mf{p}}(\mf{P}_{i})}) = 0\) for some \(i\). We will prove that this occurs if and only if \(e_{\mf{p}}(\mf{P}_{i}) \ge 2\) which will complete the proof because this is exactly when \(\mf{p}\) ramifies. So let \(\mf{P}\) be above \(\mf{p}\) and first suppose \(e_{\mf{p}}(\mf{P}) \ge 2\). Then we need to show \(d_{\F_{\mf{p}}}(\mc{O}/\mf{P}^{e_{\mf{p}}(\mf{P})}) = 0\). By uniqueness of prime factorizations of fractional ideals, there exists \(\b_{1} \in \mf{P}^{e_{\mf{p}}(\mf{P})-1}-\mf{P}^{e_{\mf{p}}(\mf{P})}\). Then \(\b_{1}^{2} \in \mf{P}^{2(e_{\mf{p}}(\mf{P})-1)} \subseteq \mf{P}^{e_{\mf{p}}(\mf{P})}\) because \(e_{\mf{p}}(\mf{P}) \ge 2\). By construction, \(\conj{\b_{1}}\) in \(\mc{O}/\mf{P}^{e_{\mf{p}}(\mf{P})}\) is nonzero and such that \(\conj{\b_{1}}^{2} = 0\). Since \(\mc{O}/\mf{P}^{e_{\mf{p}}(\mf{P})}\) is a \(\F_{\mf{p}}\)-vector space of dimension \(f_{\mf{p}}(\mf{P})\), there exists a basis of the form \(\conj{\b_{1}},\ldots,\conj{\b_{f_{\mf{p}}(\mf{P})}}\). Now
      \[
        \Trace_{(\mc{O}/\mf{P}^{e_{\mf{p}}(\mf{P})})/\F_{\mf{p}}}(\conj{\b_{1}\b_{j}}) = 0,
      \]
      for \(1 \le j \le f_{\mf{p}}(\mf{P})\) since \(T_{\conj{\b_{1}}\conj{\b_{j}}}\) is nilpotent because \(T_{\conj{\b_{1}}\conj{\b_{j}}}^{2}\) is the zero operator as \(\conj{\b_{1}}^{2} = 0\). But then the first row of \(\Trace_{(\mc{O}/\mf{P}^{e_{\mf{p}}(\mf{P})})/\F_{\mf{p}}}(\conj{\b_{1}},\ldots,\conj{\b_{f_{\mf{p}}(\mf{P})}})\) is zero and hence \(d_{\F_{\mf{p}}}(\mc{O}/\mf{P}^{e_{\mf{p}}(\mf{P})}) = 0\). Now suppose \(e_{\mf{p}}(\mf{P}) = 1\). Then it remains to show \(d_{\F_{\mf{p}}}(\F_{\mf{P}}) \neq 0\). Since the residue class extension \(\F_{\mf{P}}/\F_{\mf{p}}\) is separable by assumption, \cref{prop:discriminant_not_zero} implies \(d_{\F_{\mf{p}}}(\F_{\mf{P}}) \neq 0\). This completes the proof.
    \end{proof}

    As a corollary of \cref{thm:ramifies_if_and_only_if_divides_the_discriminant} we see that only finitely many primes can ramify in \(K\):

    \begin{corollary}\label{cor:finitely_many_primes_ramify_K}
      Let \(\mc{O}/\smc{O}\) be a Dedekind extension of a degree \(n\) separable extension \(L/K\) and assume that all residue class extensions are separable. Then only finitely many primes ramify in \(K\).
    \end{corollary}
    \begin{proof}
      There are only finitely many prime factors of \(\mf{d}_{\mc{O}/\smc{O}}\) by the prime factorization of fractional ideals. Therefore only finitely many primes can ramify in \(K\) by \cref{thm:ramifies_if_and_only_if_divides_the_discriminant}.
    \end{proof}

    Now consider the case of a number field \(K\) of degree \(n\). The \textbf{complement}\index{complement} \(\mf{C}_{K}\) of \(K\) is the complement of \(\mc{O}_{K}/\Z\), the \textbf{different}\index{different} \(\mf{D}_{K}\) of \(K\) is the different of \(\mc{O}_{K}/\Z\), and the \textbf{discriminant}\index{discriminant} \(\mf{d}_{K}\) of \(K\) is the discriminant of \(\mc{O}_{K}/\Z\). Since \(\Z\) is a principal ideal domain, the \(\mf{d}_{K}\) is related to the discriminant \(\D_{K}\) by
    \[
      \mf{d}_{K} = \D_{K}\mc{O}_{K}.
    \]
    As all of the residue class extensions are separable, it follows from \cref{thm:ramifies_if_and_only_if_divides_the_different,thm:ramifies_if_and_only_if_divides_the_discriminant} that a prime of \(K\) ramifies in \(\mc{O}_{K}/\Z\) if and only if it divides \(\mf{D}_{K}\) and a prime of \(\Q\) ramifies in \(\mc{O}_{K}/\Z\) if and only if it divides \(|\D_{K}|\). Moreover, finitely many primes of \(K\) and \(\Q\) ramify by \cref{cor:finitely_many_primes_ramify_L,cor:finitely_many_primes_ramify_K}.
  \fi\iffalse\section{\todo{The Ideal Norm}}
    Let \(\mc{O}/\smc{O}\) be a Dedekind extension of a degree \(n\) extension \(L/K\). If \(\mf{P}\) a prime of \(L\) is above the prime \(\mf{p}\) of \(K\), we define the \textbf{norm}\index{norm} \(\Norm_{\mc{O}/\smc{O}}(\mf{P})\) of \(\mf{P}\) by
    \[
      \Norm_{\mc{O}/\smc{O}}(\mf{P}) = \mf{p}^{f_{\mf{p}}(\mf{P})}.
    \]
    Setting \(\Norm_{\mc{O}/\smc{O}}(\mc{O}) = \smc{O}\), we extend this multiplicatively to all fractional ideals of \(\mc{O}\). This induces a homomorphism
    \[
      \Norm_{\mc{O}/\smc{O}}:I_{\mc{O}} \to I_{\smc{O}} \qquad \mf{P}_{1}^{e_{1}} \cdots \mf{P}_{r}^{e_{r}} \mapsto \mf{p}_{1}^{e_{1}f_{\mf{p}_{1}}(\mf{P}_{1})} \cdots \mf{p}_{r}^{e_{r}f_{\mf{p}_{r}}(\mf{P}_{r})},
    \]
    called the \textbf{ideal norm}\index{ideal norm} of \(\mc{O}/\smc{O}\). It follows from the fundamental equality and multiplicatively of the ideal norm that
    \[
      \Norm_{\mc{O}/\smc{O}}(\mf{f}\mc{O}) = \mf{f}^{n},
    \]
    for any fractional ideal \(\mf{f}\) of \(\smc{O}\). 

    \begin{remark}
      The ideal norm can be though of as an almost inverse to the mapping that sends a fractional ideal \(\mf{f}\) of \(\smc{O}\) to the fractional ideal \(\mf{f}\mc{O}\) of \(\mc{O}\).
    \end{remark}

    In the case of a degree \(n\) number field \(K\), the \textbf{ideal norm}\index{ideal norm} \(\Norm_{K}\) of \(K\) is the ideal norm of \(\mc{O}_{K}/\Z\). As \(\Z\) is a principal ideal domain, every fractional ideal is principal. Therefore, \(\Norm_{K}(\mf{f})\) is generated by an \(r_{\mf{f}} \in \Q^{\ast}\) for every fractional ideal \(\mf{f}\) of \(\mc{O}_{K}\). We define the \textbf{norm}\index{norm} \(\Norm_{K}(\mf{f})\) of \(\mf{f}\) by
    \[
      \Norm_{K}(\mf{f}) = |r_{\mf{f}}|.
    \]
    Since the ideal norm is multiplicative, we obtain a homomorphism
    \[
      \Norm_{K}:I_{K} \to \Q^{\ast} \qquad \mf{f} \mapsto |r_{\mf{f}}|,
    \]
    called the \textbf{norm}\index{norm} of \(K\). For an integral ideal \(\mf{a}\), there is a simple relationship between the norm \(\mf{a}\) and the quotient \(\mc{O}_{K}/\mf{a}\):

    \begin{proposition}\label{prop:norm_and_field_norms_are_identical}
      Let \(K\) be a number field of degree \(n\). Then for any integral ideal \(\mf{a}\), we have
      \[
        \Norm_{K}(\mf{a}) = |\mc{O}_{K}/\mf{a}|.
      \]
      Moreover, any \(\a \in \mc{O}_{K}\) satisfies
      \[
        \Norm_{K}(\a\mc{O}_{K}) = |\Norm_{K}(\a)|.
      \]
    \end{proposition}
    \begin{proof}
      Since \(K\) admits an integral basis, \(\mc{O}_{K}\) is a free abelian group of rank \(n\) as is any fractional ideal. In particular, \(|\mc{O}_{K}/\mf{a}|\) is finite. As the Chinese remainder theorem implies
      \[
        |\mc{O}_{K}/\mf{a}\mf{b}| = |\mc{O}_{K}/\mf{a}||\mc{O}_{K}/\mf{b}|,
      \]
      whenever \(\mf{a}\) and \(\mf{b}\) are relatively prime, it suffices to prove the claim in the case of a prime power. So let \(\mf{p}\) be a prime of \(\mc{O}_{K}\) above \(p\) and \(e \ge 1\). On the one hand, \(\Norm_{K}(\mf{p}^{e}) = p^{ef_{p}(\mf{p})}\) by definition of the norm. On the other hand, \cref{lem:quotient_by_prime_power_isomorphism} implies \(\mc{O}_{K}/\mf{p}^{e}\) is an \(\F_{p}\)-vector space of dimension \(ef_{p}(\mf{p})\) so that \(|\mc{O}_{K}/\mf{p}^{e}| = p^{ef_{p}(\mf{p})}\). Together, we have
      \[
        \Norm_{K}(\mf{p}^{e}) = |\mc{O}_{K}/\mf{p}^{e}|,
      \]
      and the first statement follows. For the second statement, we just have to show that \(|\Norm_{K}(\a)| = |\mc{O}_{K}/\a\mc{O}_{K}|\). To this end, let \(\a_{1},\ldots,\a_{n}\) be an integral basis for \(K\). Writing
      \[
        \a = \sum_{1 \le i \le n}a_{i}\a_{i},
      \]
      with \(a_{i} \in \Z\), we see that \(a_{1}\a_{1},\ldots,a_{n}\a_{n}\) is a basis for \(\a\mc{O}_{K}\). In particular, the base change matrix from \(\a_{1},\ldots,\a_{n}\)  to this basis is a diagonal matrix with the \(a_{i}\) on the diagonal. Then on the one hand, we have \(|\mc{O}_{K}/\a\mc{O}_{K}| = |a_{1} \cdots a_{n}|\) by \cref{prop:base_change_quotient_determinant} again. On the other hand, the multiplication by \(\a\) map in terms of the basis \(a_{1}\a_{1},\ldots,a_{n}\a_{n}\) has matrix representation
      \[
        \begin{pmatrix} a_{1} & & \\ & \ddots & \\ & & a_{n} \end{pmatrix},
      \]
      and so \(\Norm_{K}(\a) = a_{1} \cdots a_{n}\). Hence
      \[
        |\mc{O}_{K}/\a\mc{O}_{K}| = |\Norm_{K}(\a)|,
      \]
      as desired.
    \end{proof}

    As a consequence of \cref{prop:norm_and_field_norms_are_identical} and Lagrange's theorem, \(\Norm_{K}(\mf{a}) \in \mf{a}\) for any integral ideal \(\mf{a}\) and therefore for every fractional ideal as well (recall every such fractional ideal is of the form \(\frac{1}{\d}\mf{a}\) for some nonzero \(\d \in \Z\) and integral ideal \(\mf{a}\)). Computing the norm of the discriminant \(\D_{K}\) is also an easy matter:

    \begin{proposition}\label{prop:norm_of_different_is_discriminant}
      Let \(K\) be a number field of degree \(n\). Then
      \[
        \Norm_{K}(\mf{D}_{K}) = |\D_{K}|.
      \]
    \end{proposition}
    \begin{proof}
      From \cref{equ:different_isomorphism_chain} we have an isomorphism
      \[
        \mc{O}_{K}/\mf{D}_{K} \cong \mc{O}_{K}^{\vee}/\mc{O}_{K}.
      \]
      Therefore \(\Norm_{K}(\mf{D}_{K}) = |\mc{O}_{K}^{\vee}/\mc{O}_{K}|\) by \cref{prop:norm_and_field_norms_are_identical}. Now let \(\a_{1},\ldots,\a_{n}\) be an integral basis for \(\mc{O}_{K}\). Then the dual basis \(\a_{1}^{\vee},\ldots,\a_{n}^{\vee}\) is a basis for \(\mc{O}_{K}^{\vee}\) and we have
      \[
        \a_{i}^{\vee} = \sum_{1 \le j \le n}\Trace_{K}(\a_{i}\a_{j})\a_{j}.
      \]
      But then the base change matrix from \(\a_{1},\ldots,\a_{n}\) to \(\a_{1}^{\vee},\ldots,\a_{n}^{\vee}\) is \(\Trace_{K}(\a_{1},\ldots,\a_{n})\). The claim follows by \cref{prop:base_change_quotient_determinant} and the definition of \(\D_{K}\).
    \end{proof}

    We can now compute the norm of a dual ideal:

    \begin{corollary}\label{cor:norm_of_different}
      Let \(K\) be a number field and \(\mf{f}\) be a fractional ideal. Then
      \[
        \Norm_{K}(\mf{f}^{\vee}) = \frac{\Norm_{K}(\mf{f}^{-1})}{|\D_{K}|}.
      \]
    \end{corollary}
    \begin{proof}
      This follows immediately from \(\mf{f}^{\vee} = \mf{f}^{-1}\mf{D}_{K}^{-1}\), multiplicativity of the norm, and \cref{prop:norm_of_different_is_discriminant}.
    \end{proof}

    Lastly, let \(a_{K}(m)\) denote the number of integral ideals of norm \(m\). Because the ideal norm is multiplicative so is \(a_{K}(m)\). Moreover, we have the following result:

    \begin{proposition}\label{equ:ideals_of_fixed_norm_bound}
      Let \(K\) be a number field of degree \(n\). Then \(a_{K}(m) \le \s_{0}(m)^{n}\).
    \end{proposition}
    \begin{proof}
      Let \(\mf{a}\) be an integral ideal of norm \(m\). First suppose \(m = p^{k}\) for some prime \(p\) and \(k \ge 0\). As there are at most \(n\) primes \(\mf{p}_{1},\ldots,\mf{p}_{n}\) above \(p\) with inertia degrees \(f_{p}(\mf{p}_{1}),\ldots,f_{p}(\mf{p}_{n})\) respectively, we have
      \[
        \Norm_{K}(\mf{a}) = p^{e_{1}f_{p}(\mf{p}_{1})} \cdots p^{e_{n}f_{p}(\mf{p}_{n})},
      \]
      for some integers \(0 \le e_{i} \le k\) for \(1 \le i \le n\). Therefore the number of possibilities is equivalent to the number of solutions
      \[
        e_{1}f_{p}(\mf{p}_{1})+\cdots+e_{n}f_{p}(\mf{p}_{n}) = k,
      \]
      which is at most \(\s_{0}(p^{k})^{n} = (k+1)^{n}\). This proves the claim in the case \(m\) is a prime power. By multiplicativity of \(a_{K}(m)\) and the divisor function, it follows that the number of integral ideals of norm \(m\) is at most \(\s_{0}(m)^{n}\) as desired.
    \end{proof}

    As a consequence of \cref{prop:sum_of_divisors_growth_rate}, we have the slightly weaker estimate \(a_{K}(n) \ll_{\e} n^{\e}\).
  \fi